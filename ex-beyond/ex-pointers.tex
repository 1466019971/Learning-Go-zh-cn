\begin{Exercise}[title={Pointers},difficulty=6]
\label{ex:pointers}

\Question
Suppose we have defined the following structure:
\begin{lstlisting}
type Person struct {
    name string
    age	 int
}
\end{lstlisting}

What is the difference between the following two lines?
\begin{lstlisting}
var p1 Person
p2 := new(Person)
\end{lstlisting}

\Question
What is the difference between the following two allocations?
\begin{lstlisting}[numbers=none]
func Set(t *T) {
    x = t
}
\end{lstlisting}
and
\begin{lstlisting}[numbers=none]
func Set(t T) {
    x= &t
}
\end{lstlisting}
\end{Exercise}

\begin{Answer}
\Question
In first line: \lstinline{var p1 Person} allocates a
\texttt{Person}-\emph{value} to \var{p1}. The type of \var{p1} is
\type{Person}.

The second line: \lstinline{p2 := new(Person)} allocates memory
and assigns a \emph{pointer} to \var{p2}. The type of \var{p2} is
\type{*Person}.

\Question
In the second function, \var{x} points to a new
(heap-allocated) variable \var{t} which contains
a copy of whatever the actual argument value is.

In the first function, \var{x} points to the same thing
that \var{t} does, which is the same thing that the actual
argument points to.

So in the second function, we have an "extra" variable
containing a copy of the interesting value.
\end{Answer}
