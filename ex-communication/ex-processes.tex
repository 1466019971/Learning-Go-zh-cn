\begin{Exercise}[title={Processes},difficulty=8]
\label{ex:processes}
\Question\label{ex:processes q1}
Write a program that takes a list of all running processes and prints
how many child processes each parent has spawned. The output should
look like:
%% For some reason the spacing in Exercise env. does weird things
\vskip\baselineskip
\begin{display}
Pid 0 has 2 children: [1 2]
Pid 490 has 2 children: [1199 26524]
Pid 1824 has 1 child: [7293]
\end{display}
\vskip\baselineskip
\begin{itemize}
\item{For acquiring the process list, you'll need to capture the output
of \verb|ps -e -opid,ppid,comm|. This output looks like:
\vskip\baselineskip
\begin{display}
  PID  PPID COMMAND
 9024  9023 zsh
19560  9024 ps
\end{display}
\vskip\baselineskip}
\item{If a parent has one child you must print \verb|child|, is there are
more than one print \verb|children|;}
\item{The process list must be numerically sorted, so you start with 
pid 0 and work your way up.}
\end{itemize}
Here is a Perl version to help you on your way (or to create complete
and utter confusion).
\lstinputlisting[caption={Processes in Perl}]{ex-communication/src/proc.pl}
\end{Exercise}

\begin{Answer}
\Question There is lots of stuff to do here. We can divide our program
up in the following sections:
\begin{enumerate}
\item{Starting \verb|ps| and capturing the output;}
\item{Parsing the output and saving the child PIDs for each PPID;}
\item{Sorting the PPID list;}
\item{Printing the sorted list to the screen}
\end{enumerate}
In the solution presented below, we've used a \type{map[int][]int},
i.e. a map indexed with integers, pointing to a slice of ints -- which holds the
PIDs. The builtin \func{append} is used to grow the integer slice. 

A possible program is: 
\lstinputlisting[caption=Processes in Go]{ex-communication/src/proc.go}
\end{Answer}
