\begin{Exercise}[title={Channels},difficulty=4]
\label{ex:channels}
\Question\label{ex:channels q1} Modify the program you created in
exercise Q\ref{ex:for-loop}
to use channels, in other words, the function called in the body
should now be a goroutine and communication should happen via
channels. You should not worry yourself on how the goroutine
terminates.

\Question\label{ex:channels q2} There are a few annoying issues left if
you resolve question \ref{ex:channels q1}. One of the problems is
that the goroutine isn't neatly cleaned up when \func{main.main()}
exits. And worse, due to a race condition between the exit of 
\func{main.main()} and \func{main.shower()} not all numbers are printed.
It should print up until 9, but sometimes it prints only to 8. Adding
a second quit-channel you can remedy both issues. Do this.\footnote{You
will need the \func{select} statement.}

\end{Exercise}

\begin{Answer}
\Question A possible program is: 
\lstinputlisting[label=go-chan,caption=Channels in Go]{ex-channels/src/for-chan.go}
We start of in the usual way, then at line 6 we create a new channel of
\key{int}s. In the next line we fire off the function \func{shower} with
the \prog{ch} variable as it argument, so that we may communicate with
it. Next we start our for-loop (lines 8-10) and in the loop
we send (with \lstinline{<-}) our number to the function (now a goroutine) \func{shower}.

In the function \func{shower} we wait (as this blocks) until we receive a number (line
15). Any received number is printed (line 16) and then continue the endless loop
started on line 14.

\Question The answer is
\lstinputlisting[label=go-quit-chan,caption=Adding extra quit channel]{ex-channels/src/for-quit-chan.go}
On line 20 we read from the quit channel and we discard the value we
read. We could have used \lstinline{q := <-quit}, but then we would have used
the variabele only once - which is illegal in Go. Another trick you
might have pulled out of your hat may be: \lstinline{\_ = <-quit}. This is
valid in Go, but the Go idiom favors the one given on line 20.
\end{Answer}
