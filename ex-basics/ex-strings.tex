\begin{Exercise}[title={Strings},difficulty=1]
\label{ex:strings}
\Question \label{ex:strings q1} Create a Go program that prints
the following (up to 100 characters):
\begin{alltt}
A
AA
AAA
AAAA
AAAAA
AAAAAA
AAAAAAA
\ldots
\end{alltt}


\Question \label{ex:strings q2} Create a program that counts
the numbers of characters/runes in this string:
\begin{alltt}
asSASA ddd dsjkdsjs dk
\end{alltt}
Make it also output the number of bytes in that string.

\Question \label{ex:string q3} Extend the program from
the previous question to replace the three runes at
position 4 with 'abc'.

\end{Exercise}

\begin{Answer}

\Question The following program is an answer to the first question.
\lstinputlisting[label=string1,caption=Strings]{ex-basics/src/string1.go}

\Question To answer this question we need some help of
the \package{string}-package. First we check the documentation
with \prog{godoc strings | less}. When we read the documentation
we notice two functions: \lstinline{func Bytes(s string) []byte} and
\lstinline{func Runes(s string) []int}. Both return values are
almost what we need, namely \type{slices}. 

So we return the length of 
them. Putting this together leads to the following program.
\lstinputlisting[label=string2,caption=Runes in strings]{ex-basics/src/string2.go}
\end{Answer}
