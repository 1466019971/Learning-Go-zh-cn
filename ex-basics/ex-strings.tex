\begin{Exercise}[title={Strings},difficulty=1]
\label{ex:strings}
\Question \label{ex:strings q1} Create a Go program that prints
the following (up to 100 characters):
\vskip\baselineskip
\begin{display}
A
AA
AAA
AAAA
AAAAA
AAAAAA
AAAAAAA
...
\end{display}
\vskip\baselineskip

\Question \label{ex:strings q2} Create a program that counts
the numbers of characters \footnote{In the UTF-8 world characters are
sometimes called \first{runes}{runes}. Mostly, when people talk about characters, they
mean 8 bit characters. As UTF-8 characters may be up to 32 bits the word rune is used.}
in this string:
\begin{display}
asSASA ddd dsjkdsjs dk
\end{display}
Make it also output the number of bytes in that string.
\emph{Hint.} Check out the \package{utf8} package.

\Question \label{ex:string q3} Extend the program from
the previous question to replace the three runes at
position 4 with 'abc'.

\Question \label{ex:string q4} Write a Go program
that reverses a string, so "foobar" is printed as "raboof".
\emph{Hint.} Unfortunatally you need to know about
conversion already. See section "\titleref{sec:conversions}" on
\pageref{sec:conversions}".

\end{Exercise}

\begin{Answer}

\Question This program is a solution:

\lstinputlisting[label=string1,caption=Strings]{ex-basics/src/string1.go}

\Question To answer this question we need some help of
the \package{utf8} package. First we check the documentation
with \prog{godoc utf8 | less}. When we read the documentation
we notice \lstinline{func RuneCount(p []byte) int}. Secondly
we can convert \emph{string} to a \type{byte} slice with
\begin{lstlisting}
str := "hello"
b   := []byte(str) |\coderemark{Conversion, see %
page \pageref{sec:conversions}}|
\end{lstlisting}

Putting this together leads to the following program.

\begin{minipage}{\textwidth}
\lstinputlisting[label=src:string2,caption=Runes in strings]{ex-basics/src/string2.go}
\end{minipage}

\Question Reversing a string can be done as follows:

\begin{minipage}{\textwidth}
\lstinputlisting[label=src:stringrev,caption=Reverse a string]{ex-basics/src/stringrev.go}
\end{minipage}

\end{Answer}
