\begin{Exercise}[title={For-loop},difficulty=1]
\label{ex:for-loop}
\Question \label{ex:for-loop q1} Create a simple loop with the \key{for} construct. Make it loop
10 times and print out the loop counter with the \package{fmt} package.

\Question \label{ex:for-loop q2} Put the body of the loop in a separate function.

\Question \label{ex:for-loop q3} Rewrite the loop from 1. to use \key{goto}. The
keyword \key{for} may not be used.
\end{Exercise}

\begin{Answer}

\Question There are a multitude of possibilities, 
one of the solutions could be:
\lstinputlisting[label=src:for,caption=Simple for-loop]{ex-basics/src/for.go}
Lets compile this on an Intel 386 Linux machine and look at the
output.
\vskip\baselineskip
\begin{display}
\pr 8g for.go && 8l -o for for.8
\pr ./for
0
1
.
.
.
9
\end{display}
\vskip\baselineskip

\Question Next we put the body of the 
loop - the \key{fmt.Printf} - in a separate function.
\lstinputlisting[label=src:for-func,caption=Loop calls function]{ex-basics/src/for-func.go}
The presented program should be self explanatory. Note however the
"\lstinline{j int}" instead of the more usual "\lstinline{int j}" in the
function definition.
\end{Answer}
