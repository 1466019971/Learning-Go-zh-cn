\begin{lstlisting}[caption=A function definition,label=src:function definition]
|\begin{tikzpicture}[overlay]
%
\draw [->,thick] (0.5,-2.20) node [left] %
{\longremark{The keyword \key{func} is used to declare a function;}} %
to (0.5,-1.50); %
%
\draw [->,thick] (1.5,-2.20) node [left] %
{\longremark{A function can be defined to work on a specific type, a %
more common name for such a function is \index{method}{method}. This is optional;}} %
to (1.5,-1.50); %
%
\draw [->,thick] (2.9,-2.20) node [left] %
{\longremark{\emph{funcname} is the name of your function;}} %
to (2.9,-1.50); %
%
\draw [->,thick] (4.0,-2.20) node [left] %
{\longremark{The variable \var{q} of type \type{int} is the parameter %
for this function;}} %
to (4.0,-1.50); %
%
\draw [->,thick] (5.4,-2.20) node [left] %
{\longremark{The variables \var{r} and \var{s} are the %
\index{named return parameters}{named return parameters} for this function. %
Note that functions in Go can have multiple return values. See section %
\ref{sec:multiple return} for more information;}} %
to (5.4,-1.50); %
%
\draw [->,thick] (6.8,-2.20) node [left] %
{\longremark{This is the function's body.}}%
to (6.8,-1.50); %
\end{tikzpicture}|
type mytype int	    |\coderemark{Define a new type, see section %
\ref{sec:defining your own} in chapter \ref{chap:beyond}}|

func (p mytype) funcname(q int) (r,s int) { return 0,0 }
||
\end{lstlisting}
