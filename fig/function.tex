\begin{lstlisting}[caption=函数定义,label=src:function definition]
|\begin{tikzpicture}[overlay]
\ubrace{0.6,-1.5}{0.0,-1.5}{保留字 \key{func} 用于定义一个函数;}
%
\ubrace{2.2,-1.5}{0.8,-1.5}{函数可以定义用于特定的类型,%
这类函数更加通俗的称呼是 \index{method}{method}。这部分称作 %
\first{\emph{receiver}}{receiver} 而它是可选的。它将在 %
\ref{chap:interfaces} 章使用;}
%
\ubrace{3.4,-1.5}{2.4,-1.5}{\emph{funcname} 是你函数的名字;}
%
\ubrace{4.5,-1.5}{3.6,-1.5}{\type{int} 类型的变量 \var{q} 是输入参数。%
参数用 \first{\emph{pass-by-value}}{pass-by-value} 方式传递,意味着它们会被复制。%
但是留意引用类型(slice、channel、map 和 interface)用 %
\first{\emph{pass-by-reference}}{pass-by-reference} 传递,%
虽然你并没有在代码里直接看到指针;}
%
\ubrace{6.0,-1.5}{4.9,-1.5}{%
变量 \var{r} 和 \var{s} 是这个函数的 \index{named return parameters}{named return parameters}。%
注意在 Go 的函数中可以返回多个值。参阅 "\titleref{sec:multiple return}" 在 \pageref{sec:multiple return}。%
如果想要返回无命名的参数,只需要提供类型:\lstinline{(int,int)}。%
如果只有一个返回值,可以省略圆括号。如果函数是一个子过程,并且没有任何返回值,也可以省略这些内容;}
%
\ubrace{8.2,-1.5}{6.3,-1.5}{这是函数体,注意 \func{return} 是一个语句,所以包裹参数的括号是可选的。}
\end{tikzpicture}|
type mytype int	|\coderemark{新的类型,参阅 \ref{chap:beyond} 章节}|

func (p mytype) funcname(q int) (r,s int) { return 0,0 }
||
\end{lstlisting}
