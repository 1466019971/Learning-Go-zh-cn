\begin{lstlisting}[caption=Introspection using reflection,label=src:introspection]
|\begin{tikzpicture}[overlay]
\ubrace{3.2,-4.2}{2.2,-4.2}{%
We are dealing with a \type{Type} and according %
to the documentation\footnote{\texttt{go doc reflect}}:%
\begin{quote} %
// Elem returns a type's element type. \\ %
// It panics if the type's Kind is not Array, Chan, Map, Ptr, or Slice.\\ %
\func{Elem() Type} %
\end{quote} %
So on \var{t} we use \func{Elem()} to get the value the pointer %
points to;}%
\ubrace{4.9,-4.2}{3.8,-4.2}{We have now dereferenced the pointer %
and are "inside" our structure. %
We now use \func{Field(0)} to access the zeroth field;}%
%
\ubrace{6.0,-4.2}{5.1,-4.2}{%
The struct \type{StructField} has a \var{Tag} member which %
returns the tag-name as a string. So on the $0^{th}$ field we can %
unleash \func{.Tag} to access this name: \texttt{Field(0).Tag}. This %
gives us \texttt{namestr}.}
%
\end{tikzpicture}|
type Person struct {
    name string "namestr" |\coderemark{\texttt{"namestr"} is the tag}|
    age  int
}

func ShowTag(i interface{}) {   |\coderemark{Called with *Person}|
 switch t := reflect.TypeOf(i); t.Kind() {
 case reflect.Ptr: |\coderemark{A pointer, thus \var{reflect.Ptr}}|
   tag := t.Elem().Field(0).Tag
\end{lstlisting}
