\begin{Exercise}[title={Minimum and maximum},difficulty=0]
\label{ex:minmax}
\Question\label{ex:minmax q1} Write a function that finds the
maximum value in an \type{int} slice (\type{[]int}).

\Question\label{ex:minmax q2} Write a function that finds the
minimum value in an \type{int} slice (\type{[]int}).

\end{Exercise}

\begin{Answer}
\Question This function returns the largest int in the slice \var{l}:
\begin{lstlisting}
func max(l []int) (max int) {   |\longremark{We use a named return parameter;}|
        max = l[0]      
        for _, v := range l {   |\longremark{Loop over \var{l}. The index of the element is %
not important;}|
                if v > max {    |\longremark{If we find a new maximum, remember it;}|
                        max = v 
                }   
        }   
        return  |\longremark{A ``lone'' return, the current value of \var{max} is now returned.}|
}
\end{lstlisting}
\showremarks

\Question This function returns the smallest int in the slice \var{l}. It is almost identical to \func{max}:
\begin{lstlisting}
func min(l []int) (min int) {
        min = l[0]
        for _, v := range l { 
                if v < min {
                        min = v 
                }   
        }   
        return
}
\end{lstlisting}
The interested reader may combine \func{max} and \func{min} into one function with a selector
that lets you choose between the minimum or the maximum, or one that returns both values.
\end{Answer}
