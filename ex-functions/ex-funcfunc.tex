\begin{Exercise}[title={函数返回一个函数},difficulty=1]
\label{ex:function}
\Question\label{ex:function q1} 编写一个函数返回另一个函数,返回的函数的作用是对一个整数~$+2$。
函数的名称叫做~\func{plusTwo}。
然后可以像下面这样使用:
\begin{lstlisting}
p := plusTwo()
fmt.Printf("%v\n", p(2))
\end{lstlisting}
应该打印 4。
参阅第~\pageref{sec:callbacks} 页的~``\titleref{sec:callbacks}''小节了解更多相关信息。

\Question\label{ex:function q2} 使~\ref{ex:function q1} 中的函数更加通用化,
创建一个~\func{plusX(x)} 函数,返回一个函数用于对整数加上~\var{x}。
\end{Exercise}

\begin{Answer}
\Question
\begin{lstlisting}
func main() {
        p2 := plusTwo()
        fmt.Printf("%v\n",p2(2))
}

func plusTwo() func(int) int { |\longremark{定义新的函数返回一个函数。%
看看你写的跟要表达的意思是如何的;}|
        return func(x int) int { return x + 2 } |\longremark{函数符号,%
在返回语句中定义了一个~+2 的函数。}|
}
\end{lstlisting}
\showremarks

\Question
这里我们使用闭包:
\begin{lstlisting}
func plusX(x int) func(int) int { |\longremark{再次定义一个函数返回一个函数;}|
        return func(y int) int { return x + y } |\longremark{在函数符号中使用\emph{局部}变量 \var{x}。}|
}
\end{lstlisting}
\showremarks
\end{Answer}
