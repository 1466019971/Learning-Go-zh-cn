\begin{Exercise}[title={Map function},difficulty=4]
\label{ex:map function}
A \func{map()}-function is a function that takes
a function and a list. The function is applied to 
each member in the list and a new list containing
these calculated values is returned.
Thus: 
$$ map(f(), (a_1,a_2,\ldots,a_{n-1},a_n)) =  (f(a_1), f(a_2),\ldots,f(a_{n-1}), f(a_n)) $$
\Question \label{ex:map function q1} Write a simple
\func{map()}-function in Go. It is sufficient
for this function only to work for \type{int}s.
\Question \label{ex:map function q2} Expand your code to also work on a list of \type{strings}.

\end{Exercise}

\begin{Answer}

\Question 
\begin{lstlisting}[caption=A \func{Map} function]
func Map(f func(int) int, l []int) []int {
        j := make([]int, len(l))
        for k, v := range l {
                j[k] = f(v)
        }
        return j
}

func main() {
        m := []int{1, 3, 4}
        f := func(i int) int {
                return i * i
        }
        fmt.Printf("%\v", (Map(f, m)))
}
\end{lstlisting}

\Question Answer to question but now with strings
\end{Answer}


