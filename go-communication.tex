Communication with the outside world:
\begin{itemize}
\item{Files}
\item{Input/Output, Stdin, Stdout}
\item{Networking}
\item{Starting other programs}
\item{Forking}
\end{itemize}

\section{Files}
Reading from (and writing to) files is easy in Go. This program
only uses the \package{os} package to read data from the file \file{/etc/passwd}.
\lstinputlisting[numbers=right,caption=Reading from a file
(unbufferd),label=src:read]{src/file.go}
If you want to use \first{buffered} IO there is the \package{bufio} package:
\lstinputlisting[numbers=right,caption=Reading from a file
(bufferd),label=src:bufread]{src/buffile.go}

On line 12 we create a \type{bufio.Reader} from \var{f} which is of
type \type{*File}. \func{NewReader} expects an \type{io.Reader}, so you
might think this will fail. But it doesn't. An \type{io.Reader} is
defined as:
\begin{lstlisting}
type Reader interface {
    Read(p []byte) (n int, err os.Error)
}
\end{lstlisting}
So \emph{anything} that has such a \func{Read()} function implements this
interface. And from listing \ref{src:read} (line 10) we can see
the \type{*File} indeed does so. 
