Communication with the outside world:
\begin{itemize}
\item{Files}
\item{Input/Output, Stdin, Stdout}
\item{Networking}
\item{Starting other programs}
\item{Forking}
\end{itemize}

\section{Files}
Reading from (and writing to) files is easy in Go. This program
only uses the \package{os} package to read data from the file \file{/etc/passwd}.
\lstinputlisting[numbers=right,caption=Reading from a file (unbufferd),label=src:read]{src/file.go}
If you want to use \first{buffered} IO there is the \package{bufio} package:
%%\lstinputlisting[numbers=right,caption=Reading from a file (bufferd),label=src:bufread]{src/buffile.go}

On line 12 we create a \type{bufio.Reader} from \var{f} which is of
type \type{*File}. \func{NewReader} expects an \type{io.Reader}, so you
might think this will fail. But it doesn't. An \type{io.Reader} is
defined as:
\begin{lstlisting}
type Reader interface {
    Read(p []byte) (n int, err os.Error)
}
\end{lstlisting}
So \emph{anything} that has such a \func{Read()} function implements this
interface. And from listing \ref{src:read} (line 10) we can see
the \type{*File} indeed does so. 

\section{Executing commands}
The \package{exec} package has function to run external commands, and it the premier way to
execute commands from within a Go program. We start commands with 
the \func{Run} function:
\begin{lstlisting}
func Run(argv0 string, argv, envv []string, dir string, stdin, stdout, stderr int) (p *Cmd, err os.Error)
\end{lstlisting}
\begin{quote}
Run starts the binary prog running with
arguments \var{argv} and environment \var{envv}.
It returns a pointer to a new \type{Cmd} representing the command or an error.
\end{quote}
Lets execute \verb|ls -l|:
\begin{lstlisting}
import "exec"

cmd, err := exec.Run("/bin/ls", []string{"ls", "-l"}, nil, "", exec.DevNull, exec.DevNull, exec.DevNull})
\end{lstlisting}
In the \package{os} package we find the \func{ForkExec} function. This
is another way (but more low level) to start executables.\footnote{There is talk on
the Gonuts mailing list about separating \func{Fork} and
\func{Exec}.} 
The prototype for \func{ForkExec} is:
\begin{lstlisting}
func ForkExec(argv0 string, argv []string, envv []string, dir string, fd []*File) (pid int, err Error)
\end{lstlisting}
With the following documentation:
\begin{quote}
\func{ForkExec} forks the current process and invokes \func{Exec} with the
file, arguments, and environment specified by \var{argv0}, \var{argv}, and
\var{envv}. It returns the process id of the forked process and an
\type{Error}, if any. The \var{fd} array specifies the file descriptors to be
set up in the new process: \var{fd[0]} will be Unix file descriptor 0 (standard
input), \var{fd[1]} descriptor 1, and so on.  A \var{nil} entry will cause the
child to have no open file descriptor with that index.  If \var{dir} is not
empty, the child chdirs into the directory before execing the program.
\end{quote}
Suppose we want to execute \verb|ls -l| again:
\begin{lstlisting}
import "os"

pid, err := os.ForkExec("/bin/ls", []string{"ls", "-l"}, nil, "", []*os.File{ os.Stdin, os.Stdout, os.Stderr})
defer os.Wait(pid, os.WNOHANG) |\coderemark{Otherwise you create a zombie}|
\end{lstlisting}


\section{Exercises}
\begin{Exercise}[title={Processes},difficulty=8]
\label{ex:processes}
\Question\label{ex:processes q1}
Write a program that takes a list of all running processes and prints
how many child processes each parent has spawned. The output should
look like:
%% For some reason the spacing in Exercise env. does weird things
\vskip\baselineskip
\begin{display}
Pid 0 has 2 children: [1 2]
Pid 490 has 2 children: [1199 26524]
Pid 1824 has 1 child: [7293]
\end{display}
\vskip\baselineskip
\begin{itemize}
\item{For acquiring the process list, you'll need to capture the output
of \verb|ps -e -opid,ppid,comm|. This output looks like:
\vskip\baselineskip
\begin{display}
  PID  PPID COMMAND
 9024  9023 zsh
19560  9024 ps
\end{display}
\vskip\baselineskip}
\item{If a parent has one child you must print \verb|child|, is there are
more than one print \verb|children|;}
\item{The process list must be numerically sorted, so you start with 
pid 0 and work your way up.}
\end{itemize}
Here is a Perl version to help you on your way (or to create complete
and utter confusion).
\lstinputlisting[caption={Processes in Perl}]{ex-communication/src/proc.pl}
\end{Exercise}

\begin{Answer}
\Question There is lots of stuff to do here. We can divide our program
up in the following sections:
\begin{enumerate}
\item{Starting \verb|ps| and capturing the output;}
\item{Parsing the output and saving the child PIDs for each PPID;}
\item{Sorting the PPID list;}
\item{Printing the sorted list to the screen}
\end{enumerate}
In the solution presented below, we've opted to use
\package{container/vector} to hold the PIDs. This "list" grows
automatically.

The function \func{atoi} (lines 19 through 22) is defined to get rid of the multiple return
values of the original \func{strconv.Atoi}, so that it can be used
inside function calls that only accept one argument, as we do on lines
45, 47 and 50.

A possible program is: 
\lstinputlisting[caption=Processes in Go,numbers=right]{ex-communication/src/proc.go}
\end{Answer}


\cleardoublepage
\section{Answers}
\shipoutAnswer
