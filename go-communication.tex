\epi{``好的沟通就像是一杯刺激的浓咖啡,然后就难以入睡。''}{\textsc{ANNE MORROW LINDBERGH}}
\noindent{}在这章中将介绍 Go 中与外部通讯的通讯模块。将会了解文件、目录、网络通讯和运行其他程序。Go 的 I/O 核心是接口 \type{io.Reader} 和 \type{io.Writer}。

在 Go 中,从文件读取(或写入)是非常容易的。程序只需要使用
\package{os} 包就可以从文件 \file{/etc/passwd} 中读取数据。
\lstinputlisting[caption=从文件读取(无缓冲),label=src:read]{src/file.go}
接下来展示了如何做到这点:
\showremarks
如果想要使用\first{缓冲}{buffered} IO,则有
\package{bufio}\index{package!bufio} 包:
\lstinputlisting[caption=从文件读取(缓冲),label=src:bufread]{src/buffile.go}
\showremarks

\section{io.Reader}
在前面已经提到 \first{io.Reader}{io.Reader} 接口对于 Go 语言来说非常重要。许多(如果不是全部的话)函数需要通过 
\type{io.Reader}\index{package!io} 读取一些数据作为输入。
为了满足这个接口,只需要实现一个方法:
\func{Read(p []byte) (n int, err error)}。
写入则是(你可能已经猜到了)实现了 \func{Write} 方法的 \type{io.Writer}。
如果你让自己的程序或者包中的类型实现了 \type{io.Reader} 或者
\type{io.Writer} 接口,那么\emph{整个 Go 标准库都可以使用}这个类型!

\section{一些例子}

前面的程序将整个文件读出,但是通常情况下会希望一行一行的读取。下面的片段展示了如何实现:

\begin{lstlisting}
f, _ := os.Open("/etc/passwd"); defer f.Close()
r := bufio.NewReader(f) |\coderemark{使其成为一个 bufio,以便访问 ReadString 方法}|
s, ok := r.ReadString('\n') {|\coderemark{从输入中读取一行}|
// ... |\coderemark{\var{s} 保存了字符串,通过 \package{string} 包就可以解析它}|
\end{lstlisting}
这两个例子的相似之处展示了 Go 拥有的``脚本''化特性,例如,用 Go  编写程序感觉上类似使用动态语言(Python、Ruby、Perl 或者 PHP)。

\section{命令行参数}
\label{sec:option parsing}
来自命令行的参数在程序中通过字符串 slice \var{os.Args} 获取,导入包 \package{os} 即可。
\package{flag} 包有着精巧的接口,同样提供了解析标识的方法。这个例子是一个 DNS 查询工具:
\begin{lstlisting}
dnssec := flag.Bool("dnssec", false, "Request DNSSEC records") |\longremark{定义 \texttt{bool} 标识,%%
\texttt{-dnssec}。变量必须是指针,否则 package 无法设置其值;}|
port := flag.String("port", "53", "Set the query port")      |\longremark{类似的,\texttt{port} 选项;}|
flag.Usage = func() {   |\longremark{简单的重定义 \func{Usage} 函数,有点罗嗦;}|
    fmt.Fprintf(os.Stderr, "Usage: %s [OPTIONS] [name ...]\n", os.Args[0])
    flag.PrintDefaults() |\longremark{指定的每个标识,\func{PrintDefaults} 将输出帮助信息;}|
}
flag.Parse()   |\longremark{解析标识,并填充变量。}|
\end{lstlisting}
\showremarks
当参数被解析之后,就可以使用它们:
\begin{lstlisting}
if *dnssec {    |\coderemark{定义传入参数 \var{dnssec}}|
    // 做点啥
}
\end{lstlisting}

\section{执行命令}
\package{os/exec}\index{package!os/exec} 包有函数可以执行外部命令,这也是在 Go 中主要的执行命令的方法。
通过定义一个有着数个方法的 \var{*exec.Cmd} 结构来使用。

执行 \verb|ls -l|:
\begin{lstlisting}
import "os/exec"

cmd := exec.Command("/bin/ls", "-l")
err := cmd.Run()
\end{lstlisting}
上面的例子运行了``ls -l'',但是没有对其返回的数据进行任何处理,
通过如下方法从命令行的标准输出中获得信息:
\begin{lstlisting}
import "exec"

cmd := exec.Command("/bin/ls", "-l")
buf, err := cmd.Output()    |\coderemark{\var{buf} 是一个 \type{[]byte}}|
\end{lstlisting}

\section{网络}
所有网络相关的类型和函数可以在 \package{net} 包中找到。这其中最重要的函数是 \func{Dial}\index{networking!Dial}。
当 \func{Dial} 到远程系统,这个函数返回 \var{Conn} 接口类型,可以用于发送或接收信息。
函数 \func{Dial} 简洁的抽象了网络层和传输层。因此 IPv4 或者 IPv6,TCP 或者 UDP 可以共用一个接口。

通过 TCP 连接到远程系统(端口 80),然后是 UDP,最后是 TCP 通过 IPv6,大致是这样
\footnote{在这个例子中,可以认为 192.0.32.10 和 2620:0:2d0:200::10 是 \url{www.example.org}。}:
\begin{lstlisting}
conn, e := Dial("tcp", "192.0.32.10:80")
conn, e := Dial("udp", "192.0.32.10:80")
conn, e := Dial("tcp", "[2620:0:2d0:200::10]:80") |\coderemark{方括号是强制的}|
\end{lstlisting}

如果没有错误(由 \var{e} 返回),就可以使用 \var{conn} 从套接字中读写。
在包 \package{net} 中的原始定义是:
\begin{quote}
// \func{Read} reads data from the connection.\\
\lstinline{Read(b []byte) (n int, err error)}
\end{quote}
这使得 \var{conn} 成为了 \type{io.Reader}。

\begin{quote}
// \func{Write} writes data to the connection.\\
\lstinline{Write(b []byte) (n int, err error)}
\end{quote}
这同样使得 \var{conn} 成为了 \type{io.Writer},事实上 \var{conn} 是 \type{io.ReadWriter}。\footnote{变量 \var{conn} 同样实现了 \func{close} 方法,这使其成为一个 \type{io.ReadWriteCloser}。}

但是这些都是隐含的低层\footnote{练习 Q\ref{ex:echo} 是关于使用这些的。},通常总是应该使用更高层次的包。
例如 \package{http} 包。一个简单的 http Get 作为例子:
\begin{lstlisting}
package main
import ( "io/ioutil"; "net/http"; "fmt" ) |\longremark{需要的导入;}|

func main() {
    r, err := http.Get("http://www.google.com/robots.txt") |\longremark{使用 http 的 \func{Get} 获取 html;}|
    if err != nil { fmt.Printf("%s\n", err.String()); return } |\longremark{错误处理;}|
    b, err := ioutil.ReadAll(r.Body)    |\longremark{将整个内容读入 \var{b};}|
    r.Body.Close()  
    if err == nil { fmt.Printf("%s", string(b)) } |\longremark{如果一切 OK 的话,打印内容。}|
}
\end{lstlisting}
\showremarks

\section{练习}
\begin{Exercise}[title={Processes},difficulty=8]
\label{ex:processes}
\Question\label{ex:processes q1}
Write a program that takes a list of all running processes and prints
how many child processes each parent has spawned. The output should
look like:
%% For some reason the spacing in Exercise env. does weird things
\vskip\baselineskip
\begin{display}
Pid 0 has 2 children: [1 2]
Pid 490 has 2 children: [1199 26524]
Pid 1824 has 1 child: [7293]
\end{display}
\vskip\baselineskip
\begin{itemize}
\item{For acquiring the process list, you'll need to capture the output
of \verb|ps -e -opid,ppid,comm|. This output looks like:
\vskip\baselineskip
\begin{display}
  PID  PPID COMMAND
 9024  9023 zsh
19560  9024 ps
\end{display}
\vskip\baselineskip}
\item{If a parent has one child you must print \verb|child|, is there are
more than one print \verb|children|;}
\item{The process list must be numerically sorted, so you start with 
pid 0 and work your way up.}
\end{itemize}
Here is a Perl version to help you on your way (or to create complete
and utter confusion).
\lstinputlisting[caption={Processes in Perl}]{ex-communication/src/proc.pl}
\end{Exercise}

\begin{Answer}
\Question There is lots of stuff to do here. We can divide our program
up in the following sections:
\begin{enumerate}
\item{Starting \verb|ps| and capturing the output;}
\item{Parsing the output and saving the child PIDs for each PPID;}
\item{Sorting the PPID list;}
\item{Printing the sorted list to the screen}
\end{enumerate}
In the solution presented below, we've opted to use
\package{container/vector} to hold the PIDs. This "list" grows
automatically.

The function \func{atoi} (lines 19 through 22) is defined to get rid of the multiple return
values of the original \func{strconv.Atoi}, so that it can be used
inside function calls that only accept one argument, as we do on lines
45, 47 and 50.

A possible program is: 
\lstinputlisting[caption=Processes in Go,numbers=right]{ex-communication/src/proc.go}
\end{Answer}


\begin{Exercise}[title={单词和字母统计},difficulty=5]
\label{ex:wc}
\Question\label{ex:wc q1}编写一个从标准输入中读取文本的小程序,
并进行下面的操作:
\begin{enumerate}
\item{计算字符数量(包括空格);}
\item{计算单词数量;}
\item{计算行数。}
\end{enumerate}
换句话说,实现一个 \prog{wc(1)}(参阅本地的手册页面),
然而只需要从标准输入读取。
\end{Exercise}

\begin{Answer}
\Question 下面是 \prog{wc(1)} 的一种实现。
\lstinputlisting[caption=wc(1) 的 Go 实现]{ex-communication/src/wc.go}
\showremarks
\end{Answer}


\begin{Exercise}[title={Uniq},difficulty=4]
\label{ex:Uniq}
\Question\label{ex:Uniq q1} Write a Go program that mimics the function
of the Unix \prog{uniq} command. This program should work as follows,
given a list with the following items: 

\begin{display}
'a' 'b' 'a' 'a' 'a' 'c' 'd' 'e' 'f' 'g'
\end{display}

it should print only those item which don't have the same successor:

\begin{display}
'a' 'b' 'a' 'c' 'd' 'e' 'f'
\end{display}
\exdisfix
Listing \ref{src:uniq} is a Perl implementation of the algorithm.
\lstinputlisting[label=src:uniq,caption=uniq(1) in Perl,language=Perl]{ex-communication/src/uniq.pl}

\end{Exercise}

\begin{Answer}
\Question The following is a uniq implementation in Go.
\lstinputlisting[caption=uniq(1) in Go]{ex-communication/src/uniq.go}
\end{Answer}


\begin{Exercise}[title={Quine},difficulty=2]
A \emph{Quine} is a program that prints itself.
\label{ex:quine}
\Question\label{ex:quine q1} Write a Quine in Go.
\end{Exercise}

\begin{Answer}
\begin{lbar}
This solution is from Russ Cox. It was posted to 
the Go Nuts mailing list.
\end{lbar}
\Question 
\begin{lstlisting}[caption=A Go quine]
/* Go quine */
package main
import "fmt"
func main() {
 fmt.Printf("%s%c%s%c\n", q, 0x60, q, 0x60)
}
var q = `/* Go quine */
package main
import "fmt"
func main() {
 fmt.Printf("%s%c%s%c\n", q, 0x60, q, 0x60)
}
var q = `
\end{lstlisting}
\end{Answer}


\begin{Exercise}[title={Echo server},difficulty=8]
\label{ex:echo}
\Question\label{ex:echo q1}
Write a simple echo server. Make it listen to TCP port number 8053 on localhost. It should
be able to read a line (up to the newline), echo back that line and then close the connection. 

\Question\label{ex:echo q2}
Make the server concurrent so that every request is taken care of in a separate
goroutine.

\end{Exercise}

\begin{Answer}
\Question
A simple echo server might be:
\lstinputlisting[caption=A simple echo server]{ex-communication/src/echo.go}

When started you should see the following:
\vskip\baselineskip
\begin{display}
\pr \user{nc 127.0.0.1 8053}
\user{Go is *awesome*}
Go is *awesome*
\end{display}

\Question
To make the connection handling concurrent we \emph{only need to change one line} in our
echo server, the line:
\begin{lstlisting}
if c, err := l.Accept(); err == nil { Echo(c) }
\end{lstlisting}
becomes:
\begin{lstlisting}
if c, err := l.Accept(); err == nil { go Echo(c) }
\end{lstlisting}
\end{Answer}


\begin{Exercise}[title={Number cruncher},difficulty=9]
\label{ex:numbercruncher}
\begin{itemize}
\item{Pick six (6) random numbers from this list:
$$1, 2, 3, 4, 5, 6, 7, 8, 9, 10, 25, 50, 75, 100$$
Numbers may be picked multiple times;}
\item{Pick one (1) random number ($i$) in the range: $1 \ldots 1000$;}
\item{Tell how, by combining the first 6 numbers (or a subset thereof)
with the operators $+$,$-$,$*$ and $/$, you can make $i$;}
\end{itemize}
An example. We have picked the numbers: 1, 6, 7, 8, 8 and 75. And $i$ is
977. This can be done in many different ways, one way is:
$$ ((((1 * 6) * 8) + 75) * 8) - 7 = 977$$ 
or
$$ (8*(75+(8*6)))-(7/1) = 977$$

\Question\label{ex:cruncher q1}
Implement a number cruncher that works like that. Make it print the
solution in a similar format (i.e. output should be infix with
parenthesis) as used above.
\Question\label{ex:cruncher q2}
Calculate \emph{all} possible solutions and show them (or only show how
many there are). In the example above there are 544 ways to do it.
\end{Exercise}

\begin{Answer}
\Question 
The following is one possibility. It uses recursion and backtracking to get
an answer.
\lstinputlisting[caption=Number cruncher]{ex-communication/src/permrec.go}

\Question
When starting \prog{permrec} we give 977 as the first argument:
\vspace{1em}
\begin{display}
\pr ./permrec 977
1+(((6+7)*75)+(8/8)) = 977  #1
...                         ...
((75+(8*6))*8)-7 = 977      #542
(((75+(8*6))*8)-7)*1 = 977  #543
(((75+(8*6))*8)-7)/1 = 977  #544
\end{display}

\end{Answer}


\begin{Exercise}[title={*Finger 守护进程},difficulty=8]
\label{ex:finger}
\Question
编写一个 finger 守护进程,可以工作于 finger(1) 命令。

来自 Debian 的包描述:
\begin{quote}
Fingerd 是一个基于 RFC 1196 \cite{RFC1196} 的简单的守护进程,它为许多站点提供了``finger''程序的接口。
这个程序支持返回一个友好的、面向用户的系统或用户当前状况的详细报告。
\end{quote}

\end{Exercise}


\cleardoublepage
\section{答案}
\shipoutAnswer
