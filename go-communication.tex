\epi{``Good communication is as stimulating as black coffee, and just as hard
to sleep after.''}{\textsc{ANNE MORROW LINDBERGH}}
\noindent{}In this chapter we are going to look at the building blocks in Go for 
communicating with the outside world. We will look at files, directories, networking
and executing other programs. Central to Go's I/O are the interfaces \type{io.Reader}
and \type{io.Writer}.

Reading from (and writing to) files is easy in Go. This program
only uses the \package{os} package to read data from the file \file{/etc/passwd}.
\lstinputlisting[caption=Reading from a file (unbuffered),label=src:read]{src/file.go}
The following is happening here:
\showremarks
If you want to use \first{buffered}{buffered} IO there is the
\package{bufio}\index{package!bufio} package:
\lstinputlisting[caption=Reading from a file (bufferd),label=src:bufread]{src/buffile.go}
\showremarks

\section{io.Reader}
As mentioned above the \first{io.Reader}{io.Reader} is an important interface in the language Go. A lot
(if not all) functions that need to read from something take an \type{io.Reader}\index{package!io}
as input. To fulfill the interface a type needs to implement only one method: \func{Read(p []byte) (n
int, err error)}. The writing side is (you may have guessed) an \type{io.Writer}, which has
the \func{Write} method.

If you think of a new type in your program or package and you make it fulfill the \type{io.Reader}
or \type{io.Writer} interface, \emph{the whole standard Go library can be used} on that type!

\section{Some examples}
The previous program reads a file in its entirety, but a more common scenario is that
you want to read a file on a line-by-line basis. The following snippet shows a way
to do just that:

\begin{lstlisting}
f, _ := os.Open("/etc/passwd"); defer f.Close()
r := bufio.NewReader(f) |\coderemark{Make it a bufio to access the ReadString method}|
s, ok := r.ReadString('\n') |\coderemark{Read a line from the input}|
// ... \coderemark{\var{s} holds the string, with the \package{strings} package you can parse it}
\end{lstlisting}

A more robust method (but slightly more complicated) is \func{ReadLine}, see the documentation
of the \package{bufio} package.

A common scenario in shell scripting is that you want to check if a directory
exists and if not, create one. 

\begin{minipage}{.5\textwidth}
\begin{lstlisting}[language=sh,caption={Create a directory with the shell}]
if [ ! -e name ]; then
    mkdir name
else
    # error
fi
\end{lstlisting}
\end{minipage}
\hspace{1em}
\begin{minipage}{.5\textwidth}
\begin{lstlisting}[caption={Create a directory with Go}]
if f, e := os.Stat("name"); e != nil {
    os.Mkdir("name", 0755)
} else {
    // error
}
\end{lstlisting}
\end{minipage}
The similarity between these two examples have prompted comments that Go has a
``script''-like feel to it, i.e. programming in Go can be compared to programming in 
a interpreted language (Python, Ruby, Perl or PHP).

\section{Command line arguments}
\label{sec:option parsing}
Arguments from the command line are available inside your program via
the string slice \var{os.Args}, provided you have imported the package
\package{os}. The \package{flag} package has a more sophisticated
interface, and also provides a way to parse flags. Take this example
from a DNS query tool:
\begin{lstlisting}
dnssec := flag.Bool("dnssec", false, "Request DNSSEC records") |\longremark{Define a \texttt{bool} flag, %%
\texttt{-dnssec}. The variable must be a pointer otherwise the package can not set its value;}|
port := flag.String("port", "53", "Set the query port")      |\longremark{Idem, but for a \texttt{port} option;}|
flag.Usage = func() {   |\longremark{Slightly redefine the \func{Usage} function, to be a little more verbose;}|
    fmt.Fprintf(os.Stderr, "Usage: %s [OPTIONS] [name ...]\n", os.Args[0])
    flag.PrintDefaults() |\longremark{For every flag given, \func{PrintDefaults} will output the help string;}|
}
flag.Parse()   |\longremark{Parse the flags and fill the variables.}|
\end{lstlisting}
\showremarks
After the flags have been parsed you can used them:
\begin{lstlisting}
if *dnssec {    |\coderemark{Dereference the \var{dnssec} flag variable}|
    // do something
}
\end{lstlisting}

\section{Executing commands}
The \package{os/exec}\index{package!os/exec} package has functions to run external commands, and is the premier way to
execute commands from within a Go program. It works by defining a \var{*exec.Cmd} structure for which it
defines a number of methods.
Let's execute \verb|ls -l|:
\begin{lstlisting}
import "os/exec"

cmd := exec.Command("/bin/ls", "-l")    |\coderemark{Create a \var{*cmd}}|
err := cmd.Run()                        |\coderemark{\func{Run()} it}|
\end{lstlisting}
The above example just runs ``ls -l'' without doing anything with the returned data,
capturing the standard output from a command is done as follows:
\begin{lstlisting}
import "os/exec"

cmd := exec.Command("/bin/ls", "-l")
buf, err := cmd.Output()                 |\coderemark{\var{buf} is a (\type{[]byte})}|
\end{lstlisting}

\section{Networking}
All network related types and functions can be found in the package \package{net}. One of the
most important functions in there is \func{Dial}\index{networking!Dial}. When you \func{Dial}
into a remote system the function returns a \var{Conn} interface type, which can be used
to send and receive information. The function \func{Dial} neatly abstracts away the network
family and transport. So IPv4 or IPv6, TCP or UDP can all share a common interface. 

Dialing a remote system (port 80) over TCP, then UDP and lastly TCP over IPv6 looks
like this\footnote{In case
you are wondering, 192.0.32.10 and 2620:0:2d0:200::10 are \url{www.example.org}.}:
\begin{lstlisting}
conn, e := Dial("tcp", "192.0.32.10:80")
conn, e := Dial("udp", "192.0.32.10:80")
conn, e := Dial("tcp", "[2620:0:2d0:200::10]:80") |\coderemark{Mandatory brackets}|
\end{lstlisting}

If there were no errors (returned in \var{e}), you can use \var{conn} to read and write.
The primitives defined in the package \package{net} are:
\begin{quote}
// \func{Read} reads data from the connection.\\
\lstinline{Read(b []byte) (n int, err error)}
\end{quote}
This makes \var{conn} an \type{io.Reader}.

\begin{quote}
// \func{Write} writes data to the connection.\\
\lstinline{Write(b []byte) (n int, err error)}
\end{quote}
This makes \var{conn} also an \type{io.Writer}, in fact \var{conn} is an
\type{io.ReadWriter}.\footnote{The variable \var{conn} also implements a \func{close} method, this really makes
it an \type{io.ReadWriteCloser}.}

But these are the low level nooks and crannies\footnote{Exercise Q\ref{ex:echo} is about using
these.}, you will almost always use higher level packages.
Such as the \package{http} package. For instance a simple Get for http:
\begin{lstlisting}
package main
import ( "io/ioutil"; "http"; "fmt" ) |\longremark{The imports needed;}|

func main() {
        r, err := http.Get("http://www.google.com/robots.txt") |\longremark{Use http's \func{Get} to retrieve the html;}|
        if err != nil { fmt.Printf("%s\n", err.String()); return } |\longremark{Error handling;}|
        b, err := ioutil.ReadAll(r.Body)    |\longremark{Read the entire document into \var{b};}|
        r.Body.Close()  
        if err == nil { fmt.Printf("%s", string(b)) } |\longremark{If everything was OK, print the document.}|
}
\end{lstlisting}
\showremarks

\section{Exercises}
\begin{Exercise}[title={Processes},difficulty=8]
\label{ex:processes}
\Question\label{ex:processes q1}
Write a program that takes a list of all running processes and prints
how many child processes each parent has spawned. The output should
look like:
%% For some reason the spacing in Exercise env. does weird things
\vskip\baselineskip
\begin{display}
Pid 0 has 2 children: [1 2]
Pid 490 has 2 children: [1199 26524]
Pid 1824 has 1 child: [7293]
\end{display}
\vskip\baselineskip
\begin{itemize}
\item{For acquiring the process list, you'll need to capture the output
of \verb|ps -e -opid,ppid,comm|. This output looks like:
\vskip\baselineskip
\begin{display}
  PID  PPID COMMAND
 9024  9023 zsh
19560  9024 ps
\end{display}
\vskip\baselineskip}
\item{If a parent has one child you must print \verb|child|, is there are
more than one print \verb|children|;}
\item{The process list must be numerically sorted, so you start with 
pid 0 and work your way up.}
\end{itemize}
Here is a Perl version to help you on your way (or to create complete
and utter confusion).
\lstinputlisting[caption={Processes in Perl}]{ex-communication/src/proc.pl}
\end{Exercise}

\begin{Answer}
\Question There is lots of stuff to do here. We can divide our program
up in the following sections:
\begin{enumerate}
\item{Starting \verb|ps| and capturing the output;}
\item{Parsing the output and saving the child PIDs for each PPID;}
\item{Sorting the PPID list;}
\item{Printing the sorted list to the screen}
\end{enumerate}
In the solution presented below, we've opted to use
\package{container/vector} to hold the PIDs. This "list" grows
automatically.

The function \func{atoi} (lines 19 through 22) is defined to get rid of the multiple return
values of the original \func{strconv.Atoi}, so that it can be used
inside function calls that only accept one argument, as we do on lines
45, 47 and 50.

A possible program is: 
\lstinputlisting[caption=Processes in Go,numbers=right]{ex-communication/src/proc.go}
\end{Answer}


\begin{Exercise}[title={单词和字母统计},difficulty=5]
\label{ex:wc}
\Question\label{ex:wc q1}编写一个从标准输入中读取文本的小程序,
并进行下面的操作:
\begin{enumerate}
\item{计算字符数量(包括空格);}
\item{计算单词数量;}
\item{计算行数。}
\end{enumerate}
换句话说,实现一个 \prog{wc(1)}(参阅本地的手册页面),
然而只需要从标准输入读取。
\end{Exercise}

\begin{Answer}
\Question 下面是 \prog{wc(1)} 的一种实现。
\lstinputlisting[caption=wc(1) 的 Go 实现]{ex-communication/src/wc.go}
\showremarks
\end{Answer}


\begin{Exercise}[title={Uniq},difficulty=4]
\label{ex:Uniq}
\Question\label{ex:Uniq q1} Write a Go program that mimics the function
of the Unix \prog{uniq} command. This program should work as follows,
given a list with the following items: 

\begin{display}
'a' 'b' 'a' 'a' 'a' 'c' 'd' 'e' 'f' 'g'
\end{display}

it should print only those item which don't have the same successor:

\begin{display}
'a' 'b' 'a' 'c' 'd' 'e' 'f'
\end{display}
\exdisfix
Listing \ref{src:uniq} is a Perl implementation of the algorithm.
\lstinputlisting[label=src:uniq,caption=uniq(1) in Perl,language=Perl]{ex-communication/src/uniq.pl}

\end{Exercise}

\begin{Answer}
\Question The following is a uniq implementation in Go.
\lstinputlisting[caption=uniq(1) in Go]{ex-communication/src/uniq.go}
\end{Answer}


\begin{Exercise}[title={Quine},difficulty=2]
A \emph{Quine} is a program that prints itself.
\label{ex:quine}
\Question\label{ex:quine q1} Write a Quine in Go.
\end{Exercise}

\begin{Answer}
\begin{lbar}
This solution is from Russ Cox. It was posted to 
the Go Nuts mailing list.
\end{lbar}
\Question 
\begin{lstlisting}[caption=A Go quine]
/* Go quine */
package main
import "fmt"
func main() {
 fmt.Printf("%s%c%s%c\n", q, 0x60, q, 0x60)
}
var q = `/* Go quine */
package main
import "fmt"
func main() {
 fmt.Printf("%s%c%s%c\n", q, 0x60, q, 0x60)
}
var q = `
\end{lstlisting}
\end{Answer}


\begin{Exercise}[title={Echo server},difficulty=8]
\label{ex:echo}
\Question\label{ex:echo q1}
Write a simple echo server. Make it listen to TCP port number 8053 on localhost. It should
be able to read a line (up to the newline), echo back that line and then close the connection. 

\Question\label{ex:echo q2}
Make the server concurrent so that every request is taken care of in a separate
goroutine.

\end{Exercise}

\begin{Answer}
\Question
A simple echo server might be:
\lstinputlisting[caption=A simple echo server]{ex-communication/src/echo.go}

When started you should see the following:
\vskip\baselineskip
\begin{display}
\pr \user{nc 127.0.0.1 8053}
\user{Go is *awesome*}
Go is *awesome*
\end{display}

\Question
To make the connection handling concurrent we \emph{only need to change one line} in our
echo server, the line:
\begin{lstlisting}
if c, err := l.Accept(); err == nil { Echo(c) }
\end{lstlisting}
becomes:
\begin{lstlisting}
if c, err := l.Accept(); err == nil { go Echo(c) }
\end{lstlisting}
\end{Answer}


\begin{Exercise}[title={Number cruncher},difficulty=9]
\label{ex:numbercruncher}
\begin{itemize}
\item{Pick six (6) random numbers from this list:
$$1, 2, 3, 4, 5, 6, 7, 8, 9, 10, 25, 50, 75, 100$$
Numbers may be picked multiple times;}
\item{Pick one (1) random number ($i$) in the range: $1 \ldots 1000$;}
\item{Tell how, by combining the first 6 numbers (or a subset thereof)
with the operators $+$,$-$,$*$ and $/$, you can make $i$;}
\end{itemize}
An example. We have picked the numbers: 1, 6, 7, 8, 8 and 75. And $i$ is
977. This can be done in many different ways, one way is:
$$ ((((1 * 6) * 8) + 75) * 8) - 7 = 977$$ 
or
$$ (8*(75+(8*6)))-(7/1) = 977$$

\Question\label{ex:cruncher q1}
Implement a number cruncher that works like that. Make it print the
solution in a similar format (i.e. output should be infix with
parenthesis) as used above.
\Question\label{ex:cruncher q2}
Calculate \emph{all} possible solutions and show them (or only show how
many there are). In the example above there are 544 ways to do it.
\end{Exercise}

\begin{Answer}
\Question 
The following is one possibility. It uses recursion and backtracking to get
an answer.
\lstinputlisting[caption=Number cruncher]{ex-communication/src/permrec.go}

\Question
When starting \prog{permrec} we give 977 as the first argument:
\vspace{1em}
\begin{display}
\pr ./permrec 977
1+(((6+7)*75)+(8/8)) = 977  #1
...                         ...
((75+(8*6))*8)-7 = 977      #542
(((75+(8*6))*8)-7)*1 = 977  #543
(((75+(8*6))*8)-7)/1 = 977  #544
\end{display}

\end{Answer}


\begin{Exercise}[title={*Finger 守护进程},difficulty=8]
\label{ex:finger}
\Question
编写一个 finger 守护进程,可以工作于 finger(1) 命令。

来自 Debian 的包描述:
\begin{quote}
Fingerd 是一个基于 RFC 1196 \cite{RFC1196} 的简单的守护进程,它为许多站点提供了``finger''程序的接口。
这个程序支持返回一个友好的、面向用户的系统或用户当前状况的详细报告。
\end{quote}

\end{Exercise}


\cleardoublepage
\section{Answers}
\shipoutAnswer
