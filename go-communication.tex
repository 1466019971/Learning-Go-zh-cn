\epi{"Good communication is as stimulating as black coffee, and just as hard
to sleep after."}{\textsc{ANNE MORROW LINDBERGH}}
\noindent{}In this chapter we are going to look at the building blocks in Go for 
communicating with the outside world.

\section{Files}
Reading from (and writing to) files is easy in Go. This program
only uses the \package{os} package to read data from the file \file{/etc/passwd}.
\lstinputlisting[caption=Reading from a file (unbufferd),label=src:read]{src/file.go}
\showremarks
If you want to use \first{buffered}{buffered} IO there is the
\package{bufio}\index{package!bufio} package:
\lstinputlisting[caption=Reading from a file (bufferd),label=src:bufread]{src/buffile.go}
\showremarks

\subsection{Line by line}
The previous program reads a file in its entirely, but a more common scenario is that
you want to read a file on a line-by-line basis. The following snippet show a way
to do just that:

\begin{lstlisting}
f, _ := os.Open("/etc/passwd")
defer f.Close()
r := bufio.NewReader(f)
s, ok := r.ReadString('\n')     |\coderemark{Read a line from the input}|
// ... \coderemark{\var{s} holds the string, with the \package{strings} package you can parse it}
\end{lstlisting}

A more robust method (but slightly more complicated) is \func{ReadLine}, see the documentation
of the \package{bufio} package.

\section{Command line arguments}
\label{sec:option parsing}
Arguments from the command line are available inside your program via
the string slice \var{os.Args}, provided you have imported the package
\package{os}. The \package{flag} package has a more sophisticated
interface, and also provides a way to parse flags. Take this example
from a DNS query tool:
\begin{lstlisting}
dnssec := flag.Bool("dnssec", false, "Request DNSSEC records") |\longremark{Define a \texttt{bool} flag, %%
\texttt{-dnssec}. The variable must be a pointer otherwise the package can not set its value;}|
port := flag.String("port", "53", "Set the query port")      |\longremark{Idem, but for a \texttt{port} option;}|
flag.Usage = func() {   |\longremark{Slightly redefine the \func{Usage} function, to be a little more verbose;}|
    fmt.Fprintf(os.Stderr, "Usage: %s [OPTIONS] [name ...]\n", os.Args[0])
    flag.PrintDefaults() |\longremark{For every flag given, \func{PrintDefaults} will output the help string;}|
}
flag.Parse()   |\longremark{Parse the flags and fill the variables.}|
\end{lstlisting}
\showremarks

\section{Executing commands}
The \package{exec}\index{package!exec} package has functions to run external commands, and it the premier way to
execute commands from within a Go program. It works by defining a \var{*exec.Cmd} structure for which it
defines a number of methods.
Lets execute \verb|ls -l|:
\begin{lstlisting}
import "exec"

cmd := exec.Command("/bin/ls", "-l")    |\coderemark{Create a \var{*cmd}}|
err := cmd.Run()                        |\coderemark{\func{Run()} it}|
\end{lstlisting}
Capturing standard output from a command is also easy to do:
\begin{lstlisting}
import "exec"

cmd := exec.Command("/bin/ls", "-l")
buf, err := cmd.Ouput()                 |\coderemark{\var{buf} is a (\type{[]byte})}|
\end{lstlisting}

\section{Networking}
All network related types and functions can be found in the package \package{net}. One of the
most important functions in there is \func{Dial}\index{networking!Dial}. When you \func{Dial}
into a remote system the function returns a \var{Conn} interface type, which can be used
to send and receive information. The function \func{Dial} neatly abstracts away the network
family and transport. So IPv4 or IPv6, TCP or UDP can all share a common interface. 

Dialing a remote system (port 80) over TCP, then UDP and lastly TCP over IPv6 looks
like this:\footnote{In case
you are wondering, 192.0.32.10 and 2620:0:2d0:200::10 are \url{www.example.org}.}
\begin{lstlisting}
conn, _ := Dial("tcp", "192.0.32.10:80")
conn, _ := Dial("udp", "192.0.32.10:80")
conn, _ := Dial("tcp", "[2620:0:2d0:200::10]:80") |\coderemark{Mandatory brackets}|
\end{lstlisting}

If there were no errors, you can use \var{conn} to read and write from and to sockets.
The primitives defined in the package \package{net} are:
\begin{quote}
// \func{Read} reads data from the connection.\\
// \func{Read} can be made to time out and return a \var{net.Error} with \lstinline{Timeout() == true}\\
// after a fixed time limit; see \func{SetTimeout} and \func{SetReadTimeout}.\\
\lstinline{Read(b []byte) (n int, err os.Error)}
\end{quote}

\begin{quote}
// \func{Write} writes data to the connection.\\
// \func{Write} can be made to time out and return a \var{net.Error} with \lstinline{Timeout() == true}\\
// after a fixed time limit; see \func{SetTimeout} and \func{SetWriteTimeout}.\\
\lstinline{Write(b []byte) (n int, err os.Error)}
\end{quote}

But these are the lowlevel nooks and crannies\footnote{Exercise Q\ref{ex:echo} is about using
these.}, you will almost always use higher level packages.
Such as the \package{http} package. For instance a simple Get for http:
\begin{lstlisting}
package main
import (
        "io/ioutil"
        "http"
        "fmt"
)

func main() {
        r, err := http.Get("http://www.google.com/robots.txt")
        if err != nil {
                fmt.Printf("%s\n", err.String())
                return
        }   
        b, err := ioutil.ReadAll(r.Body)
        r.Body.Close()

        if err == nil {
                fmt.Printf("%s", string(b))                                                                             
        }   
}
\end{lstlisting}

\section{Netchan: networking and channels}
%%http://blog.golang.org/2010/09/go-concurrency-patterns-timing-out-and.html

\section{Exercises}
\begin{Exercise}[title={Processes},difficulty=8]
\label{ex:processes}
\Question\label{ex:processes q1}
Write a program that takes a list of all running processes and prints
how many child processes each parent has spawned. The output should
look like:
%% For some reason the spacing in Exercise env. does weird things
\vskip\baselineskip
\begin{display}
Pid 0 has 2 children: [1 2]
Pid 490 has 2 children: [1199 26524]
Pid 1824 has 1 child: [7293]
\end{display}
\vskip\baselineskip
\begin{itemize}
\item{For acquiring the process list, you'll need to capture the output
of \verb|ps -e -opid,ppid,comm|. This output looks like:
\vskip\baselineskip
\begin{display}
  PID  PPID COMMAND
 9024  9023 zsh
19560  9024 ps
\end{display}
\vskip\baselineskip}
\item{If a parent has one child you must print \verb|child|, is there are
more than one print \verb|children|;}
\item{The process list must be numerically sorted, so you start with 
pid 0 and work your way up.}
\end{itemize}
Here is a Perl version to help you on your way (or to create complete
and utter confusion).
\lstinputlisting[caption={Processes in Perl}]{ex-communication/src/proc.pl}
\end{Exercise}

\begin{Answer}
\Question There is lots of stuff to do here. We can divide our program
up in the following sections:
\begin{enumerate}
\item{Starting \verb|ps| and capturing the output;}
\item{Parsing the output and saving the child PIDs for each PPID;}
\item{Sorting the PPID list;}
\item{Printing the sorted list to the screen}
\end{enumerate}
In the solution presented below, we've opted to use
\package{container/vector} to hold the PIDs. This "list" grows
automatically.

The function \func{atoi} (lines 19 through 22) is defined to get rid of the multiple return
values of the original \func{strconv.Atoi}, so that it can be used
inside function calls that only accept one argument, as we do on lines
45, 47 and 50.

A possible program is: 
\lstinputlisting[caption=Processes in Go,numbers=right]{ex-communication/src/proc.go}
\end{Answer}


\begin{Exercise}[title={单词和字母统计},difficulty=5]
\label{ex:wc}
\Question\label{ex:wc q1}编写一个从标准输入中读取文本的小程序,
并进行下面的操作:
\begin{enumerate}
\item{计算字符数量(包括空格);}
\item{计算单词数量;}
\item{计算行数。}
\end{enumerate}
换句话说,实现一个 \prog{wc(1)}(参阅本地的手册页面),
然而只需要从标准输入读取。
\end{Exercise}

\begin{Answer}
\Question 下面是 \prog{wc(1)} 的一种实现。
\lstinputlisting[caption=wc(1) 的 Go 实现]{ex-communication/src/wc.go}
\showremarks
\end{Answer}


\begin{Exercise}[title={Uniq},difficulty=4]
\label{ex:Uniq}
\Question\label{ex:Uniq q1} Write a Go program that mimics the function
of the Unix \prog{uniq} command. This program should work as follows,
given a list with the following items: 

\begin{display}
'a' 'b' 'a' 'a' 'a' 'c' 'd' 'e' 'f' 'g'
\end{display}

it should print only those item which don't have the same successor:

\begin{display}
'a' 'b' 'a' 'c' 'd' 'e' 'f'
\end{display}
\exdisfix
Listing \ref{src:uniq} is a Perl implementation of the algorithm.
\lstinputlisting[label=src:uniq,caption=uniq(1) in Perl,language=Perl]{ex-communication/src/uniq.pl}

\end{Exercise}

\begin{Answer}
\Question The following is a uniq implementation in Go.
\lstinputlisting[caption=uniq(1) in Go]{ex-communication/src/uniq.go}
\end{Answer}


\begin{Exercise}[title={Quine},difficulty=2]
A \emph{Quine} is a program that prints itself.
\label{ex:quine}
\Question\label{ex:quine q1} Write a Quine in Go.
\end{Exercise}

\begin{Answer}
\begin{lbar}
This solution is from Russ Cox. It was posted to 
the Go Nuts mailing list.
\end{lbar}
\Question 
\begin{lstlisting}[caption=A Go quine]
/* Go quine */
package main
import "fmt"
func main() {
 fmt.Printf("%s%c%s%c\n", q, 0x60, q, 0x60)
}
var q = `/* Go quine */
package main
import "fmt"
func main() {
 fmt.Printf("%s%c%s%c\n", q, 0x60, q, 0x60)
}
var q = `
\end{lstlisting}
\end{Answer}


\begin{Exercise}[title={Echo server},difficulty=8]
\label{ex:echo}
\Question\label{ex:echo q1}
Write a simple echo server. Make it listen to TCP port number 8053 on localhost. It should
be able to read a line (up to the newline), echo back that line and then close the connection. 

\Question\label{ex:echo q2}
Make the server concurrent so that every request is taken care of in a separate
goroutine.

\end{Exercise}

\begin{Answer}
\Question
A simple echo server might be:
\lstinputlisting[caption=A simple echo server]{ex-communication/src/echo.go}

When started you should see the following:
\vskip\baselineskip
\begin{display}
\pr \user{nc 127.0.0.1 8053}
\user{Go is *awesome*}
Go is *awesome*
\end{display}

\Question
To make the connection handling concurrent we \emph{only need to change one line} in our
echo server, the line:
\begin{lstlisting}
if c, err := l.Accept(); err == nil { Echo(c) }
\end{lstlisting}
becomes:
\begin{lstlisting}
if c, err := l.Accept(); err == nil { go Echo(c) }
\end{lstlisting}
\end{Answer}


\begin{Exercise}[title={Number cruncher},difficulty=9]
\label{ex:numbercruncher}
\begin{itemize}
\item{Pick six (6) random numbers from this list:
$$1, 2, 3, 4, 5, 6, 7, 8, 9, 10, 25, 50, 75, 100$$
Numbers may be picked multiple times;}
\item{Pick one (1) random number ($i$) in the range: $1 \ldots 1000$;}
\item{Tell how, by combining the first 6 numbers (or a subset thereof)
with the operators $+$,$-$,$*$ and $/$, you can make $i$;}
\end{itemize}
An example. We have picked the numbers: 1, 6, 7, 8, 8 and 75. And $i$ is
977. This can be done in many different ways, one way is:
$$ ((((1 * 6) * 8) + 75) * 8) - 7 = 977$$ 
or
$$ (8*(75+(8*6)))-(7/1) = 977$$

\Question\label{ex:cruncher q1}
Implement a number cruncher that works like that. Make it print the
solution in a similar format (i.e. output should be infix with
parenthesis) as used above.
\Question\label{ex:cruncher q2}
Calculate \emph{all} possible solutions and show them (or only show how
many there are). In the example above there are 544 ways to do it.
\end{Exercise}

\begin{Answer}
\Question 
The following is one possibility. It uses recursion and backtracking to get
an answer.
\lstinputlisting[caption=Number cruncher]{ex-communication/src/permrec.go}

\Question
When starting \prog{permrec} we give 977 as the first argument:
\vspace{1em}
\begin{display}
\pr ./permrec 977
1+(((6+7)*75)+(8/8)) = 977  #1
...                         ...
((75+(8*6))*8)-7 = 977      #542
(((75+(8*6))*8)-7)*1 = 977  #543
(((75+(8*6))*8)-7)/1 = 977  #544
\end{display}

\end{Answer}


\cleardoublepage
\section{Answers}
\shipoutAnswer
