\begin{Exercise}[title={Stack as package},difficulty=2]
\label{ex:stack-package}
\Question\label{ex:stack-package q1} 
See the Q\ref{ex:stack} exercise. In this exercise we want to create
a separate package for that code.
Create a proper package for your
stack implementation, \func{Push}, \func{Pop} and the \type{Stack} type need to be
exported.

\Question\label{ex:stack-package q2} Write a simple unit test for this package.
You should at least test that a \func{Pop} works after a \func{Push}.

\Question\label{ex:stack-package q3} Which official Go package could
also be used as a (FIFO) stack?
\end{Exercise}

\begin{Answer}
\Question There are a few details that should be changed to make a proper package
for our stack. First, the exported functions should begin with a capital 
letter and so should \type{Stack}. The full package becomes:
\lstinputlisting[caption=Stack in a package]{ex-packages/src/stack-as-package.go}

\Question To make the unit testing work properly you need to do some
preparations. I'll come to does in a minute. First the actual unit test.
Create a file with the "pushpop\_test.go", with the following contents:
\lstinputlisting[caption=Push/Pop test]{ex-packages/src/pushpop_test.go}
As \prog{gotest} calls \prog{gomake} to compile the package we need a makefile:
\lstinputlisting[language=make,caption=Package Makefile]{ex-packages/src/Makefile}
With this work done, we can call \prog{gotest} to test the pacakge:

\begin{display}
\pr \user{gotest}
rm -f _test/stack.a
6g  -o _gotest_.6 stack-as-package.go  pushpop_test.go
rm -f _test/stack.a
gopack grc _test/stack.a _gotest_.6 
PASS
\end{display}

\Question That would be the package \package{container/vector}.
\end{Answer}
