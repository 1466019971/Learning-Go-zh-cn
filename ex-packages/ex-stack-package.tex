\begin{Exercise}[title={stack 包},difficulty=2]
\label{ex:stack-package}
\Question\label{ex:stack-package q1} 
参考 Q\ref{ex:stack} 练习。在这个练习中将从那个代码中建立一个独立的包。
为 stack 的实现创建一个合适的包,\func{Push}、\func{Pop} 和 \type{Stack} 类型需要被导出。

\Question\label{ex:stack-package q2} 为这个包编写一个单元测试,
至少测试 \func{Push} 后 \func{Pop} 的工作情况。

\Question\label{ex:stack-package q3} 还有哪个官方提供的包可以做为(FIFO)stack 使用?
\end{Exercise}

\begin{Answer}
\Question 在创建 stack 包时,仅有一些小细节需要修改。
首先,导出的函数应当大写首字母,因此应该是 \type{Stack}。
完整的包变为:
\lstinputlisting[caption=包里的 Stack]{ex-packages/src/stack-as-package.go}

\Question 为了让单元测试正常工作,需要做一些准备。
下面用一分钟的时间来做这些。首先是单元测试本身。
创建文件“pushpop\_test.go”,有如下内容:
\lstinputlisting{ex-packages/src/pushpop_test.go}
因为 \prog{gotest} 调用 \prog{gomake} 编译包,所以需要一个 makefile:
\lstinputlisting[language=make]{ex-packages/src/Makefile}
完成了这些工作,可以调用 \prog{gotest} 来测试包:
\begin{display}
\pr \user{gotest}
rm -f _test/stack.a
6g  -o _gotest_.6 stack-as-package.go  pushpop_test.go
rm -f _test/stack.a
gopack grc _test/stack.a _gotest_.6 
PASS
\end{display}

\Question 应当是包 \package{container/vector}。

\end{Answer}
