\begin{Exercise}[title={Calculator},difficulty=7]
\label{ex:calc}
\Question\label{ex:calc q1} Create a reverse polish calculator. Use your
stack package.

\Question\label{ex:calc q2} Bonus. Rewrite your calculator to use the
package you found for question \ref{ex:stack-package q3} of previous
question.
\end{Exercise}

\begin{Answer}
\Question This is one answer:
\lstinputlisting[caption=A (rpn) calculator]{ex-packages/src/calc.go}
\Question The \package{container/vector} package would be a good candidate. It
even comes with \func{Push} and \func{Pop} functions \emph{predefined}.
The changes needed to our program are \emph{minimal} to say the least,
the following unified diff shows the differences:
\vskip\baselineskip
\begin{display}
--- calc.go     2010-05-16 10:19:13.886855818 +0200
+++ calcvec.go  2010-05-16 10:13:35.000000000 +0200
@@ -5,11 +5,11 @@
        "os"
        "strconv"
        "fmt"
-       "./stack"
+       "container/vector"
 )

 var reader *bufio.Reader = bufio.NewReader(os.Stdin)
-var st = new(Stack)
+var st = new(vector.IntVector)

 func main() \{
        for \{
\end{display}
\vskip\baselineskip
\noindent{}Only two lines need to be changed. \emph{Nice}.
\end{Answer}
