\begin{Exercise}[title={计算器},difficulty=7]
\label{ex:calc}
\Question\label{ex:calc q1} 使用 stack 包创建逆波兰计算器。

\Question\label{ex:calc q2} 扩展一下,用你在问题 \ref{ex:stack-package q2} 中发现的包重写计算器。
\end{Exercise}

\begin{Answer}
\Question 这是第一个答案
\lstinputlisting[caption=逆波兰计算器,basicstyle=\tiny\ttfamily]{ex-packages/src/calc.go}
\Question \package{container/vector} 包应当是不错的选择。它同样\emph{预定义}了
\func{Push} 和 \func{Pop} 函数。
对于我们的程序来说修改是非常\emph{小}的,下面的差异文件显示了不同的地方:
\vskip\baselineskip
\begin{display}
--- calc.go     2010-05-16 10:19:13.886855818 +0200
+++ calcvec.go  2010-05-16 10:13:35.000000000 +0200
@@ -5,11 +5,11 @@
        "os"
        "strconv"
        "fmt"
-       "./stack"
+       "container/vector"
 )

 var reader *bufio.Reader = bufio.NewReader(os.Stdin)
-var st = new(Stack)
+var st = new(vector.IntVector)

 func main() \{
        for \{
\end{display}
\vskip\baselineskip
\noindent{}只有两行需要修改。\emph{太棒了}。
\end{Answer}
