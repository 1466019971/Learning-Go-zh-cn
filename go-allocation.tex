\epi{\lstinline{fmt.Printf("\%p", i)}}{Printing the address of a pointer in Go.}
\noindent
You may have wished otherwise, but Go has pointers.
There is however now pointer arithmetic and they are still useful.
Remember Go when you call a function in Go the variables you pass are
pass-by-value. So, for efficiency and the possibility to modify a
passed value \emph{in} the function we have pointers.

%% Do we need a whole chapter on Pointers in Go
Just like in C you declare a pointer by prefixing the type with an `*`,
so:

are declared after variable names, and all type modifiers precede the
\draft{}%
types. So *X is a pointer to an X; [3]X is an array of three X's. The
types are therefore really easy to read just read out the names of the
type modifiers: [] declares something called an array slice; "*"
declares a pointer; [size] declares an array. So []*[3]*int is an array
slice of pointers to arrays of three pointers to ints

\noindent\lstinline{var pint *int   // declare pint to be pointer to int}

Note that it's perfectly OK to return the address of a local variable; the
storage associated with the variable survives after the function returns. In
fact, taking the address of a composite literal allocates a fresh instance each
time it is evaluated, so we can combine these last two lines. \cite{effective_go}

\section{Allocation}
Go has garbage collection, meaning that you don't have to worry about
memory allocation and deallocation. Of course almost every language
since 1980 has this, but it is nice to see garbage collection in a
C-like language. The following sections show how to handle allocation
in Go. There is somewhat an artifical distinction between
\first{\func{new()}} and \first{\func{make()}}. Details follow.

\section{Allocation with \func{new()}}
Go has two allocation primitives, \func{new()} and \func{make()}. They do different
things and apply to different types, which can be confusing, but the
rules are simple. Let's talk about \func{new()} first. It's a built-in function
essentially the same as its namesakes in other languages: \func{new(T)}
allocates zeroed storage for a new item of type \type{T} and returns its
address, a value of type \type{*T}. In Go terminology, it returns a pointer to
a newly allocated zero value of type \type{T}.

Since the memory returned by \func{new()} is zeroed, it's helpful to arrange
that the zeroed object can be used without further initialization. This
means a user of the data structure can create one with \func{new()} and get
right to work. For example, the documentation for \type{bytes.Buffer} states
that "the zero value for Buffer is an empty buffer ready to use."
Similarly, \func{sync.Mutex} does not have an explicit constructor or Init
method. Instead, the zero value for a \func{sync.Mutex} is defined to be an
unlocked mutex.

The zero-value-is-useful property works transitively. Consider this type
declaration.

\begin{lstlisting}
type SyncedBuffer struct {
    lock    sync.Mutex
    buffer  bytes.Buffer
}
\end{lstlisting}
Values of type \type{SyncedBuffer} are also ready to use immediately upon
allocation or just declaration. In this snippet, both \var{p} and
\var{v} will work
correctly without further arrangement.
\begin{lstlisting}
p := new(SyncedBuffer)  // type *SyncedBuffer
var v SyncedBuffer      // type  SyncedBuffer
\end{lstlisting}

\section{Constructors and composite literals}
Sometimes the zero value isn't good enough and an initializing
constructor is necessary, as in this example derived from package
\package{os}.
\begin{lstlisting}
func NewFile(fd int, name string) *File {
    if fd < 0 {
        return nil
    }
    f := new(File)
    f.fd = fd
    f.name = name
    f.dirinfo = nil
    f.nepipe = 0
    return f
}
\end{lstlisting}
There's a lot of boiler plate in there. We can simplify it using a
composite literal, which is an expression that creates a new instance
each time it is evaluated.

\begin{lstlisting}
func NewFile(fd int, name string) *File {
    if fd < 0 {
        return nil
    }
    f := File{fd, name, nil, 0}
    return &f
}
\end{lstlisting}
Note that it's perfectly OK to return the address of a local variable;
the storage associated with the variable survives after the function
returns. In fact, taking the address of a composite literal allocates a
fresh instance each time it is evaluated, so we can combine these last
two lines.

\begin{lstlisting}
return &File{fd, name, nil, 0}
\end{lstlisting}
The fields of a composite literal are laid out in order and must all be
present. However, by labeling the elements explicitly as field:value
pbairs, the initializers can appear in any order, with the missing ones
left as their respective zero values. Thus we could say

\begin{lstlisting}
return &File{fd: fd, name: name}
\end{lstlisting}
As a limiting case, if a composite literal contains no fields at all, it
creates a zero value for the type. The expressions
\lstinline{new(File)} and 
\lstinline|&File{]| are equivalent.

Composite literals can also be created for arrays, slices, and maps,
with the field labels being indices or map keys as appropriate. In these
examples, the initializations work regardless of the values of Enone,
Eio, and Einval, as long as they are distinct.
\begin{lstlisting}
a := [...]string   {Enone: "no error", Eio: "Eio", Einval: "invalid argument"}
s := []string      {Enone: "no error", Eio: "Eio", Einval: "invalid argument"}
m := map[int]string{Enone: "no error", Eio: "Eio", Einval: "invalid argument"}
\end{lstlisting}

\section{Allocation with \func{make()}}
Back to allocation. The built-in function \func{make(T, args)} serves a purpose
different from \func{new(T)}. It creates slices, maps, and channels only, and
it returns an initialized (not zero) value of type T, not *T. The reason
for the distinction is that these three types are, under the covers,
references to data structures that must be initialized before use. A
slice, for example, is a three-item descriptor containing a pointer to
the data (inside an array), the length, and the capacity; until those
items are initialized, the slice is nil. For slices, maps, and channels,
make initializes the internal data structure and prepares the value for
use. For instance,
\lstinline{make([]int, 10, 100)}
allocates an array of 100 ints and then creates a slice structure with
length 10 and a capacity of 100 pointing at the first 10 elements of the
array. (When making a slice, the capacity can be omitted; see the
section on slices for more information.) In contrast, new([]int) returns
a pointer to a newly allocated, zeroed slice structure, that is, a
pointer to a nil slice value.

These examples illustrate the difference between new() and make().
\begin{lstlisting}
var p *[]int = new([]int)       // allocates slice structure; *p == nil; rarely useful
var v  []int = make([]int, 100) // v now refers to a new array of 100 ints

// Unnecessarily complex:
var p *[]int = new([]int)
*p = make([]int, 100, 100)

// Idiomatic:
v := make([]int, 100)
\end{lstlisting}
Remember that make() applies only to maps, slices and channels and does
not return a pointer. To obtain an explicit pointer allocate with new().

\section{Exercises}
\begin{Exercise}[title={Pointers},difficulty=6]
\label{ex:pointers}
\Question
What is the difference between the following two allocations?
\begin{lstlisting}[numbers=none]
func Set(t *T) {
    x = t
}
\end{lstlisting}
and
\begin{lstlisting}[numbers=none]
func Set(t T) {
    x= &t
}
\end{lstlisting}

\end{Exercise}

\begin{Answer}
\Question
In the second function, \var{x} points to a new
(heap-allocated) variable \var{t} which contains
a copy of whatever the actual argument value is.

In the first function, \var{x} points to the same thing
that \var{t} does, which is the same thing that the actual
argument points to.

So in the second function, we have an "extra" variable
containing a copy of the interesting value.
\end{Answer}


\cleardoublepage
\section{Answers}
\shipoutAnswer
