\epi{\lstinline{^}}{Ken Thompson's answer to whether there is a bitwise negation
operator.}\noindent

When Go was released as open source software in December 2009 it came
with a boatload of packages. In this chapter we show how to use some
of these.


\section{Documenting packages}





\section{Testing packages}
In Go it is customary to write (unit) tests for your package. Writing
tests involves the \package{testing}.
\begin{lbar}
Read the package doc!
\end{lbar}

The testing itself is carried out with \prog{gotest}.
The \prog{gotest} program run all the test functions, each
test function has the same signature:
\begin{lstlisting}
func TestXxx(t *testing.T) |\coderemark{Test<Capital>restOftheName}|
\end{lstlisting}

When writing test you will need to tell \prog{gotest} that a test has failed or was successful. A
successful test function just \func{return}s. When test fails you can signal this with the following
functions\cite{go_doc}. These are the most important ones:

\begin{lstlisting}
func (t *T) Fail()
\end{lstlisting}
Fail marks the Test function as having failed but continues execution.

\begin{lstlisting}
func (t *T) FailNow()
\end{lstlisting}
FailNow marks the Test function as having failed and stops its execution.
Execution will continue at the next Test.

\begin{lstlisting}
func (t *T) Errorf(format string, args ...interface\{\})
\end{lstlisting}
Errorf is equivalent to Logf() followed by Fail().

\begin{lstlisting}
func (t *T) Logf(format string, args ...interface\{\})
\end{lstlisting}
Log formats its arguments according to the format, analogous to Printf(),
and records the text in the error log.

\begin{lstlisting}
func (t *T) Fatal(args ...interface\{\})
\end{lstlisting}
Fatal is equivalent to Log() followed by FailNow().




\begin{lstlisting}
func (t *T) Failed() bool
\end{lstlisting}
Failed returns whether the Test function has failed.




\section{Useful packages}



\subsection{fmt}

\subsection{flag}
argument parsing

\subsection{unsafe}


\section{Exercises}

\cleardoublepage
\section{Answers}
\shipoutAnswer
