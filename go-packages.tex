\epi{\lstinline{^}}{\textit{对是否有按位非的运算符的回答。}\\\textsc{KEN THOMPSON}}
\noindent{}
包是函数和数据的集合,跟 Perl 的包\cite{perl-packages}差不多。
用 \first{\key{package}}{keyword!package} 保留字定义一个包。
文件名不需要与包名一致。包名的约定是使用小写字符。
Go 包可以由多个文件组成,但是使用相同的 \lstinline{package <name>} 这一行。
让我们在文件 \prog{even.go} 中定义一个叫做 \package{even}\index{package!even} 的包。

\lstinputlisting[label=src:even,caption=A small package]{src/even.go}
名称以大写字幕起始的是\emph{可导出}的,可以在包的外部调用,稍候会讨论这个。
现在可以在下面的程序 \prog{myeven.go} 中使用这个包了:

\lstinputlisting[label=src:myeven,caption=Use of the even package]{src/myeven.go}
\showremarks

现在只需要编译和链接,首先是那个包,然后是 \prog{myeven.go},并且链接它们:
\begin{display}
\pr 6g even.go			\coderemark{包}
\pr 6g myeven.go		\coderemark{程序}
\pr 6l -o myeven myeven.6       \coderemark{链接步骤}
\end{display}
然后测试一下:
\begin{display}
\pr ./myeven
Is 5 even? false
\end{display}

在 Go 中,当函数的首字母大写的时候,函数会被从包中导出(在包外部可见,或者说公有的),
因此函数名是 \func{\emph{E}ven}。如果修改 \prog{myeven.go} 的第 7 行,使用未导出的函数
\func{even.odd}:

\noindent\lstinline{fmt.Printf("Is %d even? %v\n", i, even.odd(i))}

由于使用了\emph{私有}的函数,会得到一个编译错误:
\begin{display}
myeven.go:7: cannot refer to unexported name even.odd
myeven.go:7: undefined: even.odd
\end{display}

\noindent{}概括来说:
\begin{itemize}
\item 公有函数的名字以\emph{大写}字母开头;
\index{public}
\item 私有函数的名字以\emph{小写}字幕开头。
\index{private}
\end{itemize}
这个规则同样适用于定义在包中的其他名字(新类型、全局变量)。
注意,“大写”的含义并不仅限于 US ASCII,它被扩展到了整个 Unicode 范围。
所以大写的希腊语、古埃及语都是可以的。

\section{构建一个包}
\label{sec:building a package}
为了创建一个别人可以用的包(只需要 \lstinline{import "even"} 即可使用),首先需要创建一个目录放入包文件。
\begin{display}
\pr mkdir even
\pr cp even.go even/
\end{display}
\noindent{}然后可以使用根据 \package{even} 包改写的 \file{Makefile} 文件。
\lstinputlisting[language=make,caption=用于包的 Makefile,linerange={5,},numbers=right]{src/Makefile}
注意第 3 行定义了 \package{even} 包,而第 4 和第 5 行录入了构成包的文件。
同样注意,这\emph{不是}在第 \ref{chap:basics} 章,第 "\titleref{sec:building a program}"
节中使用的\emph{那个} \prog{Makefile}。最后一行的 \verb|include| 语句是不同的。

如果现在执行 \prog{gomake},一个叫做 "\file{\_go\_.6}" 的文件、一个叫做
"\dir{\_obj/}" 的目录,和 "\dir{\_obj/}" 目录中,叫做 "\file{even.a}" 的文件被创建。
文件 \file{even.a} 是有编译后的 Go 代码的静态库。
通过 \prog{gomake install} 包(好吧,只有 \file{even.a})被安装在\emph{官方}包目录中:
\begin{display}
\pr make install
cp \_obj/even.a \$GOROOT/pkg/linux\_amd64/even.a
\end{display}
\noindent{}安装完后,就可以修改 \key{import} 从
\lstinline{"./even"} 到 \lstinline{"even"}。

但是,如果不希望将包安装到官方的 Go 目录,或者没有权限这么做的话。
在使用 \prog{6/8g} 的时候,可以指定 \flag{\-I} 标志替换包目录。
有了这个标志,你可以让你的导入语句变成(\lstinline{import "even"})却继续部署在自定义的目录中。
所以,下面的命令可以联合包构建(并且链接)\prog{myeven}程序。
\begin{display}
\pr 6g -I even/\_obj myeven.go	\coderemark{Building}
\pr 6l -L even/\_obj myeven.6	\coderemark{Linking}
\end{display}

\noindent{}导入但是没有使用的包会引起一个错误。

\section{标识符}
像在其他语言中一样,Go 的命名是很重要的。在某些情况下,它们甚至有语义上的作用:
例如,在包外是否可见决定于首字母是不是大写。因此有必要花点时间讨论一下 Go 程序的命名规则。

使用的规则是让众所周知的缩写保持原样,而不是去尝试到底哪里应该大写。
\lstinline{Atoi},\lstinline{Getwd},\lstinline{Chmod}。

驼峰式对那些有完整单词的会很好:\lstinline{ReadFile,NewWriter,MakeSlice}。

\subsection{包名}
当包导入(通过 \first{\key{import}}{keyword!import})时,包名成为了内容的入口。
在\index{package!bytes} 
\begin{lstlisting}
import "bytes"
\end{lstlisting}
之后,导入包的可以调用函数\func{bytes.Buffer}。任何使用这个包的人,
可以使用同样的名字访问到它的内容,因此这样的包名是好的:短的、简洁的、好记的。
根据规则,包名是小写的一个单词;不应当有下划线或混合大小写。
由于每个人都可能需要录入这个名字,所以尽可能的简短。不要提前考虑冲突。
包名是导入的默认名称。就上面的 \lstinline{bytes.Buffer} 来说。通过 
\begin{lstlisting}
import bar "bytes"
\end{lstlisting}
它变成了 \lstinline{bar.Buffer}。
因此,它无须在整个代码中唯一,在少有的冲突中,可以给导入的包选择另一个名字在局部使用。
在任何时候,冲突都是很少见的,因为导入的文件名会用来做判断,到底是哪个包使用了。

另一个规则是包名就是代码的根目录名;在 \package{src/pkg/container/vector} 的包,
作为 \var{container/vector} 导入,但名字是 \package{vector},
不是 \package{container\_vector} 也不是 \package{containerVector}。\index{package!container/vector}

导入包将使用其名字引用到内容上,所以导入的包可以利用这个避免罗嗦。
例如,缓冲类型 \package{bufio}\index{package!bufio} 包的读取方法,叫做 \func{Reader},而不是
\func{BufReader},因为用户看到的是 \func{bufio.Reader} 这个清晰、简洁的名字。
更进一步说,由于导入的实例总是它们包名指向的地址,\func{bufio.Reader} 不会与
\func{io.Reader} 冲突。相似的,\func{ring.Ring} 创建新实例的函数——在 Go 中定义的构造函数——通常叫做
\func{NewRing},但是由于 \type{Ring} 是这个包唯一的一个导出的类型,同时,这个包也叫做
\package{ring}\index{package!ring},所以它可以只称作 \func{New}。
包的客户看到的是 \func{ring.New}。用包的结构帮助你选择更好的名字。

另外一个简短的例子是 \func{once.Do};\func{once.Do(setup)} 读起来很不错,并且
命名为 \lstinline{once.DoOrWaitUntilDone(setup)} 不会有任何帮助。长的名字不会让其变得容易阅读。
如果名字表达了一些复杂并且微妙的内容,更好的办法是编写一些有帮助的注释,而不是将所有信息都放入名字里。

最后,在 Go 中使用\first{混合大小写}{MixedCaps} MixedCaps 或者 mixedCaps,
而不是下划线区分含有多个单词的名字。

%% Advanced Go, leave it
%%\section{Initialization}
%%Every source file in a package can define an \func{init()} function. This function is
%%called after the variables in the package have gotten their value. The
%%\func{init()} function can be used to setup state before the execution
%%begins.

\section{Documenting packages}
\gomarginpar{This text is copied from \cite{effective_go}.}
Every package should have a \emph{package comment}, a block comment preceding the
\key{package} clause. For multi-file packages, the package comment only needs to be
present in one file, and any one will do. The package comment should introduce
the package and provide information relevant to the package as a whole. It will
appear first on the \prog{godoc} page and should set up the detailed documentation
that follows. An example from the official \package{regexp} package:
\begin{display}
/*
    The regexp package implements a simple library for
    regular expressions.

    The syntax of the regular expressions accepted is:

    regexp:
        concatenation { '|' concatenation }
*/
package regexp
\end{display}

Each defined (and exported) function should have a small line of text
documenting the behavior of the function, from the \package{fmt}
package:
\begin{display}
// Printf formats according to a format specifier and writes to standard output.
func Printf(format string, a ...interface{}) (n int, errno os.Error) \{
\end{display}

\section{Testing packages}
In Go it is customary to write (unit) tests for your package. Writing
tests involves the \package{testing} package and the program
\first{\prog{gotest}}{gotest}. Both
have excellent documentation. When you include tests with your package
keep in mind that has to be build using the standard \file{Makefile}
(see section "\titleref{sec:building a package}").


%% CLEANFILES+=hello fib chain run.out
%%

The testing itself is carried out with \prog{gotest}.
The \prog{gotest} program run all the test functions. Without any
defined tests for our \package{even} package a \prog{gomake test}, yields:
\begin{display}
\pr gomake test
no test files found (*_test.go)
make: *** [test] Error 2
\end{display}
Let us fix this by defining a test in a test file. Test files reside
in the package directory and are named \file{*\_test.go}. Those test
files are just like other Go program, but \prog{gotest} will only
execute the test functions.
Each test function has the same signature and its name should start
with \lstinline{Test}:
\begin{lstlisting}
func TestXxx(t *testing.T) |\coderemark{Test<Capital>restOftheName}|
\end{lstlisting}

When writing test you will need to tell \prog{gotest} that a test has
failed or was successful. A successful test function just returns. When
the test fails you can signal this with the following
functions \cite{go_doc}. These are the most important ones (see \prog{godoc testing}
for more):

\begin{lstlisting}[numbers=none]
func (t *T) Fail()
\end{lstlisting}
\func{Fail} marks the test function as having failed but continues execution.

\begin{lstlisting}[numbers=none]
func (t *T) FailNow()
\end{lstlisting}
\func{FailNow} marks the test function as having failed and stops its execution.
Execution will continue at the next test. So any other test in
\emph{this} file are skipped too.

\begin{lstlisting}[numbers=none]
func (t *T) Log(args ...interface{})
\end{lstlisting}
\func{Log} formats its arguments using default formatting, analogous to
\func{Print()}, and records the text in the error log.

\begin{lstlisting}[numbers=none]
func (t *T) Fatal(args ...interface{})
\end{lstlisting}
\func{Fatal} is equivalent to \func{Log()} followed by \func{FailNow()}.

Putting all this together we can write our test. First
we pick a name: \file{even\_test.go}. Then we add the following contents:
\lstinputlisting[label=src:eventest,caption=Test file for even
package,numbers=right]{src/even_test.go}
Note that we use \lstinline{package even} on line 1, the tests fall in the same
namespace as the package we are testing. This not only convenient, but
also allows tests of unexported function and structures. We then import
the \package{testing} package and on line 5 we define the only test
function in this file. The displayed Go code should not hold any
surprises: we check if the \func{Even} function works OK. 
Now, the moment we've been waiting for, executing the test:
\begin{display}
\pr gomake test
6g -o \_gotest\_.6 even.go  even\_test.go
rm -f \_test/even.a
gopack grc \_test/even.a \_gotest\_.6 
PASS
\end{display}
\noindent{}Our test ran and reported \texttt{PASS}. Success! To show how a failed
test look we redefine our test function:
\begin{lstlisting}
// Entering the twilight zone
func TestEven(t *testing.T) {
        if ! Even(2) {
                t.Log("2 should be odd!")
                t.Fail()
        }   
}
\end{lstlisting}
We now get:
\begin{display}
--- FAIL: even.TestEven
\qquad\qquad2 should be odd!
FAIL
make: *** [test] Error 1
\end{display}
\noindent{}And you can act accordingly (by fixing the test for instance).

\begin{lbar}
Writing new packages should go hand in hand with writing (some)
documentation and test functions. It will make your code better and it
shows that your really put in the effort.
\end{lbar}

\section{Useful packages}
The standard Go repository includes a huge number of packages and it is
even possible to install more along side your current Go installation. 
It is very enlightening to browse the \file{\$GOROOT/src/pkg} directory and
look at the packages.
We cannot comment on each package, but the following a worth a mention:
\footnote{The descriptions are copied from the packages' \prog{godoc}. Extra
remarks are type set in italic.}

\begin{description}
\item[\package{fmt}]{\index{package!fmt}
Package \package{fmt} implements formatted I/O with functions analogous
to C's \func{printf} and \func{scanf}. The format verbs are derived
from C's but are simpler. Some verbs (\%-sequences) that can be used:

\begin{description}
\item[\%v]{The value in a default format.
when printing structs, the plus flag (\%+v) adds field names;}
\item[\%\#v]{a Go-syntax representation of the value.}
\item[\%T]{a Go-syntax representation of the type of the value;}
\end{description}

}

\item[\package{io}]{\index{package!io}
This package provides basic interfaces to I/O primitives.
Its primary job is to wrap existing implementations of such primitives,
such as those in package os, into shared public interfaces that
abstract the functionality, plus some other related primitives.
}
\item[\package{bufio}]{\index{package!bufio}
This package implements buffered I/O.  It wraps an 
\lstinline{io.Reader}
or
\lstinline{io.Writer}
object, creating another object (Reader or Writer) that also implements
the interface but provides buffering and some help for textual I/O.
}
\item[\package{sort}]{\index{package!sort}
The \package{sort} package provides primitives for sorting arrays
and user-defined collections.
}
\item[\package{strconv}]{\index{package!strconv}
The \package{strconv} package implements conversions to and from
string representations of basic data types.
}
\item[\package{os}]{\index{package!os}
The \package{os} package provides a platform-independent interface to operating
system functionality.  The design is Unix-like.
}
\item[\package{flag}]{\index{package!flag}
The \package{flag} package implements command-line flag parsing. 
\emph{See "\titleref{sec:option parsing}"
on page \pageref{sec:option parsing}.}
}
\item[\package{json}]{\index{package!json}
The \package{json} package implements encoding and decoding of JSON objects as
defined in RFC 4627.
}
\item[\package{template}]{\index{package!template}
Data-driven templates for generating textual output such as HTML.

Templates are executed by applying them to a data structure.  Annotations in
the template refer to elements of the data structure (typically a field of a
struct or a key in a map) to control execution and derive values to be
displayed.  The template walks the structure as it executes and the "cursor" @
represents the value at the current location in the structure.
}
\item[\package{http}]{\index{package!http}
The \package{http} package implements parsing of HTTP requests, replies,
and URLs and provides an extensible HTTP server and a basic
HTTP client.
}
\item[\package{unsafe}]{\index{package!unsafe}
The \package{unsafe} package contains operations that step around the type safety of Go programs.
\emph{Normally you don't need this package.}
}
\item[\package{reflect}]{\index{package!reflect}
The \package{reflect} package implements run-time reflection, allowing a program to
manipulate objects with arbitrary types.  The typical use is to take a
value with static type \lstinline|interface{}| and extract its dynamic type
information by calling \func{Typeof}, which returns an object with interface
type \type{Type}.  That contains a pointer to a struct of type \type{*StructType},
\type{*IntType}, etc. representing the details of the underlying type.  A type
switch or type assertion can reveal which. \emph{See chapter \ref{chap:interfaces}, 
section "\titleref{sec:introspection}"}.
}
\item[\package{exec}]{\index{package!exec}
The \package{exec} package runs external commands.
}
\end{description}

\section{Exercises}
\todo{Write exercise for testing the stack package. (MG)}

\begin{Exercise}[title={stack 包},difficulty=0]
\label{ex:stack-package}
\Question\label{ex:stack-package q1} 
参考 Q\ref{ex:stack} 练习。在这个练习中将从那个代码中建立一个独立的包。
为 stack 的实现创建一个合适的包,\func{Push}、\func{Pop} 和 \type{Stack} 类型需要被导出。

\Question\label{ex:stack-package q2} 为这个包编写一个单元测试,
至少测试 \func{Push} 后 \func{Pop} 的工作情况。

\end{Exercise}

\begin{Answer}
\Question 在创建 stack 包时,仅有一些小细节需要修改。
首先,导出的函数应当大写首字母,因此应该是 \type{Stack}。
包所在的文件被命名为 \file{stack-as-package.go},内容是:
\lstinputlisting[caption=包里的 Stack]{ex-packages/src/stack-as-package.go}

\Question 为了让单元测试正常工作,需要做一些准备。
下面用一分钟的时间来做这些。首先是单元测试本身。
创建文件 \file{pushpop\_test.go},有如下内容:
\lstinputlisting[caption=Push/Pop 测试]{ex-packages/src/pushpop_test.go}
为了让 \prog{go test} 能够工作,需要将包所在文件放到 
\var{\$GOPATH/src}:\\

\begin{display}
\pr \user{mkdir $GOPATH/src/stack}
\pr \user{cp pushpop_test.go $GOPATH/src/stack}
\pr \user{cp stack-as-package.go $GOPATH/src/stack}
\end{display}

输出:\\

\begin{display}
\pr \user{go test stack}
ok      stack   0.001s
\end{display}
\end{Answer}


\begin{Exercise}[title={Calculator},difficulty=7]
\label{ex:calc}
\Question\label{ex:calc q1} Create a reverse polish calculator. Use your
stack package.

\end{Exercise}

\begin{Answer}
\Question This is one answer:
\lstinputlisting[caption=A (rpn) calculator]{ex-packages/src/calc.go}

\end{Answer}


\cleardoublepage
\section{Answers}
\shipoutAnswer
