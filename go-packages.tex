\epi{\lstinline{^}}{\textit{Answer to whether there is a bitwise negation
operator.}\\\textsc{Ken Thompson}}
\noindent{}Packages are a collection of functions and data, they are sort alike the
Perl packages\cite{perl-packages}. You declare a package with the
\lstinline{package} keyword. Note that (at least one of) the filename(s) should
match the package name. Go packages may consist out of multiple files,
but they share the \lstinline{package <name>} line.
Lets define our package \package{even} in the file \prog{even.go}.

\lstinputlisting[label=src:even,caption=A small package]{src/even.go}
Names that start with a capital letter are exported and may be used
outside your package, more on that later. We can now use
the package as follows in our own program \prog{myeven.go}:

\lstinputlisting[label=src:myeven,caption=Use of the \package{even} package]{src/myeven.go}
Now we just need to compile and link, first the package, then \prog{myeven.go} and
then link it:
\begin{display}
\pr 6g even.go			\qquad\qquad\rem{# package}
\pr 6g myeven.go		\qquad\qquad\rem{# our program}
\pr 6l -o myeven myeven.8
\end{display}
And now we can test it:
\begin{display}
\pr ./myeven
Is 5 even? false
\end{display}

In Go, a function from a package is exported (visible
outside the package, i.e. public) when the first letter of the function name is a capital, hence
the function name \func{\emph{E}ven}. If we change our \prog{myeven.go} on line
7 to using to unexported function \func{even.uneven}:

\noindent\lstinline{fmt.Printf("Is %d even? %v\n", i, even.uneven(i))}

We get an error when compiling, because we are trying to use a
\emph{private} function:
\begin{display}
myeven.go:7: cannot refer to unexported name even.uneven
myeven.go:7: undefined: even.uneven
\end{display}
To summarize:
\begin{itemize}
\item Public functions have a name \emph{starting} with a Capital
Letter;
\item Private function have a name starting with a lowercase letter.
\end{itemize}

\section{Building a package}
\label{sec:building a package}
The create a package other people can use (by just using
\lstinline{import "even"}) we first need to create a directory
where we put our package files in. 
\begin{display}
\pr mkdir even
\pr cp even.go even
\end{display}
\noindent{}Next we 
can use the following 
\file{Makefile}, which 
is adapted for our \package{even} package.
\lstinputlisting[language=make,caption=\file{Makefile} for a
package,numbers=right]{src/Makefile}
Note that on line 5 we define our \package{even} package and on the
lines 8 and 9 we enter the files which make up the package.
Also note that this is the \emph{same} \prog{Makefile} setup as we used
in section "\titleref{sec:building a progam}" in chapter
\ref{chap:basics}.

If we now issue \prog{make}, a file named \file{\_go\_.6}, a directory
named \dir{\_obj/} and a file inside \dir{\_obj/} called \file{even.a} 
is created. The file \file{even.a} is a static library which holds
the compiled Go code.
With \prog{make install} the package (well, only the \file{even.a}) is installed in the \emph{official}
package directory:
\begin{display}
\pr make install
cp \_obj/even.a \$GOROOT/pkg/linux\_amd64/even.a
\end{display}
\noindent{}After installing we can change our \key{import} from
\lstinline{"./even"} to \lstinline{"even"}.

But what if you do not want to install packages in the official Go
directory, or maybe you do not have the permission to do so? When using
\prog{6/8g} you can use the \flag{\-I}-flag to specify an alternate
package directory. With this flag you can leave your import statement
as-is (\lstinline{import "even"}) and continue development in your
own directory. So the following commands should build (and link) our
\prog{myeven} program with our package.
\begin{display}
\pr 6g -I even/\_obj myeven.go	# building
\pr 6l -L even/\_obj myeven.6	# linking
\end{display}

\section{Names}
Names are as important in Go as in any other language. In some cases
they even have semantic effect: for instance, the visibility of a name
outside a package is determined by whether its first character is upper
case. It's therefore worth spending a little time talking about naming
conventions in Go programs.

The convention we adopted was to leave well-known legacy
not-quite-words alone rather than try to figure out where
the capital letters go.  \lstinline{Atoi}, \lstinline{Getwd},
\lstinline{Chmod}.
Camelcasing works best when you have whole words
to work with: \lstinline{ReadFile, NewWriter, MakeSlice}

\subsection{Package names}
When a package is imported, the package name becomes an accessor for the
contents. After
\begin{lstlisting}
import "bytes"
\end{lstlisting}
the importing package can talk about \func{bytes.Buffer}. It's helpful if
everyone using the package can use the same name to refer to its
contents, which implies that the package name should be good: short,
concise, evocative. By convention, packages are given lower case,
single-word names; there should be no need for underscores or mixedCaps.
Err on the side of brevity, since everyone using your package will be
typing that name. And don't worry about collisions a priori. The package
name is only the default name for imports; it need not be unique across
all source code, and in the rare case of a collision the importing
package can choose a different name to use locally. In any case,
confusion is rare because the file name in the import determines just
which package is being used.

Another convention is that the package name is the base name of its
source directory; the package in \package{src/pkg/container/vector} is imported as
"container/vector" but has name vector, not container\_vector and not
containerVector.

The importer of a package will use the name to refer to its contents, so 
exported names in the package can use that fact to avoid
stutter. For instance, the buffered reader type in the \package{bufio} package is
called \func{Reader}, not \func{BufReader}, because users see it as
\func{bufio.Reader},
which is a clear, concise name. Moreover, because imported entities are
always addressed with their package name, \func{bufio.Reader} does not conflict
with \func{io.Reader}. Similarly, the function to make new instances of
\func{ring.Ring}---which is the definition of a constructor in Go---would normally
be called \func{NewRing}, but since \type{Ring} is the only type exported by the
package, and since the package is called \package{ring}, it's called
just \func{New}.
Clients of the package see that as \func{ring.New}. Use the package structure
to help you choose good names.

Another short example is \func{once.Do}; \func{once.Do(setup)} reads well and would
not be improved by writing \lstinline{once.DoOrWaitUntilDone(setup)}. Long names
don't automatically make things more readable. If the name represents
something intricate or subtle, it's usually better to write a helpful
doc comment than to attempt to put all the information into the name.

Finally, the convention in Go is to use \first{MixedCaps} or mixedCaps rather
than underscores to write multiword names.

\section{Initialization}
Every source file in a package can define an \func{init()} function. This function is
called after the variables in the package have gotten their value. The
\func{init()} function can be used to setup state before the execution
begins.
\todo{Variable section}

\section{Documenting packages}
From \cite{effective_go}
Every package should have a \emph{package comment}, a block comment preceding the
\key{package} clause. For multi-file packages, the package comment only needs to be
present in one file, and any one will do. The package comment should introduce
the package and provide information relevant to the package as a whole. It will
appear first on the \prog{godoc} page and should set up the detailed documentation
that follows. An example is:
\begin{display}
/*
    The regexp package implements a simple library for
    regular expressions.

    The syntax of the regular expressions accepted is:

    regexp:
        concatenation { '|' concatenation }
*/
package regexp
\end{display}

Each defined (and exported) function should has a small line of text
documenting the behavior of the function, from the \package{fmt}
package:
\begin{display}
// Printf formats according to a format specifier and writes to standard output.
func Printf(format string, a ...interface{}) (n int, errno os.Error) \{
\end{display}

\section{Testing packages}
In Go it is customary to write (unit) tests for your package. Writing
tests involves the \package{testing} package and the program \prog{gotest}. Both
have excellent documentation. When you include tests with your package
keep in mind that has to be build using the standard \file{Makefile}
(see section "\titleref{sec:building a package}").

The testing itself is carried out with \prog{gotest}.
The \prog{gotest} program run all the test functions. Without any
defined tests for our \package{even} package a \prog{make test}, yields:
\begin{display}
\pr make test
no test files found (*_test.go)
make: *** [test] Error 2
\end{display}
Let us fix this by defining a test in a test file. Test files reside
in the package directory and are named \file{*\_test.go}. Those test
files are just like other Go program, but \prog{gotest} will only
execute the test functions.
Each test function has the same signature and its name should start
with \lstinline{Test}:
\begin{lstlisting}
func TestXxx(t *testing.T) |\coderemark{Test<Capital>restOftheName}|
\end{lstlisting}

When writing test you will need to tell \prog{gotest} that a test has
failed or was successful. A successful test function just returns. When
test fails you can signal this with the following
functions \cite{go_doc}. These are the most important ones (see \prog{godoc testing}
for more):

\begin{lstlisting}[numbers=none]
func (t *T) Fail()
\end{lstlisting}
\func{Fail} marks the Test function as having failed but continues execution.

\begin{lstlisting}[numbers=none]
func (t *T) FailNow()
\end{lstlisting}
\func{FailNow} marks the Test function as having failed and stops its execution.
Execution will continue at the next Test. So any other test in
\emph{this} file are skipped too.

\begin{lstlisting}[numbers=none]
func (t *T) Log(args ...interface{})
\end{lstlisting}
\func{Log} formats its arguments using default formatting, analogous to
\func{Print()}, and records the text in the error log.

\begin{lstlisting}[numbers=none]
func (t *T) Fatal(args ...interface{})
\end{lstlisting}
\func{Fatal} is equivalent to Log() followed by FailNow().

Putting all this together we can write our very own test file. First
we pick a name: \file{even\_test.go}. Then we add the following contents:
\lstinputlisting[label=src:eventest,caption=Test file for \package{even}
package,numbers=right]{src/even_test.go}
Note that we use \lstinline{package even} on line 1, the test fall in the same
namespace as the package we are testing. This not only convenient, but
also allows tests of unexported function and structures. We then import
the \package{testing} package and on line 5 we define the only test
function in this file. The displayed Go code should not hold any
surprises: we check if the \func{Even} function works OK. 

Now, the moment we've all been waiting for, executing the test:
\begin{display}
\pr make test
6g -o \_gotest\_.6 even.go  even\_test.go
rm -f \_test/even.a
gopack grc \_test/even.a \_gotest\_.6 
PASS
\end{display}
\noindent{}Our test ran and reported \texttt{PASS}. To show how a failed
test look we redefine our test function:
\begin{lstlisting}
// Entering the twilight zone
func TestEven(t *testing.T) {
        if false != Even(2) {
                t.Log("2 should be uneven!")
                t.Fail()
        }   
}
\end{lstlisting}
We now get:
\begin{display}
--- FAIL: even.TestEven
\qquad\qquad2 should be uneven!
FAIL
make: *** [test] Error 1
\end{display}
\noindent{}And you can act accordingly (by fixing the test for instance).

\begin{lbar}
Writing new packages should go hand in hand with writing (some)
documentation and test functions. It will make your code better and it
shows that your really put in the effort.
\end{lbar}

\section{Useful packages}
The standard Go repository includes a huge number of packages and it is
even possible to install more along side your current Go installation. 
We cannot comment on each and everyone of the package, but the following
a worth a mention:

\begin{description}
\item[\package{fmt}]{...}
\item[\package{io}]{...}
\item[\package{strconv}]{...}
\item[\package{os}]{...}
\item[\package{flag}]{...}
\item[\package{json}]{...}
\item[\package{template}]{...}
\item[\package{http}]{...}
\item[\package{unsafe}]{...}
\item[\package{reflect}]{...}
\item[\package{exec}]{...}
\end{description}

\section{Exercises}
\begin{Exercise}[title={Scope},difficulty=4]
\label{ex:scope}
\Question\label{ex:scope q1} Why doesn't the following program do
what you expect?

\begin{lstlisting}[caption=Scoping rules]
package main

import "fmt"
                                                                                                   
func main() {
        for i := 0; i < 10; i++ {
                fmt.Printf("%v\n", i)
        }
	fmt.Printf("%v\n", i)
}
\end{lstlisting}

\end{Exercise}

\begin{Answer}
\Question
When compiling


\end{Answer}


\cleardoublepage
\section{Answers}
\shipoutAnswer
