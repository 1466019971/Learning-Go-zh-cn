\epi{“\lstinline{^}}{\textit{对是否有按位非的运算符的回答。”}\\\textsc{KEN THOMPSON}}
\noindent{}
包是函数和数据的集合,跟 Perl 的包\cite{perl-packages}差不多。
用 \first{\key{package}}{keyword!package} 保留字定义一个包。
文件名不需要与包名一致。包名的约定是使用小写字符。
Go 包可以由多个文件组成,但是使用相同的 \lstinline{package <name>} 这一行。
让我们在文件 \prog{even.go} 中定义一个叫做 \package{even}\index{package!even} 的包。

\lstinputlisting[label=src:even,caption=A small package]{src/even.go}
名称以大写字幕起始的是\emph{可导出}的,可以在包的外部调用,稍候会讨论这个。
现在可以在下面的程序 \prog{myeven.go} 中使用这个包了:

\lstinputlisting[label=src:myeven,caption=Use of the even package]{src/myeven.go}
\showremarks

现在只需要编译和链接,首先是那个包,然后是 \prog{myeven.go},并且链接它们:
\begin{display}
\pr 6g even.go			\coderemark{包}
\pr 6g myeven.go		\coderemark{程序}
\pr 6l -o myeven myeven.6       \coderemark{链接步骤}
\end{display}
然后测试一下:
\begin{display}
\pr ./myeven
Is 5 even? false
\end{display}

在 Go 中,当函数的首字母大写的时候,函数会被从包中导出(在包外部可见,或者说公有的),
因此函数名是 \func{\emph{E}ven}。如果修改 \prog{myeven.go} 的第 7 行,使用未导出的函数
\func{even.odd}:

\noindent\lstinline{fmt.Printf("Is %d even? %v\n", i, even.odd(i))}

由于使用了\emph{私有}的函数,会得到一个编译错误:
\begin{display}
myeven.go:7: cannot refer to unexported name even.odd
myeven.go:7: undefined: even.odd
\end{display}

\noindent{}概括来说:
\begin{itemize}
\item 公有函数的名字以\emph{大写}字母开头;
\index{public}
\item 私有函数的名字以\emph{小写}字幕开头。
\index{private}
\end{itemize}
这个规则同样适用于定义在包中的其他名字(新类型、全局变量)。
注意,“大写”的含义并不仅限于 US ASCII,它被扩展到了整个 Unicode 范围。
所以大写的希腊语、古埃及语都是可以的。

\section{构建一个包}
\label{sec:building a package}
为了创建一个别人可以用的包(只需要 \lstinline{import "even"} 即可使用),首先需要创建一个目录放入包文件。
\begin{display}
\pr mkdir even
\pr cp even.go even/
\end{display}
\noindent{}然后可以使用根据 \package{even} 包改写的 \file{Makefile} 文件。
\lstinputlisting[language=make,caption=用于包的 Makefile,linerange={5,},numbers=right]{src/Makefile}
注意第 3 行定义了 \package{even} 包,而第 4 和第 5 行录入了构成包的文件。
同样注意,这\emph{不是}在第 \ref{chap:basics} 章,第 "\titleref{sec:building a program}"
节中使用的\emph{那个} \prog{Makefile}。最后一行的 \verb|include| 语句是不同的。

如果现在执行 \prog{gomake},一个叫做 "\file{\_go\_.6}" 的文件、一个叫做
"\dir{\_obj/}" 的目录,和 "\dir{\_obj/}" 目录中,叫做 "\file{even.a}" 的文件被创建。
文件 \file{even.a} 是有编译后的 Go 代码的静态库。
通过 \prog{gomake install} 包(好吧,只有 \file{even.a})被安装在\emph{官方}包目录中:
\begin{display}
\pr make install
cp \_obj/even.a \$GOROOT/pkg/linux\_amd64/even.a
\end{display}
\noindent{}安装完后,就可以修改 \key{import} 从
\lstinline{"./even"} 到 \lstinline{"even"}。

但是,如果不希望将包安装到官方的 Go 目录,或者没有权限这么做的话。
在使用 \prog{6/8g} 的时候,可以指定 \flag{\-I} 标志替换包目录。
有了这个标志,你可以让你的导入语句变成(\lstinline{import "even"})却继续部署在自定义的目录中。
所以,下面的命令可以联合包构建(并且链接)\prog{myeven}程序。
\begin{display}
\pr 6g -I even/\_obj myeven.go	\coderemark{Building}
\pr 6l -L even/\_obj myeven.6	\coderemark{Linking}
\end{display}

\noindent{}导入但是没有使用的包会引起一个错误。
\footnote{在 \file{Makefile} 文件中可以访问下面的预定义变量:
\var{\$(GC)} 标示 Go 编译器(6g/8g),
\var{\$(LD)} 当前链接器,以及 \var{\$O} 作为预链接二进制文件的后缀(6/8)。}

\section{标识符}
像在其他语言中一样,Go 的命名是很重要的。在某些情况下,它们甚至有语义上的作用:
例如,在包外是否可见决定于首字母是不是大写。因此有必要花点时间讨论一下 Go 程序的命名规则。

使用的规则是让众所周知的缩写保持原样,而不是去尝试到底哪里应该大写。
\lstinline{Atoi},\lstinline{Getwd},\lstinline{Chmod}。

驼峰式对那些有完整单词的会很好:\lstinline{ReadFile,NewWriter,MakeSlice}。

\subsection{包名}
当包导入(通过 \first{\key{import}}{keyword!import})时,包名成为了内容的入口。
在\index{package!bytes} 
\begin{lstlisting}
import "bytes"
\end{lstlisting}
之后,导入包的可以调用函数\func{bytes.Buffer}。任何使用这个包的人,
可以使用同样的名字访问到它的内容,因此这样的包名是好的:短的、简洁的、好记的。
根据规则,包名是小写的一个单词;不应当有下划线或混合大小写。
由于每个人都可能需要录入这个名字,所以尽可能的简短。不要提前考虑冲突。
包名是导入的默认名称。就上面的 \lstinline{bytes.Buffer} 来说。通过 
\begin{lstlisting}
import bar "bytes"
\end{lstlisting}
它变成了 \lstinline{bar.Buffer}。
因此,它无须在整个代码中唯一,在少有的冲突中,可以给导入的包选择另一个名字在局部使用。
在任何时候,冲突都是很少见的,因为导入的文件名会用来做判断,到底是哪个包使用了。

另一个规则是包名就是代码的根目录名;在 \package{src/pkg/container/vector} 的包,
作为 \var{container/vector} 导入,但名字是 \package{vector},
不是 \package{container\_vector} 也不是 \package{containerVector}。\index{package!container/vector}

导入包将使用其名字引用到内容上,所以导入的包可以利用这个避免罗嗦。
例如,缓冲类型 \package{bufio}\index{package!bufio} 包的读取方法,叫做 \func{Reader},而不是
\func{BufReader},因为用户看到的是 \func{bufio.Reader} 这个清晰、简洁的名字。
更进一步说,由于导入的实例总是它们包名指向的地址,\func{bufio.Reader} 不会与
\func{io.Reader} 冲突。相似的,\func{ring.Ring} 创建新实例的函数——在 Go 中定义的构造函数——通常叫做
\func{NewRing},但是由于 \type{Ring} 是这个包唯一的一个导出的类型,同时,这个包也叫做
\package{ring}\index{package!ring},所以它可以只称作 \func{New}。
包的客户看到的是 \func{ring.New}。用包的结构帮助你选择更好的名字。

另外一个简短的例子是 \func{once.Do};\func{once.Do(setup)} 读起来很不错,并且
命名为 \lstinline{once.DoOrWaitUntilDone(setup)} 不会有任何帮助。长的名字不会让其变得容易阅读。
如果名字表达了一些复杂并且微妙的内容,更好的办法是编写一些有帮助的注释,而不是将所有信息都放入名字里。

最后,在 Go 中使用\first{混合大小写}{MixedCaps} MixedCaps 或者 mixedCaps,
而不是下划线区分含有多个单词的名字。

%% Advanced Go, leave it
%%\section{Initialization}
%%Every source file in a package can define an \func{init()} function. This function is
%%called after the variables in the package have gotten their value. The
%%\func{init()} function can be used to setup state before the execution
%%begins.

\section{包的文档}
\gomarginpar{这段复制于 \cite{effective_go}。}
每个包都应该有\emph{包注释},在 \key{package} 前的一个注释块。
对于多文件包,包注释只需要出现在一个文件前,任意一个文件都可以。
包注释应当对包进行介绍,并提供相关于包的整体信息。
这会出现在 \prog{godoc} 生成的关于包的页面上,并且相关的细节会跟随气候。
来自官方 \package{regexp} 包的例子:
\begin{display}
/*
    The regexp package implements a simple library for
    regular expressions.

    The syntax of the regular expressions accepted is:

    regexp:
        concatenation { '|' concatenation }
*/
package regexp
\end{display}

每个定义(并且导出)的函数应当有一小段文字描述该函数的行为,来自于
 \package{fmt} 包:
\begin{display}
// Printf formats according to a format specifier and writes to standard output.
func Printf(format string, a ...interface{}) (n int, errno os.Error) \{
\end{display}

\section{测试包}
在 Go 中为包编写单元测试应当是一种习惯。编写测试需要包含
\package{testing} 包和程序 \first{\prog{gotest}}{gotest}。
两者都有良好的文档。当对某个包进行测试时,务必要记得必须使用
\file{Makefile} 进行编译(参阅第 "\titleref{sec:building a package}" 节)。


%% CLEANFILES+=hello fib chain run.out
%%

测试本身是由 \prog{gotest} 完成的。
\prog{gotest} 程序调用了所有的测试函数。
\package{even} 包没有定义任何测试函数,执行 \prog{gomake test},这样:
\begin{display}
\pr gomake test
no test files found (*_test.go)
make: *** [test] Error 2
\end{display}
在测试文件中定义一个测试来修复这个。测试文件也在包目录中,被命名为 \file{*\_test.go}。
这些测试文件同 Go 程序中的其他文件一样,但是 \prog{gotest} 只会执行测试函数。
每个测试函数都有相同的标识,它的名字以 \lstinline{Test} 开头:
\begin{lstlisting}
func TestXxx(t *testing.T) |\coderemark{Test<Capital>restOftheNe}|
\end{lstlisting}

编写测试时,需要告诉 \prog{gotest} 测试是失败还是成功。测试成功则直接返回。
当测试失败可以用下面的函数标记 \cite{go_doc}。这是非常重要的(参阅 \prog{godoc testing} 了解更多):

\begin{lstlisting}[numbers=none]
func (t *T) Fail()
\end{lstlisting}
\func{Fail} 标记测试函数失败,但仍然继续执行。

\begin{lstlisting}[numbers=none]
func (t *T) FailNow()
\end{lstlisting}
\func{FailNow} 标记测试函数失败,并且中断其执行。
这将会执行下一个测试。因此,\emph{当前}文件的其他所有测试都被跳过。

\begin{lstlisting}[numbers=none]
func (t *T) Log(args ...interface{})
\end{lstlisting}
\func{Log} 用默认格式对其参数进行格式化,与
\func{Print()} 类似,并且记录文本到错误日志。

\begin{lstlisting}[numbers=none]
func (t *T) Fatal(args ...interface{})
\end{lstlisting}
\func{Fatal} 等价于 \func{Log()} 后跟随 \func{FailNow()}。

将这些凑到一起,就可以编写测试了。首先,选择名字 \file{even\_test.go}。
然后添加下面的内容:
\lstinputlisting[label=src:eventest,caption=even 
包的测试,numbers=right]{src/even_test.go}

注意在第一行使用了 \lstinline{package even},测试使用与被测试的包使用相同的名字空间。
这不仅仅是为了方便,也允许了测试未导出的函数和结构。然后导入
\package{testing} 包,并且在第 5 行定义了这个文件中唯一的测试函数。
展示的 Go 代码应当没有任何惊异的地方:
检查了 \func{Even} 函数是否工作正常。
现在等待了好久的时刻到了,执行测试:
\begin{display}
\pr gomake test
6g -o \_gotest\_.6 even.go  even\_test.go
rm -f \_test/even.a
gopack grc \_test/even.a \_gotest\_.6 
PASS
\end{display}
\noindent{}测试执行并且报告 \texttt{PASS}。成功了!为了展示失败的测试,修改测试函数:
\begin{lstlisting}
// Entering the twilight zone
func TestEven(t *testing.T) {
        if ! Even(2) {
                t.Log("2 should be odd!")
                t.Fail()
        }   
}
\end{lstlisting}
然后得到:
\begin{display}
--- FAIL: even.TestEven
\qquad\qquad2 should be odd!
FAIL
make: *** [test] Error 1
\end{display}
\noindent{}然后你可以以此行事(修复测试的实例)

\begin{lbar}
在编写包的时候应当一边写代码,一边写(一些)文档和测试函数。
这可以让你的程序更好,并且它展示了你的努力。
\end{lbar}

\section{常用的包}
标准的 Go 代码库中包含了大量的包,并且在安装 Go 的时候多数会伴随一起安装。
浏览 \file{\$GOROOT/src/pkg} 目录并且查看那些包会非常有启发。
无法对每个包就加以解说,不过下面的这些值得讨论:
\footnote{描述来自包的 \prog{godoc}。额外的解释用斜体。}

\begin{description}
\item[\package{fmt}]{\index{package!fmt}
包 \package{fmt} 实现了格式化的 I/O 函数,这与 C 的 \func{printf} 和 \func{scanf} 类似。
格式化短语派生于 C 。一些短语(\%-序列)这样使用:

\begin{description}
\item[\%v]{默认格式的值。当打印结构时,加号(\%+v)会增加字段名;}
\item[\%\#v]{Go 样式的值表达;}
\item[\%T]{带有类型的 Go 样式的值表达;}
\end{description}

}

\item[\package{io}]{\index{package!io}
这个包提供了原始的 I/O 操作界面。它主要的任务是对 os 包这样的原始的 I/O 
进行封装,增加一些其他相关,使其具有抽象功能用在公共的接口上。
}
\item[\package{bufio}]{\index{package!bufio}
这个包实现了缓冲的 I/O。它封装于
\lstinline{io.Reader}
和
\lstinline{io.Writer}
对象,创建了另一个对象(Reader 和 Writer)在提供缓冲的同时实现了一些文本 I/O 的功能。
}
\item[\package{sort}]{\index{package!sort}
\package{sort} 包提供了对数组和用户定义集合的原始的排序功能。
}
\item[\package{strconv}]{\index{package!strconv}
\package{strconv} 包提供了将字符串转换成基本数据类型,或者从基本数据类型转换为字符串的功能。
}
\item[\package{os}]{\index{package!os}
\package{os} 包提供了与平台无关的操作系统功能接口。其设计是 Unix 形式的。
}
\item[\package{sync}]{\index{package!sync}
\package{sync} 包提供了基本的同步原语,例如互斥锁。
}
\item[\package{flag}]{\index{package!flag}
\package{flag} 包实现了命令行解析。
\emph{参阅 "\titleref{sec:option parsing}" 在第 \pageref{sec:option parsing} 页。}
}
\item[\package{json}]{\index{package!json}
\package{json} 包实现了编码与解码 RFC 4627 定义的 JSON 对象。
}
\item[\package{template}]{\index{package!template}
数据驱动的模板,用于生成文本输出,例如 HTML。

将模板关联到某个数据结构上进行解析。模板内容指向数据结构的元素(通常结构的字段或者 map 的键)
控制解析并且决定某个值会被显示。模板扫描结构以便解析,而“游标” @ 决定了当前位置在结构中的值。
}
\item[\package{http}]{\index{package!http}
\package{http} 实现了 HTTP 请求、响应和 URL 的解析,并且提供了可扩展的 HTTP 服务和基本的 HTTP 客户端。
}
\item[\package{unsafe}]{\index{package!unsafe}
\package{unsafe} 包包含了 Go 程序中数据类型上所有不安全的操作。
\emph{通常无须使用这个。}
}
\item[\package{reflect}]{\index{package!reflect}
\package{reflect} 包实现了运行时反射,允许程序通过抽象类型操作对象。
通常用于处理静态类型 \lstinline|interface{}| 的值,并且通过 \func{Typeof}
解析出其动态类型信息,通常会返回一个有接口类型 \type{Type} 的对象。
包含一个指向类型的指针, \type{*StructType}、\type{*IntType} 等等,
描述了底层类型的详细信息。可以用于类型转换或者类型赋值。
\emph{参阅 \ref{chap:interfaces},第 "\titleref{sec:introspection and reflection}" 节}。
}
\item[\package{exec}]{\index{package!exec}
\package{exec} 包执行外部命令。
}
\end{description}

\section{练习}
\todo{编写一个测试 stack 包的练习。}

\begin{Exercise}[title={stack 包},difficulty=0]
\label{ex:stack-package}
\Question\label{ex:stack-package q1} 
参考 Q\ref{ex:stack} 练习。在这个练习中将从那个代码中建立一个独立的包。
为 stack 的实现创建一个合适的包,\func{Push}、\func{Pop} 和 \type{Stack} 类型需要被导出。

\Question\label{ex:stack-package q2} 为这个包编写一个单元测试,
至少测试 \func{Push} 后 \func{Pop} 的工作情况。

\end{Exercise}

\begin{Answer}
\Question 在创建 stack 包时,仅有一些小细节需要修改。
首先,导出的函数应当大写首字母,因此应该是 \type{Stack}。
包所在的文件被命名为 \file{stack-as-package.go},内容是:
\lstinputlisting[caption=包里的 Stack]{ex-packages/src/stack-as-package.go}

\Question 为了让单元测试正常工作,需要做一些准备。
下面用一分钟的时间来做这些。首先是单元测试本身。
创建文件 \file{pushpop\_test.go},有如下内容:
\lstinputlisting[caption=Push/Pop 测试]{ex-packages/src/pushpop_test.go}
为了让 \prog{go test} 能够工作,需要将包所在文件放到 
\var{\$GOPATH/src}:\\

\begin{display}
\pr \user{mkdir $GOPATH/src/stack}
\pr \user{cp pushpop_test.go $GOPATH/src/stack}
\pr \user{cp stack-as-package.go $GOPATH/src/stack}
\end{display}

输出:\\

\begin{display}
\pr \user{go test stack}
ok      stack   0.001s
\end{display}
\end{Answer}


\begin{Exercise}[title={Calculator},difficulty=7]
\label{ex:calc}
\Question\label{ex:calc q1} Create a reverse polish calculator. Use your
stack package.

\end{Exercise}

\begin{Answer}
\Question This is one answer:
\lstinputlisting[caption=A (rpn) calculator]{ex-packages/src/calc.go}

\end{Answer}


\cleardoublepage
\section{答案}
\shipoutAnswer
