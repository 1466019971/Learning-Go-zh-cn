\epi{\lstinline{fmt.Printf("\%p", i)}}{Printing the address of a pointer in Go.}
\noindent{}
\subsection{Nil value}
A reference to nothing is represented in Go as \lstinline{nil}. This is
different than a zero value. \todo{nil only for references?}

\begin{lstlisting}
var *a int
a = nil
fmt.Printf("%v\n", a);
\end{lstlisting}
Prints \lstinline{nil}.

\section{Conversions}
\label{sec:conversions}
Sometimes you want to convert a type to another type. In C this is known
as casting a value to another type. This is also possible in Go, but
there are some rules.\todo{there are more rules, but we're trying to
keep it simple...}
You can convert:
\begin{itemize}
\item{
From a \lstinline{string} to a slice of \lstinline{byte}s.
\begin{lstlisting}
mystring = "hello this is string"
byteslice =  []byte(mystring)
\end{lstlisting}
}
\item{
From a slice of \lstinline{byte}s to a \lstinline{string}.
\begin{lstlisting}
The other way around
\end{lstlisting}
}
\end{itemize}
\todo{Maybe put this to the end of this chapter}
%%x is an integer or has type []byte or []int and T is a string type.
%%and back
%% integer/float truncation














You may have wished otherwise, but Go has pointers.
There is however now pointer arithmetic and they are still useful.
Remember Go when you call a function in Go the variables you pass are
pass-by-value. So, for efficiency and the possibility to modify a
passed value \emph{in} the function we have pointers.

%% Do we need a whole chapter on Pointers in Go
Just like in C you declare a pointer by prefixing the type with an `*`,
so:

are declared after variable names, and all type modifiers precede the
\todo{}%
types. So *X is a pointer to an X; [3]X is an array of three X's. The
types are therefore really easy to read just read out the names of the
type modifiers: [] declares something called an array slice; "*"
declares a pointer; [size] declares an array. So []*[3]*int is an array
slice of pointers to arrays of three pointers to ints

\noindent\lstinline{var pint *int   // declare pint to be pointer to int}

Note that it's perfectly OK to return the address of a local variable; the
storage associated with the variable survives after the function returns. In
fact, taking the address of a composite literal allocates a fresh instance each
time it is evaluated, so we can combine these last two lines. \cite{effective_go}

\section{Allocation}
Go has garbage collection, meaning that you don't have to worry about
memory allocation and deallocation. Of course almost every language
since 1980 has this, but it is nice to see garbage collection in a
C-like language. The following sections show how to handle allocation
in Go. There is somewhat an artifical distinction between
\first{\func{new()}} and \first{\func{make()}}. Details follow.

\section{Allocation with \func{new()}}
Go has two allocation primitives, \func{new()} and \func{make()}. They do different
things and apply to different types, which can be confusing, but the
rules are simple. Let's talk about \func{new()} first. It's a built-in function
essentially the same as its namesakes in other languages: \func{new(T)}
allocates zeroed storage for a new item of type \type{T} and returns its
address, a value of type \type{*T}. In Go terminology, it returns a pointer to
a newly allocated zero value of type \type{T}.

Since the memory returned by \func{new()} is zeroed, it's helpful to arrange
that the zeroed object can be used without further initialization. This
means a user of the data structure can create one with \func{new()} and get
right to work. For example, the documentation for \type{bytes.Buffer} states
that "the zero value for Buffer is an empty buffer ready to use."
Similarly, \func{sync.Mutex} does not have an explicit constructor or Init
method. Instead, the zero value for a \func{sync.Mutex} is defined to be an
unlocked mutex.

The zero-value-is-useful property works transitively. Consider this type
declaration.

\begin{lstlisting}
type SyncedBuffer struct {
    lock    sync.Mutex
    buffer  bytes.Buffer
}
\end{lstlisting}
Values of type \type{SyncedBuffer} are also ready to use immediately upon
allocation or just declaration. In this snippet, both \var{p} and
\var{v} will work
correctly without further arrangement.
\begin{lstlisting}
p := new(SyncedBuffer)  // type *SyncedBuffer
var v SyncedBuffer      // type  SyncedBuffer
\end{lstlisting}

\section{Constructors and composite literals}
Sometimes the zero value isn't good enough and an initializing
constructor is necessary, as in this example derived from package
\package{os}.
\begin{lstlisting}
func NewFile(fd int, name string) *File {
    if fd < 0 {
        return nil
    }
    f := new(File)
    f.fd = fd
    f.name = name
    f.dirinfo = nil
    f.nepipe = 0
    return f
}
\end{lstlisting}
There's a lot of boiler plate in there. We can simplify it using a
composite literal, which is an expression that creates a new instance
each time it is evaluated.

\begin{lstlisting}
func NewFile(fd int, name string) *File {
    if fd < 0 {
        return nil
    }
    f := File{fd, name, nil, 0}
    return &f
}
\end{lstlisting}
Note that it's perfectly OK to return the address of a local variable;
the storage associated with the variable survives after the function
returns. In fact, taking the address of a composite literal allocates a
fresh instance each time it is evaluated, so we can combine these last
two lines.

\begin{lstlisting}
return &File{fd, name, nil, 0}
\end{lstlisting}
The fields of a composite literal are laid out in order and must all be
present. However, by labeling the elements explicitly as field:value
pbairs, the initializers can appear in any order, with the missing ones
left as their respective zero values. Thus we could say

\begin{lstlisting}
return &File{fd: fd, name: name}
\end{lstlisting}
As a limiting case, if a composite literal contains no fields at all, it
creates a zero value for the type. The expressions
\lstinline{new(File)} and 
\lstinline|&File{]| are equivalent.

Composite literals can also be created for arrays, slices, and maps,
with the field labels being indices or map keys as appropriate. In these
examples, the initializations work regardless of the values of Enone,
Eio, and Einval, as long as they are distinct.
\begin{lstlisting}
a := [...]string   {Enone: "no error", Eio: "Eio", Einval: "invalid argument"}
s := []string      {Enone: "no error", Eio: "Eio", Einval: "invalid argument"}
m := map[int]string{Enone: "no error", Eio: "Eio", Einval: "invalid argument"}
\end{lstlisting}

\section{Allocation with \func{make()}}
Back to allocation. The built-in function \func{make(T, args)} serves a purpose
different from \func{new(T)}. It creates slices, maps, and channels only, and
it returns an initialized (not zero) value of type T, not *T. The reason
for the distinction is that these three types are, under the covers,
references to data structures that must be initialized before use. A
slice, for example, is a three-item descriptor containing a pointer to
the data (inside an array), the length, and the capacity; until those
items are initialized, the slice is nil. For slices, maps, and channels,
make initializes the internal data structure and prepares the value for
use. For instance,
\lstinline{make([]int, 10, 100)}
allocates an array of 100 ints and then creates a slice structure with
length 10 and a capacity of 100 pointing at the first 10 elements of the
array. (When making a slice, the capacity can be omitted; see the
section on slices for more information.) In contrast, new([]int) returns
a pointer to a newly allocated, zeroed slice structure, that is, a
pointer to a nil slice value.

These examples illustrate the difference between new() and make().
\begin{lstlisting}
var p *[]int = new([]int)       // allocates slice structure; *p == nil; rarely useful
var v  []int = make([]int, 100) // v now refers to a new array of 100 ints

// Unnecessarily complex:
var p *[]int = new([]int)
*p = make([]int, 100, 100)

// Idiomatic:
v := make([]int, 100)
\end{lstlisting}
Remember that make() applies only to maps, slices and channels and does
not return a pointer. To obtain an explicit pointer allocate with new().


\section{Defining your own}
\label{sec:defining your own}
Ofcourse Go allows you to define new types, it does this in (almost) the
same way as in C, with the \key{struct} keyword.

An empty struct is created with \lstinline|var empty struct {}|{}.
A more real-life example would be when we want record seomebody's name
and age in a single ... TODO. We could do

\lstinputlisting[label=src:struct,caption=Structures]{src/struct.go}

Apropos, the output of \lstinline{fmt.Printf("\%v\n", a)} is 
\begin{display}
{Pete, 42}
\end{display}
How nice is that, Go knows how to print your structure! If you
only want to print a one, or a few, field of the structure you'll
need to use \verb|<field name>|. To only print the name:
\begin{lstlisting}
fmt.Printf("%v", a.name)
\end{lstlisting}

Defining a new type:

\begin{lstlisting}
type T struct {
    name string // name of the object
    value int // its value
}
\end{lstlisting}

Or alias a build in one.
Methods on types, and methods on build in type - can not do that,
just create a new type.

\section{Interfaces}
\epi{I have this phobia about having my body penetrated surgically. You
know what I  mean?}{\textit{eXistenZ}\\\textsc{Ted Pikul}}

\noindent{}One of the interesting aspects of the Go language is \first{interface} objects.
\gomarginpar{The following text is copied (and slightly shortend) from \cite{go_interfaces} which is
written by Ian Lance Taylor - one of the original authors of Go.}
In Go, the word interface is overloaded to mean several different
things. Every type has an interface, which is the set of methods defined for
that type. This bit of code defines a struct type \type{S} with one field, and
defines two methods for \type{S}.
\begin{lstlisting}
type S struct { i int }
func (p *S) Get() int { return p.i }
func (p *S) Put(v int) { p.i = v }
\end{lstlisting}
You can also define an \first{interface type}, which is simply a set of methods.
This defines an interface \type{I} with two methods:
\begin{lstlisting}
type I interface {
  Get() int;
  Put(int);
}
\end{lstlisting}
\type{S} is a valid implementation for \type{I}, because it defines the two 
methods which \type{I} requires. Note that this is true even though there is 
no explicit declaration that \type{S} implements \type{I}. A Go program can use 
this fact via yet another meaning of interface, which is an \first{interface value}:

\begin{lstlisting}
func f(p I) { fmt.Println(p.Get()); p.Put(1) }
\end{lstlisting}
Here the variable \var{p} holds a value of interface type. Because
\type{S}
implements \type{I}, we can call \func{f} passing in a pointer to a value of type
\type{S}:

\begin{lstlisting}
var s S; f(&s)
\end{lstlisting}
The reason we need to take the address of \type{S}, rather than a value of type
\type{S}, is because we defined the methods on \type{S} to operate on pointers. This
is not a requirement --- we could have defined the methods to take
values --- but then the \func{Put} method would not work as expected.

The fact that you do not need to declare whether a type implements an
interface means that Go implements a form of \first{duck typing}. This is not
pure duck typing, because when possible the Go compiler will statically
check whether the type implements the interface. However, Go does have a
purely dynamic aspect, in that you can convert from one interface type
to another. In the general case, that conversion is checked at runtime.
If the conversion is invalid --- if the type of the value stored in the
existing interface value does not satisfy the interface to which it is
being converted --- the program will fail with a runtime error.

Interfaces in Go are similar to ideas in several other programming languages:
pure abstract virtual base classes in C++; typeclasses in Haskell; duck typing
in Python; etc. However there is no other language which combines
interface values, static type checking, dynamic runtime conversion, and no
requirement for explicitly declaring that a type satisfies an interface. The
result in Go is powerful, flexible, efficient, and easy to write.

\section{Interface names}
By convention, one-method interfaces are named by the method name plus
the \emph{-er} suffix: Read\emph{er}, Writ\emph{er}, Formatt\emph{er} etc. See chapter
\ref{chap:interfaces} for a complete picture of what interfaces are.

There are a number of such names and it's productive to honor them and
the function names they capture. Read, Write, Close, Flush, String and
so on have canonical signatures and meanings. To avoid confusion, don't
give your method one of those names unless it has the same signature and
meaning. Conversely, if your type implements a method with the same
meaning as a method on a well-known type, give it the same name and
signature; call your string-converter method String not ToString.
from \cite{effective_go}.


\section{Empty interface}
For example, since every type satisfies the empty interface
\type{interface \{\}}:
\begin{lstlisting}
func g(i interface{}) int { return i.(I).Get() }
func h() {
  var s S;
  fmt.Println(g(&s));
  fmt.Println(g(s)); // will fail at runtime
}
\end{lstlisting}
The first call to \func{g} will work fine and will print 0. The second call will fail
at runtime; when using \prog{gccgo}, the program will print
\begin{display}
panic: interface conversion failed: no 'Get' method
\end{display}
This is because, as discussed above, a value of type \type{S} rather than \type{*S} 
does not have any methods.

%% hier anders
So, how does this work? I will describe the current \prog{gccgo} implementation. The
implementation used in the \prog{6g/8g} compiler is generally similar but is different
in important respects.

\section{Methods on ... interfaces?}

To make that even better, methods aren't limited to objects. In fact, Go
doesn't really have objects. Any value, any type at all, can have methods. So
you can make an integer type with its own methods. For example:

\begin{lstlisting}
type Foo int;

func (self Foo) Emit() {
  fmt.Printf("\%v", self);
}

type Emitter interface {
  Emit();
}
\end{lstlisting}

A \key{switch} can also be used to discover the dynamic type of an interface
variable. Such a type switch\gomarginindex{\emph{type switch}}{type switch} uses
the syntax of a type assertion with the keyword type inside the
parentheses. If the switch declares a variable in the expression, the
variable will have the corresponding type in each clause.


Dynamically found out the type we are dealing wtih
\begin{lstlisting}
package main

type PersonAge struct { |\longremark{First we define two structures as a new type, %
\texttt{PersonAge};}|
	name string
	age  int
}

type PersonShoe struct { |\longremark{And \texttt{PersonShoe};}|
	name     string
	shoesize int
}

func main() {
	p1 := new(PersonAge)
	p2 := new(PersonShoe)
	WhichOne(p1)
	WhichOne(p2)
}

func WhichOne(x interface{}) { |\longremark{This function must accept \emph{both} %
types as valid input, so we use the empty interface, which every type implements;}|
	switch t := x.(type) { |\longremark{The type switch;}|
	case *PersonAge:	|\longremark{When allocated with \func{new} its a %
pointer;}|
		println("Age person")
	case *PersonShoe:
		println("Shoe person")
	}
}
\end{lstlisting}

\showremarks



%% uitproberen
\begin{lstlisting}
switch t := interfaceValue.(type) { |\coderemark{Yes, you need to type \texttt{(type)}}|
default:
    fmt.Printf("unexpected type %T", t)  // \%T prints type
case bool:
    fmt.Printf("boolean %t\n", t)
case int:
    fmt.Printf("integer %d\n", t)
case *bool:
    fmt.Printf("pointer to boolean %t\n", *t)
case *int:
    fmt.Printf("pointer to integer %d\n", *t)
}
\end{lstlisting}


\begin{lstlisting}[caption=Using reflect]

type PersonShoe struct {
    name     string "blah"
    shoesize int 
}

func Relax(i interface{}) {
        switch t := reflect.NewValue(i).(type) {
        case *reflect.PtrValue:
        // Elem to resolve the pointer
        // Type() to return the type
        // (*reflect.StructType) to get the structypeType member
        // Field(0) first member in that struct
        // Tag to get the tag
                x := t.Elem().Type().(*reflect.StructType).Field(0).Tag
                println(x)
//              i := x.Type().(*reflect.StructType)
//              fmt.Printf("%T\n", i)
//              fmt.Printf("Fields %d\n", i.NumField())
//              fmt.Printf("Tag %s\n", i.Field(0).Tag)
        default:
                println("ook")

        }
}

\end{lstlisting}
%% Get the value VIA relfect mt.Printf("RdLength %v\n",
%%val.FieldByName("Protocol").(*reflect.UintValue).Get())

%%\newpage %% TODO

\begin{lstlisting}[caption=Introspection using reflection,label=src:introspection]
|\begin{tikzpicture}[overlay]
\draw [->,thick] (2.8,-6.00) node [left] %
{\longremark{We are dealing with a \type{PtrValue} and according %
to the documentation\footnote{\texttt{godoc reflect}}:%
\begin{quote} %
\texttt{\func{func} (v *PtrValue) Elem() Value}\\%
Elem returns the value that v points to. %
If v is a nil pointer, Elem returns a nil Value. %
\end{quote} %
we can use \func{Elem()} to get the type the pointer points to. %
In this case \type{*reflect.StructValue};}} %
to (2.8,-5.20);
%
\draw [->,thick] (3.8,-6.00) node [left] %
{\longremark{\func{Type()} returns \type{reflect.Type};}} %
to (3.8,-5.20);
\draw [->,thick] (5.4,-6.00) node [left] %
{\longremark{%
Again according to the documentation, we have:\\%
\begin{quote} %
\ldots which returns an object with interface %
type \type{Type}.  That contains a pointer to a struct of type %
\type{*StructType}, %
\type{*IntType}, etc. representing the details of the underlying type. %
A type switch or type assertion can reveal which. %
\end{quote} %
So we can access your specific type as a member of this struct. Which %
we do with \type{(*reflect.StructType)};}} %
to (5.4,-5.20);
%
\draw [->,thick] (6.8,-6.00) node [left] %
{\longremark{%
A \type{StructType} has a number of methods, one of which is %
\func{Field($n$)} which returns the $n^{th}$ field of a structure. %
The type of this return is a \type{StructField}; %
}} %
to (6.8,-5.20);
%
\draw [->,thick] (8.4,-6.00) node [left] %
{\longremark{We finally have the type we are after. Now we can use the %
methods defined for \type{*StructType}, like \func{Field(n)}, which %
returns the n$^{th}$ field of our struct as a \type{StructField};}} %
to (8.4,-5.20);
%
\draw [->,thick] (9.4,-6.00) node [left] %
{\longremark{The struct \type{StructField} has a \var{Tag} member which %
returns the tag-name as a string. So on the $0^{th}$ field we can %
unleash \func{.Tag} to access this name: \texttt{Field(0).Tag}. This %
\emph{finally} gives us \texttt{namestr}.}}%
to (9.4,-5.20);
\end{tikzpicture}|
type Person {
    name string "namestr"
    age  int
}

p1 := new(Person)   |\coderemark{\func{new} returns a pointer to Person}|
ShowTag(p1)	    |\coderemark{\func{ShowTag()} is now called with this pointer}|

func ShowTag(i interface{}) {
    switch t := reflect.NewValue(i).(type) { |\coderemark{Type assertion}|
    case *reflect.PtrValue:		     |\coderemark{\var{p1} is a pointer}|
	tag := t.Elem().Type().(||*reflect.StructType).Field(0).Tag
||
\end{lstlisting}


\showremarks

\section{Exercises}
\begin{Exercise}[title={指针},difficulty=6]
\label{ex:pointers}

\Question
假设定义了下面的结构:
\begin{lstlisting}
type Person struct {
    name string
    age	 int
}
\end{lstlisting}

下面两行之间的区别是什么?
\begin{lstlisting}
var p1 Person
p2 := new(Person)
\end{lstlisting}

\Question
下面两个内存分配的区别是什么?
\begin{lstlisting}[numbers=none]
func Set(t *T) {
    x = t
}
\end{lstlisting}
和
\begin{lstlisting}[numbers=none]
func Set(t T) {
    x= &t
}
\end{lstlisting}
\end{Exercise}

\begin{Answer}
\Question
第一行:\lstinline{var p1 Person} 分配了
\texttt{Person}-\emph{值} 给 \var{p1}。\var{p1} 的类型是
\type{Person}。

第二行:\lstinline{p2 := new(Person)} 分配了内存并且将\emph{指针}赋值给
\var{p2}。\var{p2} 的类型是 \type{*Person}。

\Question
在第二个函数中,\var{x} 指向一个新的(堆上分配的)变量
\var{t},其包含了实际参数值的副本。

在第一个函数中,\var{x} 指向了 \var{t} 指向的内容,
也就是实际上的参数指向的内容。

因此在第二个函数,我们有了``额外''的变量存储了相关值的副本。
\end{Answer}


\begin{Exercise}[title={Linked List},difficulty=1]
\label{ex:linkedlist}
\Question
\label{ex:linkedlist q1}
Make use of the package \package{container/list} to create
a (double) linked list. Push the values 1, 2 and 4 to the list and then
print it.

\Question
Create your own linked list implementation. And perform the same actions
as in question \ref{ex:linkedlist q1}
\end{Exercise}

\begin{Answer}
\Question

\Question
\end{Answer}


\begin{Exercise}[title={Cat},difficulty=1]
\label{ex:cat}
\Question \label{ex:cat q1} 编写一个程序,模仿 Unix 的 \prog{cat} 程序。
对于不知道这个程序的人来说,下面的调用显示了文件 \dir{blah} 的内容:
\begin{display}
\pr cat blah
\end{display}

\Question 使其支持 \-n 开关,用于输出每行的行号。

\Question 上面问题中,\ref{q:cat} 提供的解决方案存在一个~Bug。
你能定位并修复它吗?
\end{Exercise}

\begin{Answer}
\Question 下面是 \prog{cat} 的实现,同样支持 \-n 输出每行的行号。
\label{q:cat}
\lstinputlisting[label=src:cat,caption=cat 程序]{ex-beyond/src/cat.go}
\showremarks

\Question 当最后一行不包括换行符时,这个~Bug 就会出现。
更糟糕的情况是,当输入只有一行且没有换行符的时候,什么也不显示。
下面的程序是一个更好的解决方案。
\lstinputlisting[label=src:cat2,caption=一个更好的~cat 程序]{ex-beyond/src/cat2.go}
\end{Answer}


\cleardoublepage
\section{Answers}
\shipoutAnswer
