\begin{Exercise}[title={Calculator},difficulty=7]
\label{ex:calc}
\Question \label{ex:calc q1} Create a simple \emph{reverse polish calculator}. Such a calculator
accept input like \texttt{5 11 +} and then print \texttt{16}; so first the operands and then the
operation. For now it only needs to support addition with two operands. Just like in the
example given.

You will need your stack implementation from exercise \ref{ex:stack}.

\Question \label{ex:calc q2} Extend your calculator to support subtraction (-) and multiplication (*).
\end{Exercise}

\begin{Answer}

\Question

\Question

\end{Answer}
