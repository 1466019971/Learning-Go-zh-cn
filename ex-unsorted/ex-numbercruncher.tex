\begin{Exercise}[title={Number cruncher},difficulty=9]
\label{ex:numbercruncher}
\begin{itemize}
\item{Pick six (6) random numbers from this list:
$$1, 2, 3, 4, 5, 6, 7, 8, 9, 10, 25, 50, 75, 100$$
Numbers may be picked multiple times;}
\item{Pick one (1) random number ($i$) in the range: $1 \ldots 1000$;}
\item{Tell how, by combining the first 6 numbers (or a subset thereof)
with the operators $+$,$-$,$*$ and $/$, you can make $i$;}
\end{itemize}
An example. We have picked the numbers: 1, 6, 7, 8, 8 and 75. And $i$ is
977. This can be done in many different ways, one way is:
$$ ((((1 * 6) * 8) + 75) * 8) - 7 = 977$$ 
or
$$ (8*(75+(8*6)))-(7/1) = 977$$

\Question\label{ex:cruncher q1}
Implement a number cruncher that works like that. Make it print the
solution in a simular format (i.e. output should be infix with
parenthesis) as used above.
\Question\label{ex:cruncher q2}
Calculate \emph{all} possisble solutions and show them (or only show how
many there are). In the example above there are 544 ways to do it.
\end{Exercise}

\begin{Answer}
\Question 
\lstinputlisting[caption=Number cruncher (font size reduced because of
the length of the code),
basicstyle=\tiny\ttfamily]{ex-unsorted/src/permrec.go}

\Question
When starting \prog{permrec} we give 977 as the first argument:
\vspace{1em}
\begin{display}
\pr ./permrec 977
1+(((6+7)*75)+(8/8)) = 977  #1
...                         ...
((75+(8*6))*8)-7 = 977      #542
(((75+(8*6))*8)-7)*1 = 977  #543
(((75+(8*6))*8)-7)/1 = 977  #544
\end{display}

\end{Answer}
