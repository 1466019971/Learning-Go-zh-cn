\epi{%
\begin{itemize}
\item{``并行是关于性能的;}
\item{并发是关于程序设计的。''}
\end{itemize}%
}{\textit{Google IO 2010}\\\textsc{ROBE PIKE}}
\noindent{}在这章中将展示 Go 使用 channel 和 goroutine 开发并行程序的能力。
goroutine 是 Go 并发能力的核心要素。但是,goroutine 到底 \emph{是}什么?来自
\cite{effective_go}:
\begin{quote}
叫做 goroutine 是因为已有的短语——线程、协程、进程等等——传递了不准确的含义。
goroutine 有简单的模型:\emph{它是与其他 goroutine 并行执行的,有着相同地址空间的函数。}。
它是轻量的,仅比分配栈空间多一点点消耗。而初始时栈是很小的,所以它们也是廉价的,
并且随着需要在堆空间上分配(和释放)。
\end{quote}
\first{goroutine}{goroutine} 是一个普通的函数,只是需要使用保留字
\first{\key{go}}{keyword!go} 作为开头。
\begin{lstlisting}
ready("Tea", 2)	    |\coderemark{普通函数调用}|
go ready("Tea", 2)  |\coderemark{\func{ready()} 作为 goroutine 运行}|
\end{lstlisting}
下面程序的思路来自 \cite{go_course_day3}。
让一个函数作为两个 goroutine 执行,goroutine 等待一段时间,然后打印一些内容到屏幕。
在第 14 和 15 行,启动了 goroutine。
\func{main} 函数等待足够的长的时间,这样每个 goroutine 会打印各自的文本到屏幕。
现在是在第 17 行等待 5 秒钟,
但实际上没有任何办法知道,当所有 goroutine 都已经退出应当等待多久。 
\lstinputlisting[numbers=right,label=src:sleeping,firstnumber=8,caption=Go routine 实践,linerange={8,18}]{src/sleep.go}
表 \ref{src:sleeping} 输出:
\begin{display}
I'm waiting         \coderemark{立刻}
Coffee is ready!    \coderemark{1 秒后}
Tea is ready!       \coderemark{2 秒后}
\end{display}
如果不等待 goroutine 的执行(例如,移除第 17 行),程序立刻终止,而任何正在执行的 
goroutine 都\emph{会停止}。
为了修复这个,需要一些能够同 goroutine 通讯的机制。这一机制通过 \first{channels}{channels} 
的形式使用。\first{channel}{channel} 可以与 Unix sehll 中的双向管道做类比:
可以通过它发送或者接收值。这些值只能是特定的类型:channel 类型。
定义一个 channel 时,也需要定义发送到 channel 的值的类型。注意,必须使用
\key{make} 创建 channel:
\begin{lstlisting}
ci := make(chan int)
cs := make(chan string)
cf := make(chan interface{})
\end{lstlisting}
创建 channel \var{ci} 用于发送和接收整数,创建 channel \var{cs} 用于字符串,
以及 channel \var{cf} 使用了空接口来满足各种类型。
向 channel 发送或接收数据,是通过类似的操作符完成的:
\lstinline{<-}. \index{operator!channel}
具体作用则依赖于操作符的位置:
\begin{lstlisting}
ci <- 1	    |\coderemark{\emph{发送}整数 1 到 channel \var{ci}}|
<-ci	    |\coderemark{从 channel \var{ci} \emph{接收}整数}|
i := <-ci   |\coderemark{从 channel \var{ci} \emph{接收}整数,并保存到 \var{i} 中}|
\end{lstlisting}
将这些放到实例中去。
\begin{lstlisting}[numbers=none,caption=Go routines 和 channel,label=src:sleeping with channels]
var c chan int |\longremark{定义 \var{c} 作为 int 型的 channel。就是说:这个 channel 传输整数。%
注意这个变量是全局的,这样 goroutine 可以访问它;}|

func ready(w string, sec int) {
	time.Sleep(time.Duration(sec) * time.Second)
	fmt.Println(w, "is ready!")
	c <- 1	|\longremark{发送整数 1 到 channel \var{c};}|
}

func main() {
	c = make(chan int) |\longremark{初始化 \var{c};}|
	go ready("Tea", 2) |\longremark{用保留字 \key{go} 开始一个 goroutine;}|
	go ready("Coffee", 1)
	fmt.Println("I'm waiting, but not too long")
	<-c |\longremark{等待,直到从 channel 上接收一个值。注意,收到的值被丢弃了;}|
	<-c |\longremark{两个 goroutines,接收两个值。}|
}
\end{lstlisting}

\showremarks
这里仍然有一些丑陋的东西;不得不从 channel 中读取两次(第 14 和 15 行)。
在这个例子中没问题,但是如果不知道有启动了多少个 goroutine 怎么办呢?
这里有另一个 Go 内建的保留字:\first{\key{select}}{keyword!select}。
通过 \key{select}(和其他东西)可以监听 channel 上输入的数据。

在这个程序中使用 \key{select},并不会让它变得更短,因为运行的 go\-routine 太少了。
移除第 14 和 15 行,并用下面的内容替换它们:
\begin{lstlisting}[caption=使用 select,numbers=right,firstnumber=14]
L: for {
	select {
	case <-c:
		i++ 
		if i > 1 { 
			break L
		}   
	}   
}   
\end{lstlisting}
现在将会一直等待下去。只有当从 channel \var{c} 上收到多个响应时才会退出循环 \var{L}。

\subsection{使其并行运行}
虽然 goroutine 是并发执行的,但是它们并不是并行运行的。如果不告诉 Go 额外的东西,
同一时刻只会有一个 goroutine 执行。利用 \func{runtime.GOMAXPROCS(n)} 可以设置 goroutine 并行执行的数量。
来自文档:
\begin{quote}
GOMAXPROCS 设置了同时运行的 CPU 的最大数量,并返回之前的设置。如果 n < 1,不会改变当前设置。
\emph{当调度得到改进后,这将被移除。}
\end{quote}
如果不希望修改任何源代码,同样可以通过设置环境变量
 \verb|GOMAXPROCS| 为目标值。
%% test
%%\marginpar{%
%%$$\left\{
%%\begin{array}{l}
%%\parbox{2cm}{
%%hallo Yppp hallo Yppp hallo Yppp
%%hallo Yppp hallo Yppp hallo Yppp
%%hallo Yppp
%%}
%%\end{array}
%%\right.$$
%%}

%%\section{So many channels and still \ldots}
\section{更多关于 channel}
\label{sec:more on channels}
当在 Go 中用 \lstinline{ch := make(chan bool)} 创建 chennel 时,bool 型的
\first{无缓冲 channel}{channel!unbuffered} 会被创建。
这对于程序来说意味着什么呢?首先,如果读取(\lstinline{value := <-ch})它将会被阻塞,直到有数据接收。
其次,任何发送(\lstinline{ch<-5}) 将会被阻塞,直到数据被读出。
无缓冲 channel 是在多个 goroutine 之间同步很棒的工具。
\index{channel!blocking read}
\index{channel!blocking write}

不过 Go 也允许指定 channel 的缓冲大小,很简单,就是 channel 可以存储多少元素。
\lstinline{ch := make(chan bool, 4)},创建了可以存储 4 个元素的 bool 型 channel。
在这个 channel 中,前 4 个元素可以无阻塞的写入。当写入第 5$^{个}$ 元素时,代码
\emph{将会}阻塞,直到其他 goroutine 从 channel 中读取一些元素,腾出空间。
\index{channel!non-blocking read}
\index{channel!non-blocking write}

一句话说,在 Go 中下面的为 true:
$$
\textrm{\lstinline{ch := make(chan type, value)}}
\left\{
\begin{array}{ll}
value == 0 & \rightarrow \textrm{无缓冲} \\
value >  0 & \rightarrow \textrm{缓冲 \emph{value} 的元素}
\end{array}
\right.
$$

\subsection{关闭 channel}
当 channel 被关闭后,读取端需要知道这个事情。下面的代码演示了如何检查 channel 是否被关系。
\begin{lstlisting}
x, ok = <-ch
\end{lstlisting}
当 \lstinline{ok} 被赋值为 \lstinline{true} 意味着 channel 尚未被关闭,\emph{同时} 可以读取数据。
否则 \var{ok} 被赋值为 \lstinline{false}。在这个情况下表示 channel 被关闭。

\todo{more needs to be written}

\section{练习}
\begin{Exercise}[title={Channel},difficulty=4]
\label{ex:channels}
\Question\label{ex:channels q1} 修改在练习 Q\ref{ex:for-loop} 中创建的程序,
换句话说,主体中调用的函数现在是一个 goroutine 并且使用 channel 通讯。
不用担心 goroutine 是如何停止的。

\Question\label{ex:channels q2} 在完成了问题 \ref{ex:channels q1} 后,仍有一些待解决的问题。
其中一个麻烦是 goroutine 在 \func{main.main()} 结束的时候,没有进行清理。
更糟的是,由于 \func{main.main()} 和 \func{main.shower()} 的竞争关系,不是所有数字都被打印了。
本应该打印到 9,但是有时只打印到 8。添加第二个退出 channel,可以解决这两个问题。试试吧。
\footnote{需要用到 \func{select} 语句。}

\end{Exercise}

\begin{Answer}
\Question 程序可能的形式是: 
\lstinputlisting[label=go-chan,caption=Go 的 channel,numbers=right]{ex-channels/src/for-chan.go}
以通常的方式开始,在第 6 行创建了一个新的 int 类型的 channel。下一行调用了
\func{shower} 函数,用 \prog{ch} 变量作为参数,这样就可以与其通讯。然后进入 for 循环(第 8-10 行),
在循环中发送(通过 \lstinline{<-})数字到函数(现在是 goroutine)\func{shower}。
在函数 \func{shower} 中等待(阻塞方式),直到接收到了数字(第 15 行)。
每个收到的数字都被打印(第 16 行)出来,然后继续第 14 行开始的死循环。

\Question 答案是
\lstinputlisting[label=go-quit-chan,caption=添加额外的退出 channel,numbers=right]{ex-channels/src/for-quit-chan.go}
在第 20 行从退出 channel 读取并丢弃该值。可以使用 \lstinline{q := <-quit},
但是可能只需要用这个变量一次——在 Go 中是非法的。另一种办法,你可能已经想到了:
\lstinline{_ = <-quit}。在 Go 中这是合法的,但是第 20 行的形式在 Go 中更好。
\end{Answer}


\begin{Exercise}[title={Fibonaci II},difficulty=7]
\label{ex:fibonaci II}
\Question\label{ex:fibonaci II q1}
This is the same exercise as the one given page \pageref{ex:fibonaci} 
in exercise \ref{ex:fibonaci}. For completeness the complete question:

\begin{quote}
The Fibonaci sequence starts as follows: $1, 1, 2, 3, 5, 8, 13, \ldots$
Or in mathematical terms: $ x_1 = 1; x_2 = 1; x_n = x_{n-1} +
x_{n-2}\quad\forall n > 2 $.

Write a function that takes an \type{int} value and gives to
Fibonaci sequence up to that value.
\end{quote}

\begin{lbar}
\emph{But} now the twist: You must use channels.
\end{lbar}


\end{Exercise}

\begin{Answer}
\Question
The following program calculates the Fibonaci numbers using channels.
\lstinputlisting[label=src:fib II,caption=A Fibonaci function in Go]{ex-channels/src/fib.go}
\end{Answer}




\cleardoublepage
\section{答案}
\shipoutAnswer
