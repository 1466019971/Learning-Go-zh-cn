\epi{%
\begin{itemize}
\item{Parallelism is about performance;}
\item{Concurrency is about program design.}
\end{itemize}%
}{\textit{Google IO 2010}\\\textsc{ROBE PIKE}}
\noindent{}In this chapter we will show off Go's ability for
concurrent programming using channels and goroutines. Goroutines
are the central entity in Go's ability for concurrency. But what
\emph{is} a goroutines? From \cite{effective_go}:
\begin{quote}
They're called goroutines because the existing terms --- threads, coroutines,
processes, and so on --- convey inaccurate connotations. A goroutine has a simple
model: \emph{it is a function executing in parallel with other goroutines in the same
address space}. It is lightweight, costing little more than the allocation of
stack space. And the stacks start small, so they are cheap, and grow by
allocating (and freeing) heap storage as required.
\end{quote}
A \first{goroutine}{goroutine} is a normal function, except that you start
it with the keyword \first{\key{go}}{keyword!go}.
\begin{lstlisting}
ready("Tee", 2)	    |\coderemark{Normal function call}|
go ready("Tee", 2)  |\coderemark{\func{ready()} started as goroutine}|
\end{lstlisting}
The following idea for a program was taken from \cite{go_course_day3}. 
We run a function as two goroutines, the goroutines wait for an amount of
time and then print something to the screen. 
On the lines 14 and 15 we start the goroutines.
The \func{main} function
waits long enough, so that both goroutines will have printed their text. Right
now we wait for 5 seconds (\func{time.Sleep()} counts in ns) on line 17, but in fact we have no idea how
long we should wait until all goroutines have exited.
\lstinputlisting[numbers=right,label=src:sleeping,firstnumber=8,caption=Go routines in action,linerange={8,18}]{src/sleep.go}
Listing \ref{src:sleeping} outputs:
\begin{display}
I'm waiting         \coderemark{right away}
Coffee is ready!    \coderemark{after 1 second}
Tee is ready!       \coderemark{after 2 seconds}
\end{display}

If we did not wait for the goroutines (i.e. remove line 17) the program
would be terminated immediately and any running goroutines would
\emph{die with it}. 
To fix this we need some kind of mechanism which allows us to
communicate with the goroutines. This mechanism is available
to us in the form of \first{channels}{channels}. A
\first{channel}{channel} can be
compared to a two-way pipe in Unix shells: you can send to and receive
values from it. Those values can only be of a specific type: the
type of the channel. If we define a channel, we must also define the
type of the values we can send on the channel. Note that we must use
\key{make} to create a channel:
\begin{lstlisting}
ci := make(chan int)
cs := make(chan string)
cf := make(chan interface{})
\end{lstlisting}
Makes \var{ci} a channel on which we can send and receive integers,
makes \var{cs} a channel for strings and \var{cf} a channel for types
that satisfy the empty interface. 
Sending on a channel and receiving from it, is done with the same operator:
\lstinline{<-}. \index{operator!channel}
Depending on the operands it figures out what to do:
\begin{lstlisting}
ci <- 1	    |\coderemark{\emph{Send} the integer 1 to the channel \var{ci}}|
<-ci	    |\coderemark{\emph{Receive} an integer from the channel \var{ci}}|
i := <-ci   |\coderemark{\emph{Receive} from the channel \var{ci} and storing it in \var{i}}|
\end{lstlisting}
Lets put this to use.
\begin{lstlisting}[numbers=right,caption=Go routines and a channel,label=src:sleeping with channels]
var c chan int |\longremark{Declare \var{c} to be a variable that is a %
channel of ints. That is: this channel can move integers. Note %
that this variable is global so that the goroutines have access to it;}|

func ready(w string, sec int) {
	time.Sleep(int64(sec) * 1e9)
	fmt.Println(w, "is ready!")
	c <- 1	|\longremark{Send the integer 1 on the channel \var{c};}|
}

func main() {
	c = make(chan int) |\longremark{Initialize \var{c};}|
	go ready("Tee", 2) |\longremark{Start the goroutines with the keyword \key{go};}|
	go ready("Coffee", 1)
	fmt.Println("I'm waiting, but not too long")
	<-c |\longremark{Wait until we receive a value from the channel. Note that the value we receive is discarded;}|
	<-c |\longremark{Two goroutines, two values to receive.}|
}
\end{lstlisting}
\todo{Do this with range, use select later}

\showremarks
There is still some remaining ugliness; we have to read twice from
the channel (lines 14 and 15). This is OK in this case, but what if
we don't know how many goroutines we started? This is where another
Go built-in comes in: \first{\key{select}}{keyword!select}. With \key{select} you 
can (among other things) listen for incoming data on a channel.

Using \key{select} in our program does not really make it shorter,
because we run too few goroutines. We remove the lines 14 and 15 and
replace them with the following:
\begin{lstlisting}[caption=Using \key{select},numbers=right,firstnumber=14]
L: for {
	select {
	case <-c:
		i++ 
		if i > 1 { 
			break L
		}   
	}   
}   
\end{lstlisting}

\subsection{Make it run in parallel}
While our goroutines were running concurrent, they were not running in
parallel. When you do not tell Go anything there can only be one
goroutine running at a time. With \func{runtime.GOMAXPROCS(n)} you
can set the number of goroutines that can run in parallel. From
the documentation:
\begin{quote}
GOMAXPROCS sets the maximum number of CPUs that can be executing
simultaneously and returns the previous setting. If n < 1, it does not
change the current setting. \emph{This call will go away when the scheduler
improves.}
\end{quote}
If you do not want to change any source code you can also set an
environment variable \verb|GOMAXPROCS| to the desired value.
%% test
%%\marginpar{%
%%$$\left\{
%%\begin{array}{l}
%%\parbox{2cm}{
%%hallo Yppp hallo Yppp hallo Yppp
%%hallo Yppp hallo Yppp hallo Yppp
%%hallo Yppp
%%}
%%\end{array}
%%\right.$$
%%}

%%\section{So many channels and still \ldots}
\section{More on channels}
\label{sec:more on channels}
When you create a channel in Go with \lstinline{ch := make(chan bool)}, 
an \first{unbuffered channel}{channel!unbuffered} for
bools is created. What does this mean for your program? For one, if you
read (\lstinline{value := <-ch}) it will block until there is data to
receive. Secondly anything sending (\lstinline{ch<-5}) will block until there
is somebody to read it. 
Unbuffered channels make a perfect tool for synchronising multiple
goroutines.
\index{channel!blocking read}
\index{channel!blocking write}

But Go allows you to specify the buffer size of
a channel, which is quite simple how many elements a channel can hold.
\lstinline{ch := make(chan bool, 4)}, creates a buffered channels of
bools that can hold 4 elements. The first 4 elements in this channels
are written without any blocking.
When you write the 5$^{th}$ element, your
code \emph{will} block, until another goroutine reads some elements from the
channel to make room. 
\index{channel!non-blocking read}
\index{channel!non-blocking write}

Although reads from channels block, you can perform an
non-blocking read with the following syntax: \todo{Test this}
\begin{lstlisting}
x, ok = <-ch
\end{lstlisting}
Where \lstinline{ok} is set to \lstinline{true} when there was something
to read (otherwise it is \lstinline{false}. 
And if that was the case \lstinline{x} get the value read
from the channel. 
In conclusion, the following is true in Go:
$$
\textrm{\lstinline{ch := make(chan type, value)}}
\left\{
\begin{array}{ll}
value == 0 & \rightarrow \textrm{unbuffered (blocking)} \\
value >  0 & \rightarrow \textrm{buffered (non-blocking) up to \emph{value} elements)}
\end{array}
\right.
$$

\subsection{Closing channels}
\todo{close 'n closed}
%%Furthermore, while receives from a channel normally block, you can also do a non-blocking receive from a channel.

\section{Exercises}
\begin{Exercise}[title={Channel},difficulty=4]
\label{ex:channels}
\Question\label{ex:channels q1} 修改在练习 Q\ref{ex:for-loop} 中创建的程序,
换句话说,主体中调用的函数现在是一个 goroutine 并且使用 channel 通讯。
不用担心 goroutine 是如何停止的。

\Question\label{ex:channels q2} 在完成了问题 \ref{ex:channels q1} 后,仍有一些待解决的问题。
其中一个麻烦是 goroutine 在 \func{main.main()} 结束的时候,没有进行清理。
更糟的是,由于 \func{main.main()} 和 \func{main.shower()} 的竞争关系,不是所有数字都被打印了。
本应该打印到 9,但是有时只打印到 8。添加第二个退出 channel,可以解决这两个问题。试试吧。
\footnote{需要用到 \func{select} 语句。}

\end{Exercise}

\begin{Answer}
\Question 程序可能的形式是: 
\lstinputlisting[label=go-chan,caption=Go 的 channel,numbers=right]{ex-channels/src/for-chan.go}
以通常的方式开始,在第 6 行创建了一个新的 int 类型的 channel。下一行调用了
\func{shower} 函数,用 \prog{ch} 变量作为参数,这样就可以与其通讯。然后进入 for 循环(第 8-10 行),
在循环中发送(通过 \lstinline{<-})数字到函数(现在是 goroutine)\func{shower}。
在函数 \func{shower} 中等待(阻塞方式),直到接收到了数字(第 15 行)。
每个收到的数字都被打印(第 16 行)出来,然后继续第 14 行开始的死循环。

\Question 答案是
\lstinputlisting[label=go-quit-chan,caption=添加额外的退出 channel,numbers=right]{ex-channels/src/for-quit-chan.go}
在第 20 行从退出 channel 读取并丢弃该值。可以使用 \lstinline{q := <-quit},
但是可能只需要用这个变量一次——在 Go 中是非法的。另一种办法,你可能已经想到了:
\lstinline{_ = <-quit}。在 Go 中这是合法的,但是第 20 行的形式在 Go 中更好。
\end{Answer}


\begin{Exercise}[title={Fibonaci II},difficulty=7]
\label{ex:fibonaci II}
\Question\label{ex:fibonaci II q1}
This is the same exercise as the one given page \pageref{ex:fibonaci} 
in exercise \ref{ex:fibonaci}. For completeness the complete question:

\begin{quote}
The Fibonaci sequence starts as follows: $1, 1, 2, 3, 5, 8, 13, \ldots$
Or in mathematical terms: $ x_1 = 1; x_2 = 1; x_n = x_{n-1} +
x_{n-2}\quad\forall n > 2 $.

Write a function that takes an \type{int} value and gives to
Fibonaci sequence up to that value.
\end{quote}

\begin{lbar}
\emph{But} now the twist: You must use channels.
\end{lbar}


\end{Exercise}

\begin{Answer}
\Question
The following program calculates the Fibonaci numbers using channels.
\lstinputlisting[label=src:fib II,caption=A Fibonaci function in Go]{ex-channels/src/fib.go}
\end{Answer}




\cleardoublepage
\section{Answers}
\shipoutAnswer
