\begin{Exercise}[title={Interfaces and max()},difficulty=2]
\Question
In exercise Q\ref{ex:minmax} we created a max function that works on
a slice of integers.
The question now is to create
a program that shows the maximum number and that works for both integers and floats.
Try to make your program as generic as possible, although that is quite difficult in
this case.
\end{Exercise}

\begin{Answer}
\Question
The following program calculates a maximum. It is as generic as you can get
with Go.

\begin{lstlisting}[caption=Generic way of calculating a maximum]
package main

func Less(l, r interface{}) bool { |\longremark{We could have chosen to make the %
return type of this function an \mbox{\type{interface\{\}}}, but that would mean %
that a caller would always have to do a type assertion to extract the actual type %
from the interface;}|
	switch l.(type) {
	case int:
		if _, ok := r.(int); ok {
			return l.(int) < r.(int) |\longremark{All parameters are confirmed to be %
integers. Now perform the comparison;}|
		}
	case float32:
		if _, ok := r.(float32); ok {
			return l.(float32) < r.(float32) |\longremark{Parameters are \type{float32};}|
		}
	}
	return false
}

func main() {
	var a, b, c int = 5, 15, 0
	var x, y, z float32 = 5.4, 29.3, 0.0

	if c = a; Less(a, b) { |\longremark{Get the maximum of \var{a} and \var{b};}|
		c = b
	}
	if z = x; Less(x, y) { |\longremark{Same for the floats.}|
		z = y
	}
	println(c, z)
}
\end{lstlisting}
\showremarks
\end{Answer}
