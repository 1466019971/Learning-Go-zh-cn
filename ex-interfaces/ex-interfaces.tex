\begin{Exercise}[title={Interfaces and compilation},difficulty=6]
\Question
The code in listing \ref{src:interface fail} on page
\pageref{src:interface fail} compiles OK --- as stated 
in the text. But when you run it you'll get a runtime error, so
something \emph{is} wrong. Why does the code compile cleanly then?
\end{Exercise}

\begin{Answer}
\Question
The code compiles because an integer type implements the empty interface
and that is the check that happens at compile time.

A proper way to fix this is to test if such an empty interface can
be converted and, if so, call the appropriate method. The Go code
that defines the function \func{g} in listing \ref{src:interface empty}
-- repeated here:
\begin{lstlisting}
func g(any interface{}) int { return any.(I).Get() }
\end{lstlisting}

\noindent{}Should be changed to become:
\begin{lstlisting}
func g(any interface{}) int {
    if v, ok := any.(I); ok {	// Check if any can be converted
	return v.Get()		// If so invoke Get()
    }
    return -1			// Just so we return anything
}
\end{lstlisting}
If \func{g()} is called now there are no run-time errors anymore. The
idiom used is called ``comma ok'' in Go.
\end{Answer}
