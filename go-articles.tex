\section{表达式与语句的对比}
\label{sec:expression versus statement}
在这部分讨论一下表达式和语句,但是它们之前的区别\emph{是}什么呢?
\gomarginpar{这部分来自于 \cite{so_expression_vs_statement}。}
简单来说:
\begin{description}
\item[Expression] 某些同值有关的定义,例如:
\lstinline{1+2/x};
\item[Statement] 有一些功能的一行代码,例如:
\lstinline{goto Error} .
\end{description}

在早期通用语言中,例如 Fortran,这一区别是相当明确的。
在 Fortran 中,语句是一个执行单元:"一件要做的事"。 
表达式本身无法做任何事情。必须将其赋值给一个变量。
\begin{display}
1 + 2 / X
\end{display}
在 Fortran 中是错误的,因为这什么都不能做。
必须对这个表达式做一些事情:
\begin{display}{X = 1 + 2 / X}\end{display}

第一个混淆这个界限的早期流行语言是 C。C 的设计者认为,允许执行一个表达式,
并且将结果丢弃不会带来任何的坏处。在 C 中,每个表达式都可以是一个语句:
\begin{display}1 + 2 / x\end{display}
是绝对合法的语句,虽然完全没有发生任何事情。
为什么?因为在 C 中,表达式可能会有副作用——可能会改变一些内容:
\begin{display}{1 + 2 / callfunc(12)}\end{display}

由于 \func{callfunc()} 可能会做一些有用的事情。
一旦允许任何表达式都是语句,同样就是允许赋值符号(=)出现在表达式中。
这也就是为什么 C 允许这样做:\lstinline{callfunc(x = 2)}。
这样执行了表达式 \lstinline{x = 2}(赋值 2 到 x)然后将其(那个 2)
传递到函数 \func{callfunc()}。

幸运的是这个混淆在 Go 中明显降低了。
Go 中将表达式当作语句使用的,仅有函数调用、方法调用、和 channel 操作。
这和 Fortran 很像。

函数语言例如 Lisp 没有语句。所有都是表达式。
