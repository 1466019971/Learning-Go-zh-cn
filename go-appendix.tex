\section{Expression versus statement}
\label{sec:expression versus statement}
In this book we talk about expressions and statements, but%
\gomarginpar{This section comes from \cite{so_expression_vs_statement}.}
what \emph{is} the difference between the two?
In short:
\begin{description}
\item[Expression] Something which evaluates to a value, like:
\lstinline{1+2/x} ;
\item[Statement] A line of code which does something, like
\lstinline{goto Error} .
\end{description}

The distinction was crystal-clear in the earliest general-purpose
programming languages, like FORTRAN. In FORTRAN, a statement was one
unit of execution: "a thing that you did". 
An expression on its own couldn't do anything. You had to assign it to a
variable.
\begin{display}
1 + 2 / X
\end{display}
is an error in FORTRAN, because it doesn't do anything. You had to do
something with that expression: 
\begin{display}{X = 1 + 2 / X}\end{display}

The earliest popular language to blur the lines was C. The designers of
C realized that no harm was done if you were allowed to evaluate an
expression and throw away the result. In C, every expression could be a
statement: 
\begin{display}1 + 2 / x\end{display}
is a totally legit statement even though absolutely nothing will happen.
Why? Because in C, expressions could have side-effects --- they could
change something: \begin{display}{1 + 2 / callfunc(12)}\end{display}

Because \func{callfunc()} might just do something useful.
Once you allow any expression to be a statement, you might as well allow
the assignment operator (=) inside expressions. That's why C lets you do
things like: \lstinline{callfunc(x = 2)}.
This evaluates the expression \lstinline{x = 2} (assigning the value of 2 to x) and
then passes that (the 2) to the function \func{callfunc()}.

This blurring of expressions and statements occurs in all the
C-derivatives (C, C++, C\#, Java and of course Go), which still have some
statements (like \key{while}) but which allow almost any expression to be used
as a statement. Functional languages like Lisp don't have statements.
All they have is expressions. 
