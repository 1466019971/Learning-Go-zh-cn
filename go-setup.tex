\defaultfontfeatures{Scale=MatchLowercase,Mapping=tex-text}
\setmainfont{Droid Serif}
\setsansfont{Droid Sans}
%%\setmonofont[Scale=0.70]{Droid Sans Mono} %% No italic font face??
\setmonofont[SmallCapsFont={DejaVu Sans Mono},Scale=0.70]{DejaVu Sans Mono}
\def\inputGnumericTable{}

%% list of answers
\newlistof{listofex}{ex}{List of Exercises}
\newlistentry{exercise}{ex}{0}

\newlistof{listofcode}{code}{List of Code Examples}
\newlistentry{code}{code}{0}

%% need to do this for code examples too
%%\renewcommand{\lstlistlistingname}{List of Code Examples}
%% toc
\renewcommand{\tocheadstart}{}
\renewcommand{\aftertoctitle}{\pagestyle{blocks}}
\renewcommand{\aftertoctitle}{\thispagestyle{empty}\afterchaptertitle\pagestyle{blocks}}

%%\renewcommand{\printtoctitle}[1]{}
\renewcommand{\contentsname}{Table of Contents}
\renewcommand{\tocmark}{\markboth{\myfamily \typename: \contentsname}{\myfamily \contentsname}}
%% new print the titles as section not as chapter, that explains
%% why the page is headerless. the page style is empty
%% lof
\renewcommand{\lofheadstart}{}
\renewcommand{\afterloftitle}{\thispagestyle{blocks}}
\renewcommand{\printloftitle}[1]{\section*{#1}}
\renewcommand{\lotmark}{\markboth{\myfamily \typename:
\contentsname}{\myfamily \listfigurename}}
%% lot
\renewcommand{\lotheadstart}{}
\renewcommand{\afterlottitle}{\thispagestyle{blocks}}
\renewcommand{\printlottitle}[1]{\section*{#1}}
\renewcommand{\lofmark}{\markboth{\myfamily \typename:
\contentsname}{\myfamily \listtablename}}
%% ex
\renewcommand{\exheadstart}{}
\renewcommand{\afterextitle}{\thispagestyle{blocks}}
\renewcommand{\printextitle}[1]{\section*{#1}}
\renewcommand{\exmark}{\markboth{\myfamily \typename: \contentsname}{\myfamily List of Exercises}}
%% code
\renewcommand{\codeheadstart}{}
\renewcommand{\aftercodetitle}{\thispagestyle{blocks}}
\renewcommand{\printcodetitle}[1]{\section*{#1}}
\renewcommand{\codemark}{\markboth{\myfamily \typename: \contentsname}{\myfamily List of Code Examples}}

\nobibintoc
\renewcommand*{\indexmark}{%
\markboth{\myfamily \typename{} \thechapter: \indexname}{\myfamily\indexname}%
}

\onecolindexfalse  %% doesn't do its things as advertised
\noindexintoc
\makeindex

%% make quote print italics and other fancy stuff
\newcommand{\qquote}{{\scalefont{4.00}{``}}}
\expandafter\def\expandafter\quote\expandafter{\quote\em}

%%\DeclareCaptionFont{white}{\color{white}}
%%\DeclareCaptionFormat{listing}{\colorbox[cmyk]{0.43, 0.35,
%%0.35,0.01}{\parbox{0.5\textwidth}{\hspace{0.5em}#1#2#3}}} % 0.98 same as listings
%%\captionsetup[lstlisting]{format=listing,labelfont=white,textfont=white,singlelinecheck=false,margin=0em,font={bf,footnotesize}}

%% Listings
\lstdefinelanguage{Go}
  {morekeywords={break,cap,case,chan,const,continue,copy,default,defer,else,fallthrough,%
  for,func,go,goto,if,import,interface,len,make,map,new,package,range,return,select,%
  struct,switch,type,var,%  % types
  uint8,uint16,uint32,uint64,int8,int16,int32,int64,float32,float64,byte,%
  complex,complex32,complex64,%
  int,uint,float,bool,uintptr,string,%
  iota,%
  },%
  otherkeywords={<-,!,;,\{,\}},%  %% nog beter maken
    sensitive=true,%
    morecomment=[l]{//},%
    morecomment=[s]{/*}{*/},%
    morecomment=[n]{(*}{*)},%
    morestring=[b]",%
    morestring=[b]',%
  }[]%
\lstset{language=Go,inputencoding=utf8,extendedchars=false,texcl,escapechar=\|,basicstyle=\ttfamily,keywordstyle=\bfseries,numbers=none,numberblanklines=false,showstringspaces=false,breaklines=true,numberstyle=\small\ttfamily,xleftmargin=\parindent,xrightmargin=1em,linewidth=0.98\linewidth}

\newcommand{\coderemark}[1]{\qquad$\leftarrow \textit{\small #1}$}

%% Exercises
\renewcommand{\ExerciseHeaderTitle}{\ExerciseTitle}
\renewcommand{\ExerciseHeaderLabel}{}
\renewcommand{\ExerciseName}{}	%% was 'Exercise'
\renewcommand{\ExerciseHeaderNB}{\theExercise}
%% This one is actually used
\renewcommand{\ExerciseHeader}{\vspace{.7ex}\noindent\textbf{Q\theExercise}. (\number\ExerciseDifficulty) \ExerciseTitle\quad%
\addcontentsline{ex}{exercise}{\numberline{\theExercise}(\number\ExerciseDifficulty) \ExerciseTitle}}
\renewcommand{\AnswerHeader}{\vspace{.7ex}\noindent\textbf{A\theExercise}.  (\number\ExerciseDifficulty) \ExerciseTitle\quad}

%% Style commands
\newcommand{\func}[1]{\texttt{#1}}
\newcommand{\key}[1]{\texttt{\textbf{#1}}}
\newcommand{\type}[1]{\texttt{\textbf{#1}}}
\newcommand{\prog}[1]{\texttt{#1}}
\newcommand{\flag}[1]{\textit{#1}}
\newcommand{\dir}[1]{\texttt{#1}}
\newcommand{\file}[1]{\texttt{#1}}
\newcommand{\var}[1]{\texttt{#1}}
\newcommand{\rem}[1]{\texttt{\textit{#1}}}
\newcommand{\package}[1]{{\textit{#1}}}
\newcommand{\first}[2]{#1\index{#2}}
\newcommand{\error}[1]{\texttt{#1}}

% Colors
%%\definecolor{gray20}{gray}{0.20}
%%\newcommand{\lgray}[1]{{\color{gray20}#1}}
\newcommand{\pr}{\%}

\newenvironment{display}{\def\FrameCommand{\hskip\parindent}%%
\MakeFramed{\advance\hsize-\width\FrameRestore}%%
\vspace*{-2ex}\small\begin{alltt}}%
{\end{alltt}\vspace*{-2ex}\endMakeFramed}

\newenvironment{lbar}{%
\def\FrameCommand{\rightskip=\parindent\hskip\parindent\vrule width 1pt \hspace{10pt}}%
\MakeFramed{\rightskip=\parindent\advance\hsize-\width\FrameRestore\noindent\hskip-0.6ex}}%
{\endMakeFramed}

%% Footnotes
%%\renewcommand*{\thefootnote}{\textbf{\emph\MakeUppercase{\alph{footnote}}}}
\renewcommand*{\thefootnote}{\textbf{\emph\alph{footnote}}}

%% Epigraph
\newcommand{\epi}[2]{\epigraph{#1}{#2}}
\setlength{\epigraphwidth}{1.2\epigraphwidth}

%% a5paper
%% \setlength{\voffset}{-0.10\stockheight}	    % a5paper
%% \addtolength{\textheight}{0.12\stockheight}    % a5paper

%% Margin notes
\setmarginnotes{0.04\stockwidth}{0.12\stockwidth}{\onelineskip}
\newcommand{\gomarginpar}[1]{%
\marginpar{\rmfamily#1}}
\newcommand{\todo}[1]{%
\marginpar{{\sffamily{todo}}\\#1}}
%% typeset text in margin and index arg 2
\newcommand{\gomarginindex}[2]{%
\gomarginpar{#1}\index{#2}}

\newenvironment{cjk}{%
\begin{CJK*}{UTF8}{song}
\setCJKfamilyfont{Japanese}{Sazanami Gothic}
\CJKfamily{Japanese}
}%
{%
\end{CJK*}%
}
% start, end, text
\newcommand{\ubrace}[3]{%
\draw [thick,decorate,decoration={brace,amplitude=4pt},xshift=0pt,yshift=0pt] %
(#1) -- (#2) node [black,midway,below=4pt,xshift=0pt] %
{\longremark{#3}}; %
}
