\epi{""}
{\textit{}\\ \textsc{}}

\section{读者}
\noindent{}这是关于来自 Google 的 Go 语言的简介。
目标是为这个新的、革命性的语言提供一个指南。

这本书的目标读者是那些熟悉编程,并且了解多种编程语言,例如 C\cite{c},C++\cite{c++},
Perl \cite{perl},Java \cite{java},Erlang\cite{erlang},Scala\cite{scala},
Haskell\cite{haskell}。这\emph{不是}教你如何编程的书,只是教你如何使用 Go。

学习某样新东西,最佳的方式可能是通过建立自己的程序来探索它。
因此每章都包含了若干练习(和答案)让你熟悉这个语言。
练习标有编号 \textbf{Q$n$},而 $n$ 是一个数字。 
在练习编号后面的圆括号中指定了该题的难度。
难度范围从 0 到 9,0 是最简单,而 9 最难。
其后为了容易索引,提供了一个简短的名字。
例如:
\begin{verse}
\textbf{Q1}. (1) map 函数 \ldots
\end{verse}
展示了难度等级 1、编号 \textbf{Q1} 的关于 \func{map()} 函数的问题。
相关答案在练习的下一页。
答案的顺序和练习一致,而对于以 \textbf{A$n$} 开头的问题,对应编号 $n$ 的练习。

\section{内容布局}
第 \ref{chap:intro} 章提供了关于 Go 的简介和历史。
同时讨论了如何获得 Go 本身的代码。它假设使用了类 Unix 环境。

第 \ref{chap:basics} 章讨论了语言中可用的基本类型,变量和控制结构。

在第三章中了解了函数,这是 Go 程序中的基本部件。

在第 \ref{chap:packages} 章中,了解了在包中整合函数和数据。包提供了名字空间。

5:进阶,创建新的类型。

Go 并没有面向对象,它用接口作为替代。在第 \ref{chap:interfaces} 章将展示这是如何工作的。

通过 go 关键字,函数可以在不同的例程(叫做 goroutines)中执行。
通过 channel 来完成这些 goroutines 之间的通讯。

最后一章展示了如何用接口来完成 Go 程序的其他部分。如何创建、读取和写入文件。
同时也简要了解一下网络。
