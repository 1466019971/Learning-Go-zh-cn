\epi{``Go 是面向对象的语言吗?是也不是。''}
{\textit{FAQ}\\ \textsc{GO AUTHORS}}

\section{读者}
\noindent{}这是关于来自 Google 的 Go 语言的简介。
目标是为这个新的、革命性的语言提供一个指南。

这本书的目标读者是那些熟悉编程,并且了解某些编程语言,例如 C\cite{c},C++\cite{c++},Perl\cite{perl}, 
Java\cite{java},Erlang\cite{erlang},Scala\cite{scala},Haskell\cite{haskell}。
这\emph{不是}教你如何编程的书,只是教你如何使用 Go。

学习某样新东西,最佳的方式可能是通过建立自己的程序来探索它。
因此每章都包含了若干练习(和答案)让你熟悉这个语言。
练习标有编号 \textbf{Q$n$},而 $n$ 是一个数字。 
在练习编号后面的圆括号中指定了该题的难度。
难度范围从 0 到 9,0 是最简单,而 9 最难。
其后为了容易索引,提供了一个简短的名字。
例如:
\begin{verse}
\textbf{Q1}. (1) map 函数 \ldots
\end{verse}
展示了难度等级 1、编号 \textbf{Q1} 的关于 \func{map()} 函数的问题。
相关答案在练习的下一页。
答案的顺序和练习一致,而对于以 \textbf{A$n$} 开头的问题,对应编号 $n$ 的练习。
一些练习没有答案,它们被用星号标识出来。

\section*{内容布局}
\begin{description}
\item[第 \ref{chap:intro} 章:\titleref{chap:intro}]
提供了关于 Go 的简介和历史。同时讨论了如何获得 Go 本身的代码。
虽然 Go 完全可能在 Windows 平台上使用,但这里还是假设使用类 Unix 环境。

\item[第 \ref{chap:basics} 章:\titleref{chap:basics}]
讨论了语言中可用的基本类型,变量和控制结构。

\item[第 \ref{chap:functions} 章:\titleref{chap:functions}]
在第三章中了解了函数,这是 Go 程序中的基本部件。

\item[第 \ref{chap:packages} 章:\titleref{chap:packages}]
在第 \ref{chap:packages} 章中,会了解在包中整合函数和数据。
同时也将了解如何对包编写文档和进行测试。

\item[第 \ref{chap:beyond} 章:\titleref{chap:beyond}]
然后,在第 \ref{chap:beyond} 章中会看到如何创建自定义的类型。
同时也将了解 Go 中的内存分配。

\item[第 \ref{chap:interfaces} 章:\titleref{chap:interfaces}]
Go 不支持传统意义上的面向对象。在 Go 中接口是核心概念。

\item[第 \ref{chap:channels} 章:\titleref{chap:channels}]
通过 \func{go} 关键字,函数可以在不同的例程(叫做 goroutines)中执行。
通过 channel 来完成这些 goroutines 之间的通讯。

\item[第 \ref{chap:communication} 章:\titleref{chap:communication}]
最后一章展示了如何用接口来完成 Go 程序的其他部分。如何创建、读取和写入文件。
同时也简要了解一下网络。
\end{description}

希望你喜欢这本书,同时也喜欢 Go 语言。

\section{本书使用的设置}
\label{sec:settings used}
\begin{itemize}                            
\item Go 被安装在 \file{\~{}/go};
\item 希望编译的 Go 代码放在 \file{\~{}/g/src} 
而 \var{\$GOPATH} 设置为 \var{GOPATH=\~{}/g}。
\end{itemize}

\section*{翻译}
本书的内容可随意取用。这里已经有相关翻译:
\begin{itemize}
\item 中文,邢兴:\url{http://www.mikespook.com/learning-go/}
\item 俄文,\ldots
\end{itemize}

\begin{raggedright}
Miek Gieben, 2011 -- \url{miek@miek.nl}
邢兴, 2011 -- \url{mikespook@gmail.com}
\end{raggedright}
