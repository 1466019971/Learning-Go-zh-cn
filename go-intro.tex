\epi{``对此感兴趣,并且希望做点什么。''}
{\textit{在为~Go 添加复数支持时}\\ \textsc{KEN THOMPSON}}

\noindent{}什么是~Go?来自其网站\cite{go_web}的介绍:
\begin{quote}
Go 编程语言是一个使得程序员更加有效率的开源项目。Go 是有表达力、简洁、清晰和有效率的。
它的并行机制使其很容易编写多核和网络应用,而新奇的类型系统允许构建有弹性的模块化程序。
Go 编译到机器码非常快速,同时具有便利的垃圾回收和强大的运行时反射。
它是快速的、静态类型编译语言,但是感觉上是动态类型的,解释型语言。
\end{quote}

Go 1 是~Go 语言的第一个稳定发布版本。
本文档的所有练习都工作于~Go 1 -- 如果不能工作,那就一定是~bug。

本书使用了下面的约定:
\begin{itemize}
\item 代码、关键字和注释使用~\prog{Source Code Pro} 显示;
\item 代码中的额外标识~\coderemark{像这样显示};
\item 较长的标识提供数字 -- \gocircle{1} -- 详细解释在其后显示;
\item (如果需要)行号在右边显示;
\item shell 的例子使用~\pr{} 作为输入符;
\item 用户在~shell 输入内容的例子~\texttt{\user{用黑体显示}},系统反馈~\texttt{用普通的黑体显示};
\item 强调的段落会缩进,并在左边有竖线。
\end{itemize}

\section{官方文档}
Go 已经有大量的文档。
\gomarginpar{在互联网上搜索时,应当使用``golang''这个词来代替原始的``go''。}
例如~Go Tutorial \cite{go_tutorial} 和~Effective Go \cite{effective_go}。
网站~\url{http://golang.org/doc/} 也是绝佳的起点 
\footnote{\url{http://golang.org/doc/} 本身是由~\prog{go doc} 提供服务的。}。
虽然并不一定要阅读这些文档,但是强烈建议这么做。

Go 1 通过叫做~\prog{go doc}
的标准程序提供其文档。如果你想了解内建相关(参阅下一章
``\titleref{sec:builtins}''小节)的文档,可以像这样获取:
\begin{display}
\pr \user{go doc builtin}
\end{display}

在第~\ref{chap:packages} 章解释了如何构造你自己的包的文档。

\section{前身}
Go 的语法来自~Modula-2 系列语言。尤其是~Oberon 和~Modula-2(都是~N. Wirth 发明)的成功者~Oberon-2。
大致包括:借鉴了函数的符号;导入和导出的工作方式几乎完全一致(除了符号)。 
还有~Oberon 舍弃了独立的定义(“头”)文件,Go 也是一样。

结构体和接口的设计(参阅第~\ref{chap:interfaces} 章)
是从~Smalltalk 和类似语言,还有~Go 作者的个人试验语言的面向对象的工作方式鉴而来。
同时还基于~OO 界越来越广泛的共识:继承是有害的(但是通常会给予实现)。

使用~channel(参阅第~\ref{chap:channels} 章)与其他进程进行通讯,是由~A. R. Hoare \cite{hoare}提出的
叫做通讯序列化过程(CSP)的思路。他也是发明快排算法~\cite{quicksort} 的人。

如果对~Go 的历史进行更进一步的挖掘会发现其与``Newsqueak'' \cite{newsqueak} 的关联。
它是在~C 中能够像语言一样使用~channel 进行通讯的先驱。另一个重要的非类~C 语言
~Erlang \cite{erlang} 也会使用它们。

\begin{lbar}[]
Go 是第一个实现了简单的(或更加简单的)并行开发,且跨平台的类~C 语言。
\end{lbar}

\section{获得 Go}
在这一节中将介绍如何在本地设备上安装~Go。也可以在线上 \url{http://play.golang.org/} 编译~Go 代码,这是最为便捷的快速体验的方法。

也可以从网站\cite{go_install}获得已经编译好的二进制版本。

Ubuntu 和~Debian 在其仓库中都有~Go 包,可以查找``golang''包进行安装。
但是工作的时候可能会有一些小问题。所以这里仍然使用源码进行安装。

手工从~mercurial 中获取~Go 代码并且编译。对于其他类~Unix 系统,过程类似。
\begin{itemize}
\item 首先安装~Mercurial(获取~\prog{hg} 命令)。
在~Ubuntu/Debian/Fedora 需要安装~\prog{mercurial} 包;

\item 为了编译~Go 需要包:\prog{bison},\prog{gcc},\prog{libc6-dev},\prog{ed},\prog{gawk} 和\prog{make};

\item 设置环境变量~\prog{GOROOT} 作为~Go 的安装目录:
\begin{display}
\pr \user{export GOROOT=\~{}/go}
\end{display}

\item 然后获取~Go 最新的发布版(=Go 1)源代码:
\begin{display}
\pr \user{hg clone -r release https://go.googlecode.com/hg/ \$GOROOT}
\end{display}

\item 设置~PATH 指向到~Go 的二进制文件所在目录,这样就可以让~Shell 找到它们:
\begin{display}
\pr \user{export PATH=$GOROOT/bin:$PATH}
\end{display}
\item 编译~Go
\begin{display}
\pr \user{cd $GOROOT/src}
\pr \user{./all.bash}
\end{display}
\end{itemize}
如果全部都没问题,你最终会看到下面的内容:
\begin{display}
--- cd ../test
0 known bugs; 0 unexpected bugs

ALL TESTS PASSED

---
Installed Go for linux/amd64 in /home/go
Installed commands in /home/go/bin
\end{display}

\section{在~Windows 下获得~Go}
最好的方式仍然是遵循网站\cite{go_install}的介绍,为了方便重复如下。

\begin{itemize}
\item 下载~Go 1:
\url{http://code.google.com/p/go/downloads/list?q=OpSys-Windows+Type%3DArchive};
\item 解压缩到~\verb|C:\| 盘;
\item 确保内容在~\verb|C:\Go|。注意:在解压缩~zip 文件时,这个目录会被创建;
\item 添加~\verb|C:\Go\bin| 到~\$PATH:
\begin{display}
set PATH=\verb|%PATH%;C:\Go\bin|
\end{display}
\end{itemize}

\section{练习}
\begin{Exercise}[title={文档},difficulty=1]
\label{ex:doc}
\Question
Go 的文档可以通过~\prog{go doc} 程序阅读,它包含在~Go 的发布包中。

\prog{go doc hash} 给出了~\package{hash} 包的信息:
\vskip\baselineskip
\begin{display}
\pr \user{go doc hash}
PACKAGE

package hash

...
...
...

SUBDIRECTORIES

        adler32
        crc32
        crc64
        fnv

\end{display}
\vskip\baselineskip
哪个~\prog{go doc} 的命令可以显示~\package{hash} 包中的~\package{fnv} 文档?

\end{Exercise}

\begin{Answer}
\Question
\package{fnv} 包在~\package{hash} 的\emph{子目录}中,所以只需要 
\quad \texttt{go doc hash/fnv} 即可。


所有的内建函数同样可以通过~\prog{godoc} 程序访问:\prog{go doc builtin}。
\end{Answer}


\cleardoublepage
\section{答案}
\shipoutAnswer
